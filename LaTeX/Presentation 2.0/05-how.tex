\section{Результаты}
%------------------------------------------------
\begin{frame}
	\frametitle{Основные результаты работы}
	\begin{itemize}
		\item Изменение коэффициента проскальзывания в резонансной структуре повышает эффективность стохастического охлаждения, но эффекты ВПР делают предпочтительной регулярную структуру для экспериментов с тяжелыми ионами;
		\vspace{1em}
		\item Для коллайдерных экспериментов с поляризованными протонами резонансная структура повышает критическую энергию, искажая дисперсионную функцию;
		\vspace{1em}
		\item Численные исследования выявили нестабильность продольного фазового движения при прохождении критической энергии. Процедура скачка критической энергии помогает решить эту проблему. Экспериментальные данные с синхротрона У-70 согласуются с численными оценками, учитывающими высшие порядки разложения коэффициента уплотнения орбиты и импедансы для различных интенсивностей сгустка;
	\end{itemize}
\end{frame}
%------------------------------------------------
\begin{frame}
	\frametitle{Основные результаты работы}
	\begin{itemize}
		\item Скачок критической энергии ограничен его величиной и темпом изменения. Различия применения гармонического и барьерного ВЧ влияют на скачок. Оценки показывают ограничения на параметры сгустка из-за продольной микроволновой неустойчивости;
		\vspace{1em}
		\item Изучена спиновая динамика для измерения ЭДМ. Предложена концепция квази-замороженного спина с использованием метода фильтров Вина в кольце с обводными каналами;
		\vspace{1em}
		\item Модернизированный синхротрон Nuclotron сохраняет функцию бустера для коллайдера NICA. В 8/16-периодичных структурах возможны эксперименты по ЭДМ и поиску аксиона.
	\end{itemize}
\end{frame}
%------------------------------------------------
\begin{frame}
	\frametitle{Апробация работы}
	Основные результаты работы были представлены~на:
	\small
	\begin{itemize}
		\item 63, 65, 66-ая Всероссийская научная конференция МФТИ в 2020, 2023, 2024 гг. г. Долгопрудный,
		Россия;
		\item XXVII и XXVIII Всероссийская конференции по ускорителям заряженных частиц RuPAC'21, RuPAC'23. Алушта; Новосибирск, Россия.
		\item VII, VIII, IX и X Международная конференция Лазерные и Плазменные технологии ЛаПлаз'21, ЛаПлаз'22, ЛаПлаз'23, ЛаПлаз'24, ЛаПлас'25. Москва, Россия;
		\item XIII, XIV, XVI международная конференция по ускорителям заряженных частиц IPAC'22 IPAC'23, IPAC'25. Бангкок, Тайланд; Венеция, Италия; Тайпей, Тайвань;
		\item XIX Международная конференции по спиновой физике высоких энергий DSPIN'23. Дубна, Россия;
		\item XI-я Международная конференция по ядерной физике в накопительных кольцах STORI’24. Хуэйчжоу, провинция Гуандун, Китай;
	\end{itemize}
\end{frame}
%------------------------------------------------
\begin{frame}
	\frametitle{Публикации}
	Основные результаты по теме диссертации изложены в 17 печатных
	изданиях: 13 печатных работ изданы в журналах, рекомендованных
	ВАК, 13 статьей — в журналах, индексируемых международными
	базами цитирования Scopus и Web of Science.
	\small
	\begin{itemize}
		\item	1. Features of dual-purpose structure for heavy ion and light particles / S. D. Kolokolchikov, Y. V. Senichev, A. E. Aksentyev, A. A. Melnikov // Nuclear Science and Techniques. –– 2025. –– Т. 36, № 11. –– С. 210. –– URL:https://doi.org/10.1007/s41365-025-01791-4.
		\item	2. Formation of Polarized Proton Beams in the NICA Collider-Accelerator Complex / E. M. Syresin, A. V. Butenko, P. R. Zenkevich, O. S. Kozlov, S. D. Kolokolchikov, S. A. Kostromin, I. N. Meshkov, N. V. Mityanina, Y. V. Senichev, A. O. Sidorin, G. V. Trubnikov // Physics of Particles and Nuclei. –– 2021. –– Т. 52, № 5. –– С. 997––1017. –– URL: https://doi.org/10.1134/S1063779621050051.
		\item	3. Колокольчиков С.Д., Сеничев Ю.В. Магнито-оптическая Структура Коллайдера NICA c Высокой Критической Энергией. Яд. Физ. и Инж. том 13, номер 1, стр. 27-36 (2022). DOI: 10.56304/S2079562922010171
		\item	4. Колокольчиков С.Д., Сеничев Ю.В. Особенности Прохождения и Повышения Критической Энергии Синхротрона. Яд. Физ. и Инж. том 14, номер 6, стр. 587-592 (2023).  DOI: 10.56304/S2079562923010153
	\end{itemize}
\end{frame}
%------------------------------------------------
\begin{frame}
	\frametitle{Публикации}
	\small
	\begin{itemize}
		\item	5. Kolokolchikov S.D., Senichev. Y.V. \& Kalinin. V.A. Transition Energy Crossing in Harmonic RF at Proton Synchrotron U-70. Phys. Atom. Nuclei 87, 1355–1362 (2024). DOI: 10.1134/S106377882410020X
		\item	6. Kolokolchikov S.D. et al. Transition Energy Crossing in NICA Collider of Polarized Proton Beam in Harmonic and Barrier RF. Phys. Atom. Nuclei 87, 1449–1454 (2024). DOI: 10.1134/S1063778824100211
		\item	7. Kolokolchikov S. et al. Longitudinal Dynamic in NICA Barrier Bucket RF System at Transition Energy Including Impedances in BLonD. Phys. Part. Nuclei Lett. 21, 419–424 (2024). DOI: 10.1134/S1547477124700389
		\item	8. Kolokolchikov S., Acceleration and crossing of transition energy investigation using an RF structure of the barrier bucket type in the NICA accelerator complex. J.Phys.Conf.Ser. Vol. 2420, 012001 (2023). DOI: 10.1088/1742-6596/2420/1/012001
	\end{itemize}
\end{frame}
%------------------------------------------------
\begin{frame}
	\frametitle{Публикации}
	\small
	\begin{itemize}
		\item	9. Kolokolchikov S. et al. Transition Energy Crossing of Polarized Proton Beam at NICA. Phys. Atom. Nuclei 87, 212–215 (2024). DOI: 10.1134/S1063778824700054
		\item	10. Quasi-frozen spin concept of magneto-optical structure of NICA adapted to
		study the electric dipole moment of the deuteron and to search for the axion / Y. Senichev, A. Aksentyev, S. Kolokolchikov, A. Melnikov, V. Ladygin, E. Syresin, N. Nikolaev // Journal of Physics: Conference Series. –– 2023. –– Т. 2420, № 1. –– С. 012052. –– URL: https://dx.doi.org/10.1088/1742-6596/2420/1/012052.
		\item	11. Consideration of an Adapted Nuclotron Structure for Searching for the Electric Dipole Moment of Light Nuclei / Y. V. Senichev, A. E. Aksentyev, S. D. Kolokolchikov, A. A. Melnikov, V. P. Ladygin, E. M. Syresin // Physics of Atomic Nuclei. –– 2023. –– Т. 86, № 11. –– С. 2434––2438. –– URL: https://doi.org/10.1134/S1063778823110418.
		\item	12. (to be published) Kolokolchikov S.D., et al. Quasi-frozen spin for both deuteron and proton beam at periodic EDM storage ring lattice, Nucl.Sci. and Tech.

	\end{itemize}
\end{frame}
%------------------------------------------------
\begin{frame}
	\frametitle{Публикации}
	\small
	\begin{itemize}
	\item	13. Колокольчиков С.Д. и др. Проектирование Каналов Bypass в Ускорительном Комплексе NICA для Экспериментов с Поляризованными Пучками по Поиску ЭДМ. Яд. Физ. и Инж. том 15, номер 5, стр. 457-463 (2024). DOI: 10.56304/S2079562924050257
	\item	14. S. Kolokolchikov et al. ByPass optics design in NICA storage ring for experiment with polarized beams for EDM search. J.Phys.Conf.Ser. Vol. 2687, 022026 (2024). DOI: 10.1088/1742-6596/2687/2/022026
	\item	15. NICA Facilities for the Search for EDM Light Nuclei / Y. Senichev, A. Aksentyev, S. Kolokolchikov, A. Melnikov, N. Nikolaev, V. Ladygin, E. Syresin // Physics of Atomic Nuclei. –– 2024. –– Т. 87, № 4. –– С. 436––441. –– URL: https://doi.org/10.1134/S1063778824700534.
	\item	16. Kolokolchikov S. et al. Spin Coherence and Betatron Chromaticity of Deuteron Beam in “Quasi-Frozen” Spin Regime. Phys. Atom. Nuclei 86, 2684–2688 (2023). DOI: 10.1134/S106377882311025X
	\item	17. S. Kolokolchikov et al. Spin coherence and betatron chromaticity of deuteron beam in NICA storage ring. J.Phys.Conf.Ser. Vol. 2687, 022027 (2024). DOI: 10.1088/1742-6596/2687/2/022027
	
	\end{itemize}
\end{frame}
%------------------------------------------------