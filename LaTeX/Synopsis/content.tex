\pdfbookmark{Общая характеристика работы}{characteristic}             % Закладка pdf
\section*{Общая характеристика работы}

\newcommand{\actuality}{\pdfbookmark[1]{Актуальность темы и степень её разработанности}{actuality}\underline{\textbf{\actualityTXT}}}
\newcommand{\progress}{\pdfbookmark[1]{Разработанность темы}{progress}\underline{\textbf{\progressTXT}}}
\newcommand{\aim}{\pdfbookmark[1]{Цели}{aim}\underline{{\textbf\aimTXT}}}
\newcommand{\tasks}{\pdfbookmark[1]{Задачи}{tasks}\underline{\textbf{\tasksTXT}}}
\newcommand{\aimtasks}{\pdfbookmark[1]{Цели и задачи}{aimtasks}\aimtasksTXT}
\newcommand{\novelty}{\pdfbookmark[1]{Научная новизна}{novelty}\underline{\textbf{\noveltyTXT}}}
\newcommand{\influence}{\pdfbookmark[1]{Практическая значимость}{influence}\underline{\textbf{\influenceTXT}}}
\newcommand{\methods}{\pdfbookmark[1]{Методология и методы исследования}{methods}\underline{\textbf{\methodsTXT}}}
\newcommand{\defpositions}{\pdfbookmark[1]{Положения, выносимые на защиту}{defpositions}\underline{\textbf{\defpositionsTXT}}}
\newcommand{\reliability}{\pdfbookmark[1]{Достоверность}{reliability}\underline{\textbf{\reliabilityTXT}}}
\newcommand{\probation}{\pdfbookmark[1]{Апробация}{probation}\underline{\textbf{\probationTXT}}}
\newcommand{\contribution}{\pdfbookmark[1]{Личный вклад}{contribution}\underline{\textbf{\contributionTXT}}}
\newcommand{\publications}{\pdfbookmark[1]{Публикации}{publications}\underline{\textbf{\publicationsTXT}}}

\par Данная работа посвящена исследованию динамики поляризованных пучков в ускорителях и накопителях. Также будут разобраны вопросы проектирования современных ускорительных установок.

\par Возможность использования ускорительный установки для различных экспериментов является большим преимуществом. Такая практика применяется в крупных ядерных центра CERN, RHIC. Последовательные программы экспериментов расписаны на годы вперед. Такие установки отвечают в первую очередь фундаментальным исследованиям, но и привносят за собой необходимые технологии для полноценного развития научно-технической базы.

\par NICA является передовым центром, расположенным в России, город Дубна. Коллайдер NICA, имеет 2 места встречи, в которых расположены детектора: MPD(Multi-Purpose Detector) и SPD(Spin Polarized Detector). Каждый из них предназначен для разных экспериментов. MPD-детектор – будет использован для исследования кварк-гюонной плазмы, возникающей в результате столкновений тяжелых ионов золота. SPD-детектор направлен на изучение поведения сталкивающихся поляризованных пучков протонов и дейтронов. Таким образом, структура коллайдера должна быть использована как для ускорения пучков тяжелых ионов, так и легких. При этом требования, предъявляемые для удержания пучка для разного сорта частиц, отличаются. 

\par Основным требованием коллайдерных экспериментов, является достижение большого количества соударений, то есть высокого уровня светимости. Для исследования кварк-глюонной плазмы это требование должно быть на уровне $10^{27}$ $cm^{-2}s^{-1}$. Такие светимости являются рекордными и для их достижения может потребоваться существенной настройки всех система ускорителя и может занять достаточно большого времени. При ускорение тяжелых ионов высокая зарядность и интенсивность пучка вызывает серьезные ограничения на параметры пучка из-за внутрипучкового рассеяния. Для преодоления этих проблем, спроектированная структура должна высоким временем внутрипучкого рассеяния, а также содержать специальные установки стохастического и электронного охлаждения. Стохастическое охлаждение также в существенной степени зависит от конкретной оптики установки и может быть оптимизировано для компенсации эффектов ВПР. Электронное охлаждение применяется на небольших энергиях сгутска и способно охладить пучок на начальных этапах ускорения.

\par В том же кольце могут быть ускорены и другие частицы. Подготовка и ускорение поляризованных пучков для экспериментов на детекторе SPD представляет особый интерес, поскольку поляризация является дополнительной степенью свободы и может привнести дополнительную информацию, в том числе в коллайдерные эксперименты. В этом случае определенные сечения рассеяния приобретают зависимость от поляризации сталкивающихся сгустков.

\par Соотношение заряда к массе для протона отличается по сравнению с тяжелыми ионами почти в два раза. Таким образом, максимальная энергия эксперимента кратно увеличивается. Но для существующей магнитооптики, оптимальной для тяжелоионного эксперимента подобрано значение критическое энергии таким образом, что столкновение происходит до критического значения и никаких проблем по её преодолению не возникает. Стоит отметить, что критическая энергия является важным параметром ускорительный установки и при проектировании установки этому вопросу уделяется особое внимание. Долгое нахождение вблизи критической энергии или её пересечение существенно влияет на динамику пучка и его стабильность. Таким образом, для протонов прохождение критической энергии становится важным параметром, ограничивающем параметры сгустка и требующем принятия дополнительных мер по её преодолению.

\par Классическим методом преодоления является процедура скачка критической энергии. При этом изменяются параметры ускорителя для внесения соответствующего возмущения и резкого кратковременного скачка критической энергии в момент близости энергии сгустка к критическому значению. После скачка, значения установки возвращаются к исходному значению до скачка с поправкой на увеличившуюся энергию пучка. Однако, сложностью является непосредственное создание скачка с заданной величиной и темпом, что не всегда легко реализуемо.

\par Альтернативным способом, который применяется для того чтобы избегать потери стабильности, является создание или модификация структуры с заведомо большим значение критической энергии. Такая структура носит название 'резонансной' и уже применялась на установках мирового уровня CERN, J-PARC. Принципиальным отличием от регулярной структуры является обеспечение резонансного условия для количества суперпериодов и частоты бетатроных колебаний в горизонтальной плоскости. Однако, это справедливо только для не полностью регулярных структур, а содержащих регулярную модуляцию градиента квадруполей или кривизны орбиты. В таком случае, происходит изменение оптических функций ускорителя и варьирование критической энергии выше энергии эксперимента, в том числе до комплексных значений, полностью убирая зависимость установки от дополнительных процедур преодоления.

\par Отдельным большим направлением, помимо коллайдерных экспериментов, является управление поляризацией. Спин является квантовой величиной, но в силу теоремы Эренфеста для любой квантовой величины может быть записано уравнение в квази-классическом приближении для ансамбля частиц. Поведение спина частицы в ансамбле описывается уравнением Т-БМТ. Проекция спинов частиц на заданную ось и определяет поляризацию пучка. Для таких экспериментов интерес представляет долгое сохранение поляризации пучка, что может быть использовано и реализовано в накопительных установках 

\par Более тонким направлением исследований, являются не просто поляризованные пучки, а также когерентные. В этом случае, пучок становится не просто поляризованным вдоль конкретной оси, но и спины частиц прецессируют с одинаковой частотой. В таком случае появляется возможность исследовать также ЭДМ элементарных частиц. Данная величина характеризует асимметрию распределения заряда частицы. Наличие ЭДМ объясняется тем, что он нарушает CP-симметрию, последнее было предсказано Сахаровым как одно из условий бариогинеза на ранних этапах вселенной. Для накопления малой величины ЭДМ необходимо долгое удержание пучка с последующим анализом на поляриметре рассеяния. При этом влияние МДМ должно быть подавлено. Такая техника впервые была предложена в BNL и имеет название 'замороженный' спин. Позднее, была предложена концепция 'квази-замороженного' спина, в которой происходит пространственное разделение полей и интегральное подавление МДМ-компоненты за полный оборот по кольцу.

\par Представленные исследования исходят из возможности изучения в комплексе NICA-Nuclotron. Построенный ускорительный комплекс является проектом мегасайнс и оборудован передовой материально-технической базой, отвечающей мировым тенденциях в ускорительной технике. Основными функционирующими установками помимо уже упомянутого коллайдера NICA являются бустер тяжелых ионов Booster, а также синхротон Nuclotron.

\par В коллайдере NICA для реализации концепции "квази-замороженного" спина необходима установка соответствующего оборудования. Для реализации накопительного кольца из структуры коллайдера, необходима модернизация с созданием обходных каналов bypass. Таким образом, на полученных прямолинейных участках могут быть расположены прямые фильтры Вина, выполняющий функцию компенсации МДМ-компоненты в скрещенных магнитных и электрических полях, не возмущающие орбиту в силу равенства нулю силы Лоренца.

\par Nuclotron является бустером поляризованных частиц в коллайдер, однако, требующем модернизации. Соответствующей концепт модернизации рассмотрен с точки зрения использования Nuclotron в тесной связке с коллайдером NICA.
Использвание Nuclotron для полноценных спиновых экспериментов делает эту машину столь же интересной, сколько и отдельные программы на коллайдере. 
Кроме того, особенности магнитооптики Nuclotron открывают возможность измерение ЭДМ не только дейтрона, но и протона, однако, при несколько меньшей энергии. На текущий день измерений ЭДМ как дейтрона, так и протона не было осуществлено и представляется передним краем физического эксперимента на ускорительной установке.

\par Ещё одним направлением исследований в рамках формирующейся программы спиновой физике является исследование аксиона. В этом случае резонансным методом между частотой спиной прецессии и частотой осциллирующего скалярного аксионного поля может быть получена масса аксиона или получено ограничение. Для этого ускоритель будет использован в роли зондирующей антенны по частоте прецессии спина.

~\\
\par {\actuality} Исследования направлены на формирование полноценной физической программы по исследованию спиновой динамике в комплексе NICA-Nuclotron.
~\\
\par {\aim} данной диссертации является изучение 
особенностей динамики поляризованного пучка в ускорительном 
комплексе NICA-Nuclotron с учетом возможной модернизации 
магнитооптической структуры комплекса для проведения коллайдерных экспериментов, а также исследования 
электрического дипольного момента и поиска аксиона.
Для достижения поставленной цели необходимо было 
решить следующие {\tasks}:

\begin{enumerate}[beginpenalty=10000] % https://tex.stackexchange.com/a/476052/104425
  \item Моделирование магнитооптики с модулированной дисперсионной функцией;
  \item Расчёт времени внутрипучкового рассеяния для тяжелых ионов;
  \item Оценка влияния методов охлаждения пучка на время жизни;
  \item Проведение численного моделирования продольной динамики частиц с учетом высших порядков коэффициента уплотнения орбиты в высокочастотых резонаторах гармонического и барьерного типа;
  \item Обеспечение стабильности пучка с точки зрения динамической апертуры при процедуре скачка критической энергии, подавление хроматичности, компенсация нелинейных эффектов;
  \item Сохранение поляризации пучка при совершении процедуры скачка критической энергии;
  \item Проектирование кольцевого ускорителя с возможностью применения метода «квази-замороженного спина»;
  \item Спин-орбитальное моделирование в магнитном кольце с дополнительными элементами со скрещенными магнитными и электрическими полями;
\end{enumerate}
~\\
\par {\novelty}
\begin{enumerate}[beginpenalty=10000] % https://tex.stackexchange.com/a/476052/104425
    \item	Исследована возможность проектирования дуальной магнитооптической структуры. Оптимизированой с точки зрения времени жизни пучка для тяжелых ионов и возможностью вариации критической энергии для легких частиц;
  \item 	Применен метод проектирования "резонансной" магнитооптической структуры с варьированной критической энергией для обеспечения стабильности пучка;
  \item	Исследована продольная динамика поляризованного пучка при нахождении вблизи и прохождении критической энергии скачком в ВЧ ;
   \item	Исследована процедура скачка критической энергии экспериментально на сеансе синхротрона У-70, а также при помощи численного моделирования для различных импедансов и интенсивностей пучка;
  \item	Разработка альтернативных прямых секций, путем создания обходных каналов bypass для реализации метода «квази-замороженного» cпина с установленными прямыми фильтрами Вина для возможности изучения ЭДМ дейтронов в накопительном кольце NICA;
  \item	Модернизация кольца канала Nuclotron с укорочением поворотными магнитными арками для возможности создания режима «квази-замороженного» спина и изучения ЭДМ дейтрона и протона;
  \item	Изучение спин-орбитальной динамики в предложенных структурах. Исследование природы спиновой декогеренции в структуре коллайдера NICA.
\end{enumerate}
~\\
\par {\influence}:
\par Разработка дуальной магнитооптической структуры может позволит использовать кольцо коллайдера как для коллайдерных экспериментов с тяжелыми ионами на MPD детекторе с целью исследования кварк-глюонной плазмы, так и для проведения коллайдерных экспериментов по столкновению легких ядер на SPD детекторе.
\par Определение оптимальных параметров скачка критической энергии, а также его влияние на динамику сгустка. 
\par Создание обводных каналов bypass позволит избежать точек встречи, также расположить прямые фильтры Вина независимо от оборудования, используемого для тяжело-ионного эксперимента. В конечном счёте, 
это позволит использовать NICA в режиме накопительного кольца.
\par Наличие ЭДМ заряженных частиц может быть установлено с использованием ускорительных установок в качестве накопительного кольца. Такие исследования является отдельной частью программы спиновой физики, которая формируется на установке NICA-Nuclotron.
\par Модернизация кольца Nuclotron рассматривается в двух аспектах. Во-первых, использование в качестве бустера для поляризованного пучка в коллайдер. Во-вторых, для независимого эксперимента по исследованию ЭДМ и поиску аксиона.

% {\progress}
% Этот раздел должен быть отдельным структурным элементом по
% ГОСТ, но он, как правило, включается в описание актуальности
% темы. Нужен он отдельным структурынм элемементом или нет ---
% смотрите другие диссертации вашего совета, скорее всего не нужен.
~\\
\par {\methods} Основными методами исследования являются математическое и компьютерное моделирование, численный эксперимент. Для исследования поперечной динамики: MAD-X, OPTIM, продольной динамики: BLonD; спин-орбитальной динамики: COSY Infinity.
~\\
%второй вариант
\begin{comment}
\par {\defpositions}
\begin{enumerate}[beginpenalty=10000] % https://tex.stackexchange.com/a/476052/104425
  \item 	Принципы построения дуальной магнитооптической структуры с оптимизированным временем жизни пучка в регулярной структуре для многозарядных тяжелых ионов и варьированной критической энергией в резонансной структуре для легких ядер; \cite{Kolokolchikov:2025_dual}, \cite{Syresin:2021_polar}
  \item	Результаты, полученные в эксперименте на У-70 и в методе численного моделирования динамики продольного движения вблизи критической энергии с учётом влияния высших порядков зависимости от разброса по импульсу и с учетом импеданса; \cite{Kolokolchikov:2025_U70}, \cite{Kolokolchikov:2025_jump}
  \item	Результаты исследования продольной динамики поляризованного пучка для процедуры скачка критической энергии в гармоническом и барьерном ВЧ, оценка влияния продольной микроволновой неустойчивости; \cite{Kolokolchikov:2024_bb_rupac}, \cite{Kolokolchikov:2023_bb_IPAC}, \cite{Kolokolchikov:2024_bb_dspin}
  \item	Метод подавления дисперсии и влияния нелинейных эффектов, из-за нарушения периодичности за счет введения missing magnet на краях поворотных арок, для создания резонансной магнитооптической структуры; \cite{Kolokolchikov:2021trans}, \cite{Kolokolchikov:2023_pecular}
  \item	Модернизированная структура с квази-замороженным спином для исследования ЭДМ дейтронов и протонов и возможностью совместного использования Нуклотрона в качестве бустера поляризованных частиц для коллайдера; \cite{Senichev:2023_QFS}, \cite{Senichev:2023_nuclotron}, \cite{Kolokolchikov:2025_nuclotron}
  \item	Метод обводных каналов bypass для независимого исследования ЭДМ в кольце коллайдера;\cite{Kolokolchikov:2023_bypass}, \cite{Kolokolchikov:2023_bypass_IPAC}, \cite{Senichev:2024_nica_edm}, \cite{Kolokolchikov:2023_sc}, \cite{Kolokolchikov:2023_sc_IPAC}
\end{enumerate}
\end{comment}

%третий вариант
\par {\defpositions}
\begin{enumerate}[beginpenalty=10000] % https://tex.stackexchange.com/a/476052/104425
  \item 	Изучение внутрипучкового рассеяния и стохастического охлаждения для оптимизации времени жизни пучка в регулярной структуре для многозарядных тяжелых ионов и варьированной критической энергией в резонансной структуре для легких ядер с целью реализации дуальности ускорительной установки; \cite{Kolokolchikov:2025_dual}, \cite{Syresin:2021_polar}
  \item	Результаты, полученные в эксперименте на У-70 и в методе численного моделирования динамики продольного движения вблизи критической энергии с учётом влияния высших порядков зависимости от разброса по импульсу и с учетом импеданса; \cite{Kolokolchikov:2025_U70}, \cite{Kolokolchikov:2025_jump}
  \item	Результаты исследования продольной динамики поляризованного пучка для процедуры скачка критической энергии в гармоническом и барьерном ВЧ, оценка влияния продольной микроволновой неустойчивости; \cite{Kolokolchikov:2024_bb_rupac}, \cite{Kolokolchikov:2023_bb_IPAC}, \cite{Kolokolchikov:2024_bb_dspin}
  \item	Метод подавления дисперсии и влияния нелинейных эффектов в резонансной магнитооптической структуре из-за нарушения периодичности по дисперсии за счет missing magnet на краях поворотных арок; \cite{Kolokolchikov:2021trans}, \cite{Kolokolchikov:2023_pecular}
  \item	Модернизированная структура с квази-замороженным спином для исследования ЭДМ дейтронов и протонов и возможностью совместного использования Нуклотрона в качестве бустера поляризованных частиц для коллайдера; \cite{Senichev:2023_QFS}, \cite{Senichev:2023_nuclotron}, \cite{Kolokolchikov:2025_nuclotron}
  \item	Метод введения обводных каналов в кольцо синхротрона для создания независимой установки с возможностью проведения прецизионных экспериментов, в том числе изучения ЭДМ элементарных частиц;\cite{Kolokolchikov:2023_bypass}, \cite{Kolokolchikov:2023_bypass_IPAC}, \cite{Senichev:2024_nica_edm}, \cite{Kolokolchikov:2023_sc}, \cite{Kolokolchikov:2023_sc_IPAC}
\end{enumerate}

~\\
\par {\reliability} полученных результатов подтверждается согласованием аналитических вычислений с результатами численных экспериментов. Результаты находятся в соответствии с результатами, полученными другими авторами.
~\\
\par {\probation}
Основные результаты работы докладывались~на российских и международных конференциях: 
\begin{itemize}
\item Молодежная конференция по теоретической и экспериментальной физике МКТЭФ-2020. Москва, Россия;
\item 63, 65, 66-ая Всероссийская научная конференция МФТИ в 2020, 2023, 2024 гг. г. Долгопрудный,
Россия;
\item XXVII и XXVIII Всероссийская конференции по ускорителям заряженных частиц RuPAC'21, RuPAC'23. Алушта; Новосибирск, Россия.
\item VII, VIII, IX и X Международная конференция Лазерные и Плазменные технологии ЛаПлаз'21, ЛаПлаз'22, ЛаПлаз'23, ЛаПлаз'24. Москва, Россия;
\item XIII и XIV международная конференция по ускорителям заряженных частиц IPAC'22 IPAC'23. Бангкок, Тайланд; Венеция, Италия;
\item XIX Международная конференции по спиновой физике высоких энергий DSPIN'23. Дубна, Россия;
\item XI-я Международная конференция по ядерной физике в накопительных кольцах STORI’24. Хуэйчжоу, провинция Гуандун, Китай.
\end{itemize}
~\\
\par {\contribution} Все результаты, выносимые на защиту, получены автором лично, либо при его непосредственном участии. Содержание диссертации и выносимые на защиту основные положения отражают личный вклад автора в опубликованные работы. Результаты по подготовке и проведению эксперимента на ускорителе У-70 получены в соавторстве с сотрудниками ИЯИ РАН и ИФВЭ. Подготовка к публикации полученных результатов проводилась совместно с соавторами.
~\\
\par \ifnumequal{\value{bibliosel}}{0}
{%%% Встроенная реализация с загрузкой файла через движок bibtex8. (При желании, внутри можно использовать обычные ссылки, наподобие `\cite{vakbib1,vakbib2}`).
 {\publications} Основные результаты по теме диссертации изложены
    в~XX~печатных изданиях,
    X из которых изданы в журналах, рекомендованных ВАК,
    X "--- в тезисах докладов.
}%
{%%% Реализация пакетом biblatex через движок biber
    \begin{refsection}[bl-author, bl-registered]
        % Это refsection=1.
        % Процитированные здесь работы:
        %  * подсчитываются, для автоматического составления фразы "Основные результаты ..."
        %  * попадают в авторскую библиографию, при usefootcite==0 и стиле `\insertbiblioauthor` или `\insertbiblioauthorgrouped`
        %  * нумеруются там в зависимости от порядка команд `\printbibliography` в этом разделе.
        %  * при использовании `\insertbiblioauthorgrouped`, порядок команд `\printbibliography` в нём должен быть тем же (см. biblio/biblatex.tex)
        %
        % Невидимый библиографический список для подсчёта количества публикаций:
        \printbibliography[heading=nobibheading, section=1, env=countauthorvak,          keyword=biblioauthorvak]%
        \printbibliography[heading=nobibheading, section=1, env=countauthorwos,          keyword=biblioauthorwos]%
        \printbibliography[heading=nobibheading, section=1, env=countauthorscopus,       keyword=biblioauthorscopus]%
        \printbibliography[heading=nobibheading, section=1, env=countauthorconf,         keyword=biblioauthorconf]%
        \printbibliography[heading=nobibheading, section=1, env=countauthorother,        keyword=biblioauthorother]%
        \printbibliography[heading=nobibheading, section=1, env=countregistered,         keyword=biblioregistered]%
        \printbibliography[heading=nobibheading, section=1, env=countauthorpatent,       keyword=biblioauthorpatent]%
        \printbibliography[heading=nobibheading, section=1, env=countauthorprogram,      keyword=biblioauthorprogram]%
        \printbibliography[heading=nobibheading, section=1, env=countauthor,             keyword=biblioauthor]%
        \printbibliography[heading=nobibheading, section=1, env=countauthorvakscopuswos, filter=vakscopuswos]%
        \printbibliography[heading=nobibheading, section=1, env=countauthorscopuswos,    filter=scopuswos]%
        %
        \nocite{*}%
        %
        {\publications} Основные результаты по теме диссертации изложены в~\arabic{citeauthor}~печатных изданиях,
        \arabic{citeauthorvak} из которых изданы в журналах, рекомендованных ВАК\sloppy%
        \ifnum \value{citeauthorscopuswos}>0%
            , \arabic{citeauthorscopuswos} "--- в~периодических научных журналах, индексируемых Web of~Science и Scopus\sloppy%
        \fi%
        \ifnum \value{citeauthorconf}>0%
            , \arabic{citeauthorconf} "--- в~тезисах докладов.
        \else%
            .
        \fi%
        \ifnum \value{citeregistered}=1%
            \ifnum \value{citeauthorpatent}=1%
                Зарегистрирован \arabic{citeauthorpatent} патент.
            \fi%
            \ifnum \value{citeauthorprogram}=1%
                Зарегистрирована \arabic{citeauthorprogram} программа для ЭВМ.
            \fi%
        \fi%
        \ifnum \value{citeregistered}>1%
            Зарегистрированы\ %
            \ifnum \value{citeauthorpatent}>0%
            \formbytotal{citeauthorpatent}{патент}{}{а}{}\sloppy%
            \ifnum \value{citeauthorprogram}=0 . \else \ и~\fi%
            \fi%
            \ifnum \value{citeauthorprogram}>0%
            \formbytotal{citeauthorprogram}{программ}{а}{ы}{} для ЭВМ.
            \fi%
        \fi%
        % К публикациям, в которых излагаются основные научные результаты диссертации на соискание учёной
        % степени, в рецензируемых изданиях приравниваются патенты на изобретения, патенты (свидетельства) на
        % полезную модель, патенты на промышленный образец, патенты на селекционные достижения, свидетельства
        % на программу для электронных вычислительных машин, базу данных, топологию интегральных микросхем,
        % зарегистрированные в установленном порядке.(в ред. Постановления Правительства РФ от 21.04.2016 N 335)
    \end{refsection}%
    \begin{refsection}[bl-author, bl-registered]
        % Это refsection=2.
        % Процитированные здесь работы:
        %  * попадают в авторскую библиографию, при usefootcite==0 и стиле `\insertbiblioauthorimportant`.
        %  * ни на что не влияют в противном случае
        \nocite{vakbib2}%vak
        \nocite{patbib1}%patent
        \nocite{progbib1}%program
        \nocite{bib1}%other
        \nocite{confbib1}%conf
    \end{refsection}%
        %
        % Всё, что вне этих двух refsection, это refsection=0,
        %  * для диссертации - это нормальные ссылки, попадающие в обычную библиографию
        %  * для автореферата:
        %     * при usefootcite==0, ссылка корректно сработает только для источника из `external.bib`. Для своих работ --- напечатает "[0]" (и даже Warning не вылезет).
        %     * при usefootcite==1, ссылка сработает нормально. В авторской библиографии будут только процитированные в refsection=0 работы.
}


 % Характеристика работы по структуре во введении и в автореферате не отличается (ГОСТ Р 7.0.11, пункты 5.3.1 и 9.2.1), потому её загружаем из одного и того же внешнего файла, предварительно задав форму выделения некоторым параметрам

%Диссертационная работа была выполнена при поддержке грантов \dots

%\underline{\textbf{Объем и структура работы.}} Диссертация состоит из~введения,
%четырех глав, заключения и~приложения. Полный объем диссертации
%\textbf{ХХХ}~страниц текста с~\textbf{ХХ}~рисунками и~5~таблицами. Список
%литературы содержит \textbf{ХХX}~наименование.

\pdfbookmark{Содержание работы}{description}                          % Закладка pdf
\section*{Содержание работы}
Во \underline{\textbf{введении}} обосновывается актуальность
исследований, проводимых в~рамках данной диссертационной работы,
приводится обзор научной литературы по~изучаемой проблеме,
формулируется цель, ставятся задачи работы, излагается научная новизна
и практическая значимость представляемой работы. В~последующих главах
сначала описываются особенности дуальной магнитооптической структуры, а~потом идёт более детальное рассмотрение вариации дисперсионной функции в резонансных структурах, методов преодоления критической энергии и в конце возможность проведения прецизионных экспериментов по изучению ЭДМ элементарных заряженных частиц и поиску аксионоподобных частиц.

В \underline{\textbf{первой главе}}: особое внимание уделено процессам внутрипучкового рассеяния и наличию критической энергии, влияющие на динамику многозарядных тяжёлых ионов и лёгких ядер. С этой целью рассматривается дуальная магнитооптическая структура, способная адаптироваться для целей обоих типов экспериментов.
\par В случае тяжелых ионов зарядность выделяет проблему внутрипучкового рассеяния пучка на первый план. Разогрев пучка приводит к росту поперечного эмиттанса и продольного разброса по импульсам. Для предотвращения неконтролируемого роста фазового объёма применяются техники по охлаждению пучка. Рассматривается стабильность пучка с точки зрения времени жизни пучка, в стационарном, независимом, от времени случае параметры пучка при наличии внутрипучкового рассеяния и охлаждения определяются как

\[
    \begin{aligned}
& \varepsilon_{\textrm{st}}=\left.\tau_{\textrm{tr}} \cdot\left(\dv{\varepsilon}{t}\right)_{\textrm{IBS}}\right|_{\varepsilon=\varepsilon_{\textrm{st}}}, \\
& \delta_{s t}^2=\left.\tau_{\text {long }} \cdot\left(\dv{\delta^2}{t}\right)_{\textrm{IBS}}\right|_{\delta^2=\delta_{\textrm{st}}^2}.
\end{aligned}
\]

\noindent В современных установках используется как стохастическое, так и электронное охлаждение. Использование стохастического охлаждения оказывается зависимо от продольного смещения частиц относительно референсной и такого параметра как коэффициента скольжения $\eta$. Использование 'резонансных' структур с варьируемым значением коэффициента уплотнения орбиты способно уменьшить время охлаждения до оптимального значения в случае 'комбинированной' структуры, где одна поворотная арка с комплексным значением критической энергии, а другая с действительными. Однако, соответствующее внесение изменений в оптику установки может приводить к росту ВПР и в конечном счёте не скомпенсировать его.

\begin{figure}[ht]
    \centerfloat{
        \hfill
        \subcaptionbox{Зависимость времени стохастического охлаждения от энергии.}{%
        	    \includegraphics[scale=0.35]{1_SC_common}}
        \hfill
        \subcaptionbox{Зависимость постоянной времени разогрева пучка из-за внутрипучкового рассеяния.}{%
            \includegraphics[scale=0.45]{1_IBS}}
        \hfill
    }
    \caption{Сравнение времени разогрева пучка и охлаждения. Черная линия – 'регулярная' , синяя – 'резонансная', красная – 'комбинированная' структура, прерывистая – идеальный случай.}
\end{figure}

\par Для легких частиц, таких как протоны, соотношение заряда к массе отличается почти в 2 раза по сравнению с тяжелыми ионами, таким образом пропорционально увеличивается и энергия эксперимента. При этом критическая энергия остается неизменной, поскольку является характеристикой конкретной установки и определяется магнитооптикой. Преодоление критической энергии является необходимым для обеспечения стабильности, в первую очередь, продольного движения. Таким образом, для тяжелых ионов такой проблемы не возникает, а в случае легких частиц, требуется принимать меры по преодолению критической энергии. Одним из таких методов может является создание 'резонансной' структуры. 

\par Во \underline{\textbf{второй главе}} проведён учёт влияния высших порядков разброса по импульсам и моделей продольных импедансов при пересечении критической энергии. Также рассмотрен метод скачка критической энергии для различных ускоряющих потенциалов с целью сохранения стабильности сгустка. Для этого произведено математическое моделирование процесса, описываемого уравнением продольного движения:
\[
	\begin{cases}
		\begin{aligned}
			& \dv{\tau}{t}=\eta(\delta) \cdot \frac{h \cdot \Delta E}{\beta^2 \cdot E_0}, \\
			& \dv{(\Delta E)}{t}=\frac{V(\tau)}{T_0}.
		\end{aligned}
	\end{cases}
	\label{eq:long_motion_eq_t}
\]

\noindent Как видно, уравнение зависит от параметров магнитооптической структуры, ускоряющей станции, энергии пучка, а также разброса по импульсам внутри сгустка.

\par Применение различных типов ВЧ оказывает существенное влияние на динамику пучка. В зависимости от используемого типа изменяется темп ускорения, а также вид удерживающей сепаратрисы. В случае гармонического ВЧ, ускорения происходит смещением фазы равновесной частицы и в разы большее, чем в случае индукционного ускорения при использовании барьерной станции.

\par Для преодоления критической энергии классически используется процедура скачка критической энергии. Это достигается путем модулирования дисперсионной функции при приближении энергии пучка к значению критической энергии. Данные численного моделирования, также апробированы на экспериментальной установке У-70 в Протвино. Также рассмотрены эффекты влияния высших порядков коэффициента расширения орбиты и простейших моделей импедансов на динамику пучка.

\begin{figure}[h]
    \centerfloat{
        \hfill
        \subcaptionbox{Скачкообразное изменение критической энергии.}{%
        	    \includegraphics[scale=0.75]{3_g_tr_BB}}
        \hfill
        \subcaptionbox{Скачкообразное измерение первого порядка коэффициента проскальзывания.}{%
            \includegraphics[scale=0.75]{3_eta_tr_BB.png}}
        \hfill
    }
    \caption{Процедура скачка критической энергии для барьерного ВЧ.}\label{fig:latex}
\end{figure}

\par Существенное ограничение на параметры сгустка возникают из-за продольной микроволновой неустойчивости вблизи критической энергии. В конечном счёте это ограничивает величину светимости коллайдерного эксперимента.

\par Было показано, что для процедуры скачка критической энергии ключевыми являются темп изменения критической энергии по отношению к темпу ускорения от ВЧ станции, а также максимально возможная величина изменения критический энергии во время процедуры.

В \underline{\textbf{третьей главе}} рассматривается метод вариации критической энергии путем модуляции дисперсионной функции в резонансной структуре. Для этого может вводится как суперпериодическая модуляция градиентов квадрупольных линз, так и модуляция кривизны орбиты, тем самым изменяя коэффициент сжатия орбиты, который напрямую связан с критической энергий ускорителя

\[
\alpha=\frac{1}{{\gamma_{\text{tr}}}^2}=\frac{1}{C}\int_{0}^{C}\frac{D\left(s\right)}{\rho\left(s\right)}ds.
\label{eq:alpha}
\]

\begin{figure}[ht]
    \centerfloat{
        \hfill
        \subcaptionbox{Регулярный}{%
        	    \includegraphics[scale=0.25]{2_twiss_3FODO_regular}}
        \hfill
        \subcaptionbox{Модулированный}{%
            \includegraphics[scale=0.32]{2_Twiss_Superperiod}}
        \hfill
    }
    \caption{Твисс-параметры для различных суперпериодов.}\label{fig:latex}
\end{figure}

\par Для 'регулярной' магнитооптической структуры коллайдера NICA рассмотрены варианты модернизации для создания 'резонансной' структуры с поднятой критической энергией. Поскольку установка рассматривалась как стационарная, то это возможно только путем модуляции градиентов в квадрупольных линзах. Для 12 ФОДО ячеек на поворотной в NICA может быть реализовано 4 суперпериода с набегом частоты бетатронных колебаний равным 3. Тем самым реализуя резонансное условие, необходимое для модуляции дисперсионной функции.

\par Поскольку исходная 'регулярная' структура имеет на концах поворотных арок подавители дисперсии в виде 'missing magnet', то это естественным образом должно быть учтено и в подавлении дисперсии для модифицированной структуры. Поэтому рассмотрены схемы подавления дисперсии, оно может быть осуществлено как квадруполями в двух крайних ФОДО ячейками, так и при использовании только двух семейств квадруполей. 

\par Рассмотрен вопрос подавления натуральной хроматичности, а также нелинейных эффектов в таких структурах. Представлены схемы расстановки секступолей в рассмотренных оптиках.

В \underline{\textbf{четвертой главе}} рассматривается возможность изучения электрического дипольного момента легких заряженных частиц. Исследовано применение концепции "квази-замороженного" спина для накопительных колец. Рассматривается возможность модернизации колец с сохранением текущего предназначения  и расширением исследовательских возможностей установок.
\par Изучена спиновая динамика в кольце с использованием электростатических, а также элементов с совмещенной функцией, что по­казано на рис. \ref{fig:QFS}.

\begin{figure}[ht]
    \centerfloat{
        \hfill
        \subcaptionbox{С использованием электростатических дефлекторов.}{%
        	    \includegraphics[scale=0.35]{4_deflector_deutron.png}}
        \hfill
        \subcaptionbox{С использованием фильтров Вина.}{%
            \includegraphics[scale=0.35]{4_arc_B+E_WF.png}}
        \hfill
    }
    \caption{Принципиальная схема "квази-замороженной" структуры.}\label{fig:QFS}
\end{figure}

\par Для проведения эксперимента по поиску ЭДМ становится необходимым использовать альтернативный метод управления спином, концепция «квази-замороженного» спина. В отличие от метода «замороженного» спина, спин больше не сохраняет ориентацию в течение всего периода обращения, а восстанавливает на прямолинейном участке. Это возможно благодаря использованию элементов как с электрическим, так и с магнитным полями, которые называются фильтрами Вина, на прямом участке. Поворот спина в арке на определенный угол компенсируется соответствующим поворотом в фильтре Вина. Поля подбираются таким образом, чтобы создать нулевую силу Лоренца и не нарушить прямолинейность орбиты. Поляриметры, расположенные после фильтров Вина, будут обнаруживать ту же ориентацию спин-вектора, и для них она будет 'заморожена'.

\par Приведена структура NICA c обводными каналами bypass для реализации накопительного кольца с фильтрами Вина на прямых участках, без вмешательства в текущую оптику коллайдера. Для обеспечение высокого показателя время когерентности SCT (Spin Coherence Time), порядка 1000 секунд возможно использовать главное кольцо NICA в качестве накопителя, а не в режиме коллайдера. По этой причине, предлагается установить дополнительные обводные каналы bypass. Таким образом, можно создать совершенно новую регулярную структуру, что является большим преимуществом, не требующей значительной перестройки комплекса и затрат, при всём при этом, позволит задействовать NICA в различных экспериментах.

\par Текущая структура синхротрона Nuclotron не предполагает программу исследований ЭДМ. Для расширения возможностей Nuclotron в качестве самостоятельной машины рассматривается возможность модернизации. Наибольший интерес может представлять структура, способная одновременно быть использована для изучения ЭДМ как дейтронов, так и протонов. С точки зрения орбитальной динамики протон и дейтрон практически идентичны, масса дейтрона, вдвое больше, чем у протона. Спиновая же динамика отличается достаточно существенно для разного сорта частиц. 

\noindent Рассмотрена «квази-замороженная» структура с электростатическими дефлекторами и фильтрами Вина. Показано, что для компенсации отклонения спина в магнитной арке, должны быть использованы элементы, отклоняющие на одинаковый угол, то есть с одинаковой кривизной как электрического, так и магнитного полей. При этом тип отклоняющего элемента не имеет значения, это может быть как фильтр Вина, так и электростатический дефлектор. Таким образом, при неизменной магнитной арке, длина фильтра Вина окажется меньше на суммарную длину киккеров, так как в нём совмещены функции электростатического дефлектора и киккера в один элемент. Отдельно для протонов показано, длина компенсирующих элементов больше длины магнитной арки. И для исследования протонов может быть использована та же структура, но с повёрнутыми на 180 градусов фильтрами Вина при меньшей энергии \cite{Kolokolchikov:2021trans}.

\FloatBarrier
\pdfbookmark{Заключение}{conclusion}                                  % Закладка pdf

В \underline{\textbf{заключении}} приведены основные результаты работы, которые заключаются в следующем:
%% Согласно ГОСТ Р 7.0.11-2011:
%% 5.3.3 В заключении диссертации излагают итоги выполненного исследования, рекомендации, перспективы дальнейшей разработки темы.
%% 9.2.3 В заключении автореферата диссертации излагают итоги данного исследования, рекомендации и перспективы дальнейшей разработки темы.
\begin{enumerate}
  \item На основе анализа внутрипучкового рассеяния, а также стохастического охлаждения показано, что использование метода 'резонансной' структуры способно увеличить эффективность стохастического охлаждения. Особенно эффективным может быть использование 'комбинированной' структуры. Однако, эффекты ВПР для приведенных структуры оказались в несколько раз большими и в конечном счёте недостаточными, делая предпочтительной 'регулярную' структуру для тяжелоионного эксперимента.
  \item Для коллайдерных экспериментов с протонами рассмотрена 'резонансная' структура в варьированой критическая энергия, что использовано для рассмотрения адаптированной структуры коллайдера NICA.
  \item Численные исследования показали, что прохождение критической энергии может вызывать нестабильность продольного фазового движения. Использование процедуры скачка критической энергии может быть использовано для преодоления этой проблемы. Получены экспериментальные данные процедуры скачка критической с синхротрона У-70, которые находятся в соответствии с проведенным численными оценками с учетом высших порядков разложения коэффициента уплотнения орбиты и импедансов для различных интенсивностей сгустка.
  \item Использование процедуры скачка для коллайдера NICA ограничено величиной скачка критической энергии, а также для гармонического ВЧ темпом изменения критической энергии по сравнению с темпом ускорения пучка. Что делает невозможным использование процедуры для этого типа ВЧ. Для барьерного ВЧ приведены оценки продольной микроволновой неустойчивости, показывающие существенное ограничение на параметры конечного сгустка.
  \item Для исследования спиновой динамики и реализации "квази-замороженного" спина в коллайдере NICA рассмотрено введение обводных каналов bypass. На прямых участках предлагается расположение фильтров Вина для компенсации поворота спина под действием МДМ в магнитной арке.
  \item Рассмотрена модернизированная структура синхротрона Nuclotron с сохранением функции бустера поляризованного пучка в коллайдер NICA. В предложенных 8/16-периодичных структурах возможно проведение независимых прецезионных экспериментов по исследованию ЭДМ дейтрона и протона, а также осуществлению поика аксиона в режиме сканирующей антенны.
\end{enumerate}


\begin{comment}
\ifdefmacro{\microtypesetup}{\microtypesetup{protrusion=false}}{} % не рекомендуется применять пакет микротипографики к автоматически генерируемому списку литературы
\ifnumequal{\value{bibliosel}}{0}{% Встроенная реализация с загрузкой файла через движок bibtex8
	\renewcommand{\bibname}{\large \authorbibtitle}
	\nocite{*}
	\insertbiblioauthor           % Подключаем Bib-базы
	%\insertbiblioother   % !!! bibtex не умеет работать с несколькими библиографиями !!!
}{% Реализация пакетом biblatex через движок biber
	\ifnumgreater{\value{usefootcite}}{0}{
		%  \nocite{*} % Невидимая цитата всех работ, позволит вывести все работы автора
		\insertbiblioauthorcited      % Вывод процитированных в автореферате работ автора
	}{
		\insertbiblioauthor           % Вывод всех работ автора
		%  \insertbiblioauthorgrouped    % Вывод всех работ автора, сгруппированных по источникам
		%  \insertbiblioauthorimportant  % Вывод наиболее значимых работ автора (определяется в файле characteristic во второй section)
		\insertbiblioother            % Вывод списка литературы, на которую ссылались в тексте автореферата
	}
}
\ifdefmacro{\microtypesetup}{\microtypesetup{protrusion=true}}{}
\end{comment}

%\begin{comment}
\ifdefmacro{\microtypesetup}{\microtypesetup{protrusion=false}}{} % не рекомендуется применять пакет микротипографики к автоматически генерируемому списку литературы
\urlstyle{rm}                               % ссылки URL обычным шрифтом
\ifnumequal{\value{bibliosel}}{0}{% Встроенная реализация с загрузкой файла через движок bibtex8
	\renewcommand{\bibname}{\large \bibtitleauthor}
	\nocite{*}
	\insertbiblioauthor           % Подключаем Bib-базы
	%\insertbiblioexternal   % !!! bibtex не умеет работать с несколькими библиографиями !!!
}{% Реализация пакетом biblatex через движок biber
	% Цитирования.
	%  * Порядок перечисления определяет порядок в библиографии (только внутри подраздела, если `\insertbiblioauthorgrouped`).
	%  * Если не соблюдать порядок "как для \printbibliography", нумерация в `\insertbiblioauthor` будет кривой.
	%  * Если цитировать каждый источник отдельной командой --- найти некоторые ошибки будет проще.
	%
	%% authorvak
	%\nocite{vakbib1}%
	%\nocite{vakbib2}%
	%
	%% authorwos
	%\nocite{wosbib1}%
	%
	%% authorscopus
	%\nocite{scbib1}%
	%
	%% authorpathent
	%\nocite{patbib1}%
	%
	%% authorprogram
	%\nocite{progbib1}%
	%
	%% authorconf
	%\nocite{confbib1}%
	%\nocite{confbib2}%
	%
	%% authorother
	%\nocite{bib1}%
	%\nocite{bib2}%
	
	\ifnumgreater{\value{usefootcite}}{0}{
		\begin{refcontext}[labelprefix={}]
			\ifnum \value{bibgrouped}>0
			\insertbiblioauthorgrouped    % Вывод всех работ автора, сгруппированных по источникам
			\else
			\insertbiblioauthor      % Вывод всех работ автора
			\fi
		\end{refcontext}
	}{
		\ifnum \totvalue{citeexternal}>0
		\begin{refcontext}[labelprefix=A]
			\ifnum \value{bibgrouped}>0
			\insertbiblioauthorgrouped    % Вывод всех работ автора, сгруппированных по источникам
			\else
			\insertbiblioauthor      % Вывод всех работ автора
			\fi
		\end{refcontext}
		\else
		\ifnum \value{bibgrouped}>0
		\insertbiblioauthorgrouped    % Вывод всех работ автора, сгруппированных по источникам
		\else
		\insertbiblioauthor      % Вывод всех работ автора
		\fi
		\fi
		%  \insertbiblioauthorimportant  % Вывод наиболее значимых работ автора (определяется в файле characteristic во второй section)
		\begin{refcontext}[labelprefix={}]
			\insertbiblioexternal            % Вывод списка литературы, на которую ссылались в тексте автореферата
		\end{refcontext}
		% Невидимый библиографический список для подсчёта количества внешних публикаций
		% Используется, чтобы убрать приставку "А" у работ автора, если в автореферате нет
		% цитирований внешних источников.
		\printbibliography[heading=nobibheading, section=0, env=countexternal, keyword=biblioexternal, resetnumbers=true]%
	}
}
\ifdefmacro{\microtypesetup}{\microtypesetup{protrusion=true}}{}
\urlstyle{tt}                               % возвращаем установки шрифта ссылок URL
%\end{comment}