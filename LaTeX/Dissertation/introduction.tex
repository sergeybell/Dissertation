\chapter*{Введение}                         % Заголовок
\addcontentsline{toc}{chapter}{Введение}    % Добавляем его в оглавление

%\newcommand{\actuality}{}
\newcommand{\progress}{}
\newcommand{\aim}{{\textbf\aimTXT}}
\newcommand{\tasks}{\textbf{\tasksTXT}}
\newcommand{\novelty}{\textbf{\noveltyTXT}}
\newcommand{\influence}{\textbf{\influenceTXT}}
\newcommand{\methods}{\textbf{\methodsTXT}}
\newcommand{\defpositions}{\textbf{\defpositionsTXT}}
\newcommand{\reliability}{\textbf{\reliabilityTXT}}
\newcommand{\probation}{\textbf{\probationTXT}}
\newcommand{\contribution}{\textbf{\contributionTXT}}
\newcommand{\publications}{\textbf{\publicationsTXT}}

\par Данная работа посвящена исследованию динамики поляризованных пучков в ускорителях и накопителях. Также будут разобраны вопросы проектирования современных ускорительных установок.

\par Возможность использования ускорительный установки для различных экспериментов является большим преимуществом. Такая практика применяется в крупных ядерных центра CERN, RHIC. Последовательные программы экспериментов расписаны на годы вперед. Такие установки отвечают в первую очередь фундаментальным исследованиям, но и привносят за собой необходимые технологии для полноценного развития научно-технической базы.

\par NICA является передовым центром, расположенным в России, город Дубна. Коллайдер NICA, имеет 2 места встречи, в которых расположены детектора: MPD(Multi-Purpose Detector) и SPD(Spin Polarized Detector). Каждый из них предназначен для разных экспериментов. MPD-детектор – будет использован для исследования кварк-гюонной плазмы, возникающей в результате столкновений тяжелых ионов золота. SPD-детектор направлен на изучение поведения сталкивающихся поляризованных пучков протонов и дейтронов. Таким образом, структура коллайдера должна быть использована как для ускорения пучков тяжелых ионов, так и легких. При этом требования, предъявляемые для удержания пучка для разного сорта частиц, отличаются. 

\par Основным требованием коллайдерных экспериментов, является достижение большого количества соударений, то есть высокого уровня светимости. Для исследования кварк-глюонной плазмы это требование должно быть на уровне $10^{27}$ $cm^{-2}s^{-1}$. Такие светимости являются рекордными и для их достижения может потребоваться существенной настройки всех система ускорителя и может занять достаточно большого времени. При ускорение тяжелых ионов высокая зарядность и интенсивность пучка вызывает серьезные ограничения на параметры пучка из-за внутрипучкового рассеяния. Для преодоления этих проблем, спроектированная структура должна высоким временем внутрипучкого рассеяния, а также содержать специальные установки стохастического и электронного охлаждения. Стохастическое охлаждение также в существенной степени зависит от конкретной оптики установки и может быть оптимизировано для компенсации эффектов ВПР. Электронное охлаждение применяется на небольших энергиях сгутска и способно охладить пучок на начальных этапах ускорения.

\par В том же кольце могут быть ускорены и другие частицы. Подготовка и ускорение поляризованных пучков для экспериментов на детекторе SPD представляет особый интерес, поскольку поляризация является дополнительной степенью свободы и может привнести дополнительную информацию, в том числе в коллайдерные эксперименты. В этом случае определенные сечения рассеяния приобретают зависимость от поляризации сталкивающихся сгустков.

\par Соотношение заряда к массе для протона отличается по сравнению с тяжелыми ионами почти в два раза. Таким образом, максимальная энергия эксперимента кратно увеличивается. Но для существующей магнитооптики, оптимальной для тяжелоионного эксперимента подобрано значение критическое энергии таким образом, что столкновение происходит до критического значения и никаких проблем по её преодолению не возникает. Стоит отметить, что критическая энергия является важным параметром ускорительный установки и при проектировании установки этому вопросу уделяется особое внимание. Долгое нахождение вблизи критической энергии или её пересечение существенно влияет на динамику пучка и его стабильность. Таким образом, для протонов прохождение критической энергии становится важным параметром, ограничивающем параметры сгустка и требующем принятия дополнительных мер по её преодолению.

\par Классическим методом преодоления является процедура скачка критической энергии. При этом изменяются параметры ускорителя для внесения соответствующего возмущения и резкого кратковременного скачка критической энергии в момент близости энергии сгустка к критическому значению. После скачка, значения установки возвращаются к исходному значению до скачка с поправкой на увеличившуюся энергию пучка. Однако, сложностью является непосредственное создание скачка с заданной величиной и темпом, что не всегда легко реализуемо.

\par Альтернативным способом, который применяется для того чтобы избегать потери стабильности, является создание или модификация структуры с заведомо большим значение критической энергии. Такая структура носит название 'резонансной' и уже применялась на установках мирового уровня CERN, J-PARC. Принципиальным отличием от регулярной структуры является обеспечение резонансного условия для количества суперпериодов и частоты бетатроных колебаний в горизонтальной плоскости. Однако, это справедливо только для не полностью регулярных структур, а содержащих регулярную модуляцию градиента квадруполей или кривизны орбиты. В таком случае, происходит изменение оптических функций ускорителя и варьирование критической энергии выше энергии эксперимента, в том числе до комплексных значений, полностью убирая зависимость установки от дополнительных процедур преодоления.

\par Отдельным большим направлением, помимо коллайдерных экспериментов, является управление поляризацией. Спин является квантовой величиной, но в силу теоремы Эренфеста для любой квантовой величины может быть записано уравнение в квази-классическом приближении для ансамбля частиц. Поведение спина частицы в ансамбле описывается уравнением Т-БМТ. Проекция спинов частиц на заданную ось и определяет поляризацию пучка. Для таких экспериментов интерес представляет долгое сохранение поляризации пучка, что может быть использовано и реализовано в накопительных установках 

\par Более тонким направлением исследований, являются не просто поляризованные пучки, а также когерентные. В этом случае, пучок становится не просто поляризованным вдоль конкретной оси, но и спины частиц прецессируют с одинаковой частотой. В таком случае появляется возможность исследовать также ЭДМ элементарных частиц. Данная величина характеризует асимметрию распределения заряда частицы. Наличие ЭДМ объясняется тем, что он нарушает CP-симметрию, последнее было предсказано Сахаровым как одно из условий бариогинеза на ранних этапах вселенной. Для накопления малой величины ЭДМ необходимо долгое удержание пучка с последующим анализом на поляриметре рассеяния. При этом влияние МДМ должно быть подавлено. Такая техника впервые была предложена в BNL и имеет название 'замороженный' спин. Позднее, была предложена концепция 'квази-замороженного' спина, в которой происходит пространственное разделение полей и интегральное подавление МДМ-компоненты за полный оборот по кольцу.

\par Представленные исследования исходят из возможности изучения в комплексе NICA-Nuclotron. Построенный ускорительный комплекс является проектом мегасайнс и оборудован передовой материально-технической базой, отвечающей мировым тенденциях в ускорительной технике. Основными функционирующими установками помимо уже упомянутого коллайдера NICA являются бустер тяжелых ионов Booster, а также синхротон Nuclotron.

\par В коллайдере NICA для реализации концепции "квази-замороженного" спина необходима установка соответствующего оборудования. Для реализации накопительного кольца из структуры коллайдера, необходима модернизация с созданием обходных каналов bypass. Таким образом, на полученных прямолинейных участках могут быть расположены прямые фильтры Вина, выполняющий функцию компенсации МДМ-компоненты в скрещенных магнитных и электрических полях, не возмущающие орбиту в силу равенства нулю силы Лоренца.

\par Nuclotron является бустером поляризованных частиц в коллайдер, однако, требующем модернизации. Соответствующей концепт модернизации рассмотрен с точки зрения использования Nuclotron в тесной связке с коллайдером NICA.
Использвание Nuclotron для полноценных спиновых экспериментов делает эту машину столь же интересной, сколько и отдельные программы на коллайдере. 
Кроме того, особенности магнитооптики Nuclotron открывают возможность измерение ЭДМ не только дейтрона, но и протона, однако, при несколько меньшей энергии. На текущий день измерений ЭДМ как дейтрона, так и протона не было осуществлено и представляется передним краем физического эксперимента на ускорительной установке.

\par Ещё одним направлением исследований в рамках формирующейся программы спиновой физике является исследование аксиона. В этом случае резонансным методом между частотой спиной прецессии и частотой осциллирующего скалярного аксионного поля может быть получена масса аксиона или получено ограничение. Для этого ускоритель будет использован в роли зондирующей антенны по частоте прецессии спина.

~\\
\par {\actuality} Исследования направлены на формирование полноценной физической программы по исследованию спиновой динамике в комплексе NICA-Nuclotron.
~\\
\par {\aim} данной диссертации является изучение 
особенностей динамики поляризованного пучка в ускорительном 
комплексе NICA-Nuclotron с учетом возможной модернизации 
магнитооптической структуры комплекса для проведения коллайдерных экспериментов, а также исследования 
электрического дипольного момента и поиска аксиона.
Для достижения поставленной цели необходимо было 
решить следующие {\tasks}:

\begin{enumerate}[beginpenalty=10000] % https://tex.stackexchange.com/a/476052/104425
  \item Моделирование магнитооптики с модулированной дисперсионной функцией;
  \item Расчёт времени внутрипучкового рассеяния для тяжелых ионов;
  \item Оценка влияния методов охлаждения пучка на время жизни;
  \item Проведение численного моделирования продольной динамики частиц с учетом высших порядков коэффициента уплотнения орбиты в высокочастотых резонаторах гармонического и барьерного типа;
  \item Обеспечение стабильности пучка с точки зрения динамической апертуры при процедуре скачка критической энергии, подавление хроматичности, компенсация нелинейных эффектов;
  \item Сохранение поляризации пучка при совершении процедуры скачка критической энергии;
  \item Проектирование кольцевого ускорителя с возможностью применения метода «квази-замороженного спина»;
  \item Спин-орбитальное моделирование в магнитном кольце с дополнительными элементами со скрещенными магнитными и электрическими полями;
\end{enumerate}
~\\
\par {\novelty}
\begin{enumerate}[beginpenalty=10000] % https://tex.stackexchange.com/a/476052/104425
    \item	Исследована возможность проектирования дуальной магнитооптической структуры. Оптимизированой с точки зрения времени жизни пучка для тяжелых ионов и возможностью вариации критической энергии для легких частиц;
  \item 	Применен метод проектирования "резонансной" магнитооптической структуры с варьированной критической энергией для обеспечения стабильности пучка;
  \item	Исследована продольная динамика поляризованного пучка при нахождении вблизи и прохождении критической энергии скачком в ВЧ ;
   \item	Исследована процедура скачка критической энергии экспериментально на сеансе синхротрона У-70, а также при помощи численного моделирования для различных импедансов и интенсивностей пучка;
  \item	Разработка альтернативных прямых секций, путем создания обходных каналов bypass для реализации метода «квази-замороженного» cпина с установленными прямыми фильтрами Вина для возможности изучения ЭДМ дейтронов в накопительном кольце NICA;
  \item	Модернизация кольца канала Nuclotron с укорочением поворотными магнитными арками для возможности создания режима «квази-замороженного» спина и изучения ЭДМ дейтрона и протона;
  \item	Изучение спин-орбитальной динамики в предложенных структурах. Исследование природы спиновой декогеренции в структуре коллайдера NICA.
\end{enumerate}
~\\
\par {\influence}:
\par Разработка дуальной магнитооптической структуры может позволит использовать кольцо коллайдера как для коллайдерных экспериментов с тяжелыми ионами на MPD детекторе с целью исследования кварк-глюонной плазмы, так и для проведения коллайдерных экспериментов по столкновению легких ядер на SPD детекторе.
\par Определение оптимальных параметров скачка критической энергии, а также его влияние на динамику сгустка. 
\par Создание обводных каналов bypass позволит избежать точек встречи, также расположить прямые фильтры Вина независимо от оборудования, используемого для тяжело-ионного эксперимента. В конечном счёте, 
это позволит использовать NICA в режиме накопительного кольца.
\par Наличие ЭДМ заряженных частиц может быть установлено с использованием ускорительных установок в качестве накопительного кольца. Такие исследования является отдельной частью программы спиновой физики, которая формируется на установке NICA-Nuclotron.
\par Модернизация кольца Nuclotron рассматривается в двух аспектах. Во-первых, использование в качестве бустера для поляризованного пучка в коллайдер. Во-вторых, для независимого эксперимента по исследованию ЭДМ и поиску аксиона.

% {\progress}
% Этот раздел должен быть отдельным структурным элементом по
% ГОСТ, но он, как правило, включается в описание актуальности
% темы. Нужен он отдельным структурынм элемементом или нет ---
% смотрите другие диссертации вашего совета, скорее всего не нужен.
~\\
\par {\methods} Основными методами исследования являются математическое и компьютерное моделирование, численный эксперимент. Для исследования поперечной динамики: MAD-X, OPTIM, продольной динамики: BLonD; спин-орбитальной динамики: COSY Infinity.
~\\
%второй вариант
\begin{comment}
\par {\defpositions}
\begin{enumerate}[beginpenalty=10000] % https://tex.stackexchange.com/a/476052/104425
  \item 	Принципы построения дуальной магнитооптической структуры с оптимизированным временем жизни пучка в регулярной структуре для многозарядных тяжелых ионов и варьированной критической энергией в резонансной структуре для легких ядер; \cite{Kolokolchikov:2025_dual}, \cite{Syresin:2021_polar}
  \item	Результаты, полученные в эксперименте на У-70 и в методе численного моделирования динамики продольного движения вблизи критической энергии с учётом влияния высших порядков зависимости от разброса по импульсу и с учетом импеданса; \cite{Kolokolchikov:2025_U70}, \cite{Kolokolchikov:2025_jump}
  \item	Результаты исследования продольной динамики поляризованного пучка для процедуры скачка критической энергии в гармоническом и барьерном ВЧ, оценка влияния продольной микроволновой неустойчивости; \cite{Kolokolchikov:2024_bb_rupac}, \cite{Kolokolchikov:2023_bb_IPAC}, \cite{Kolokolchikov:2024_bb_dspin}
  \item	Метод подавления дисперсии и влияния нелинейных эффектов, из-за нарушения периодичности за счет введения missing magnet на краях поворотных арок, для создания резонансной магнитооптической структуры; \cite{Kolokolchikov:2021trans}, \cite{Kolokolchikov:2023_pecular}
  \item	Модернизированная структура с квази-замороженным спином для исследования ЭДМ дейтронов и протонов и возможностью совместного использования Нуклотрона в качестве бустера поляризованных частиц для коллайдера; \cite{Senichev:2023_QFS}, \cite{Senichev:2023_nuclotron}, \cite{Kolokolchikov:2025_nuclotron}
  \item	Метод обводных каналов bypass для независимого исследования ЭДМ в кольце коллайдера;\cite{Kolokolchikov:2023_bypass}, \cite{Kolokolchikov:2023_bypass_IPAC}, \cite{Senichev:2024_nica_edm}, \cite{Kolokolchikov:2023_sc}, \cite{Kolokolchikov:2023_sc_IPAC}
\end{enumerate}
\end{comment}

%третий вариант
\par {\defpositions}
\begin{enumerate}[beginpenalty=10000] % https://tex.stackexchange.com/a/476052/104425
  \item 	Изучение внутрипучкового рассеяния и стохастического охлаждения для оптимизации времени жизни пучка в регулярной структуре для многозарядных тяжелых ионов и варьированной критической энергией в резонансной структуре для легких ядер с целью реализации дуальности ускорительной установки; \cite{Kolokolchikov:2025_dual}, \cite{Syresin:2021_polar}
  \item	Результаты, полученные в эксперименте на У-70 и в методе численного моделирования динамики продольного движения вблизи критической энергии с учётом влияния высших порядков зависимости от разброса по импульсу и с учетом импеданса; \cite{Kolokolchikov:2025_U70}, \cite{Kolokolchikov:2025_jump}
  \item	Результаты исследования продольной динамики поляризованного пучка для процедуры скачка критической энергии в гармоническом и барьерном ВЧ, оценка влияния продольной микроволновой неустойчивости; \cite{Kolokolchikov:2024_bb_rupac}, \cite{Kolokolchikov:2023_bb_IPAC}, \cite{Kolokolchikov:2024_bb_dspin}
  \item	Метод подавления дисперсии и влияния нелинейных эффектов в резонансной магнитооптической структуре из-за нарушения периодичности по дисперсии за счет missing magnet на краях поворотных арок; \cite{Kolokolchikov:2021trans}, \cite{Kolokolchikov:2023_pecular}
  \item	Модернизированная структура с квази-замороженным спином для исследования ЭДМ дейтронов и протонов и возможностью совместного использования Нуклотрона в качестве бустера поляризованных частиц для коллайдера; \cite{Senichev:2023_QFS}, \cite{Senichev:2023_nuclotron}, \cite{Kolokolchikov:2025_nuclotron}
  \item	Метод введения обводных каналов в кольцо синхротрона для создания независимой установки с возможностью проведения прецизионных экспериментов, в том числе изучения ЭДМ элементарных частиц;\cite{Kolokolchikov:2023_bypass}, \cite{Kolokolchikov:2023_bypass_IPAC}, \cite{Senichev:2024_nica_edm}, \cite{Kolokolchikov:2023_sc}, \cite{Kolokolchikov:2023_sc_IPAC}
\end{enumerate}

~\\
\par {\reliability} полученных результатов подтверждается согласованием аналитических вычислений с результатами численных экспериментов. Результаты находятся в соответствии с результатами, полученными другими авторами.
~\\
\par {\probation}
Основные результаты работы докладывались~на российских и международных конференциях: 
\begin{itemize}
\item Молодежная конференция по теоретической и экспериментальной физике МКТЭФ-2020. Москва, Россия;
\item 63, 65, 66-ая Всероссийская научная конференция МФТИ в 2020, 2023, 2024 гг. г. Долгопрудный,
Россия;
\item XXVII и XXVIII Всероссийская конференции по ускорителям заряженных частиц RuPAC'21, RuPAC'23. Алушта; Новосибирск, Россия.
\item VII, VIII, IX и X Международная конференция Лазерные и Плазменные технологии ЛаПлаз'21, ЛаПлаз'22, ЛаПлаз'23, ЛаПлаз'24. Москва, Россия;
\item XIII и XIV международная конференция по ускорителям заряженных частиц IPAC'22 IPAC'23. Бангкок, Тайланд; Венеция, Италия;
\item XIX Международная конференции по спиновой физике высоких энергий DSPIN'23. Дубна, Россия;
\item XI-я Международная конференция по ядерной физике в накопительных кольцах STORI’24. Хуэйчжоу, провинция Гуандун, Китай.
\end{itemize}
~\\
\par {\contribution} Все результаты, выносимые на защиту, получены автором лично, либо при его непосредственном участии. Содержание диссертации и выносимые на защиту основные положения отражают личный вклад автора в опубликованные работы. Результаты по подготовке и проведению эксперимента на ускорителе У-70 получены в соавторстве с сотрудниками ИЯИ РАН и ИФВЭ. Подготовка к публикации полученных результатов проводилась совместно с соавторами.
~\\
\par \ifnumequal{\value{bibliosel}}{0}
{%%% Встроенная реализация с загрузкой файла через движок bibtex8. (При желании, внутри можно использовать обычные ссылки, наподобие `\cite{vakbib1,vakbib2}`).
 {\publications} Основные результаты по теме диссертации изложены
    в~XX~печатных изданиях,
    X из которых изданы в журналах, рекомендованных ВАК,
    X "--- в тезисах докладов.
}%
{%%% Реализация пакетом biblatex через движок biber
    \begin{refsection}[bl-author, bl-registered]
        % Это refsection=1.
        % Процитированные здесь работы:
        %  * подсчитываются, для автоматического составления фразы "Основные результаты ..."
        %  * попадают в авторскую библиографию, при usefootcite==0 и стиле `\insertbiblioauthor` или `\insertbiblioauthorgrouped`
        %  * нумеруются там в зависимости от порядка команд `\printbibliography` в этом разделе.
        %  * при использовании `\insertbiblioauthorgrouped`, порядок команд `\printbibliography` в нём должен быть тем же (см. biblio/biblatex.tex)
        %
        % Невидимый библиографический список для подсчёта количества публикаций:
        \printbibliography[heading=nobibheading, section=1, env=countauthorvak,          keyword=biblioauthorvak]%
        \printbibliography[heading=nobibheading, section=1, env=countauthorwos,          keyword=biblioauthorwos]%
        \printbibliography[heading=nobibheading, section=1, env=countauthorscopus,       keyword=biblioauthorscopus]%
        \printbibliography[heading=nobibheading, section=1, env=countauthorconf,         keyword=biblioauthorconf]%
        \printbibliography[heading=nobibheading, section=1, env=countauthorother,        keyword=biblioauthorother]%
        \printbibliography[heading=nobibheading, section=1, env=countregistered,         keyword=biblioregistered]%
        \printbibliography[heading=nobibheading, section=1, env=countauthorpatent,       keyword=biblioauthorpatent]%
        \printbibliography[heading=nobibheading, section=1, env=countauthorprogram,      keyword=biblioauthorprogram]%
        \printbibliography[heading=nobibheading, section=1, env=countauthor,             keyword=biblioauthor]%
        \printbibliography[heading=nobibheading, section=1, env=countauthorvakscopuswos, filter=vakscopuswos]%
        \printbibliography[heading=nobibheading, section=1, env=countauthorscopuswos,    filter=scopuswos]%
        %
        \nocite{*}%
        %
        {\publications} Основные результаты по теме диссертации изложены в~\arabic{citeauthor}~печатных изданиях,
        \arabic{citeauthorvak} из которых изданы в журналах, рекомендованных ВАК\sloppy%
        \ifnum \value{citeauthorscopuswos}>0%
            , \arabic{citeauthorscopuswos} "--- в~периодических научных журналах, индексируемых Web of~Science и Scopus\sloppy%
        \fi%
        \ifnum \value{citeauthorconf}>0%
            , \arabic{citeauthorconf} "--- в~тезисах докладов.
        \else%
            .
        \fi%
        \ifnum \value{citeregistered}=1%
            \ifnum \value{citeauthorpatent}=1%
                Зарегистрирован \arabic{citeauthorpatent} патент.
            \fi%
            \ifnum \value{citeauthorprogram}=1%
                Зарегистрирована \arabic{citeauthorprogram} программа для ЭВМ.
            \fi%
        \fi%
        \ifnum \value{citeregistered}>1%
            Зарегистрированы\ %
            \ifnum \value{citeauthorpatent}>0%
            \formbytotal{citeauthorpatent}{патент}{}{а}{}\sloppy%
            \ifnum \value{citeauthorprogram}=0 . \else \ и~\fi%
            \fi%
            \ifnum \value{citeauthorprogram}>0%
            \formbytotal{citeauthorprogram}{программ}{а}{ы}{} для ЭВМ.
            \fi%
        \fi%
        % К публикациям, в которых излагаются основные научные результаты диссертации на соискание учёной
        % степени, в рецензируемых изданиях приравниваются патенты на изобретения, патенты (свидетельства) на
        % полезную модель, патенты на промышленный образец, патенты на селекционные достижения, свидетельства
        % на программу для электронных вычислительных машин, базу данных, топологию интегральных микросхем,
        % зарегистрированные в установленном порядке.(в ред. Постановления Правительства РФ от 21.04.2016 N 335)
    \end{refsection}%
    \begin{refsection}[bl-author, bl-registered]
        % Это refsection=2.
        % Процитированные здесь работы:
        %  * попадают в авторскую библиографию, при usefootcite==0 и стиле `\insertbiblioauthorimportant`.
        %  * ни на что не влияют в противном случае
        \nocite{vakbib2}%vak
        \nocite{patbib1}%patent
        \nocite{progbib1}%program
        \nocite{bib1}%other
        \nocite{confbib1}%conf
    \end{refsection}%
        %
        % Всё, что вне этих двух refsection, это refsection=0,
        %  * для диссертации - это нормальные ссылки, попадающие в обычную библиографию
        %  * для автореферата:
        %     * при usefootcite==0, ссылка корректно сработает только для источника из `external.bib`. Для своих работ --- напечатает "[0]" (и даже Warning не вылезет).
        %     * при usefootcite==1, ссылка сработает нормально. В авторской библиографии будут только процитированные в refsection=0 работы.
}


 % Характеристика работы по структуре во введении и в автореферате не отличается (ГОСТ Р 7.0.11, пункты 5.3.1 и 9.2.1), потому её загружаем из одного и того же внешнего файла, предварительно задав форму выделения некоторым параметрам

\textbf{Объем и структура работы.} Диссертация состоит из~введения,
\formbytotal{totalchapter}{глав}{ы}{}{},
заключения и
\formbytotal{totalappendix}{приложен}{ия}{ий}{}.
%% на случай ошибок оставляю исходный кусок на месте, закомментированным
%Полный объём диссертации составляет  \ref*{TotPages}~страницу
%с~\totalfigures{}~рисунками и~\totaltables{}~таблицами. Список литературы
%содержит \total{citenum}~наименований.
%
Полный объём диссертации составляет
\formbytotal{TotPages}{страниц}{у}{ы}{}, включая
\formbytotal{totalcount@figure}{рисун}{ок}{ка}{ков} и
\formbytotal{totalcount@table}{таблиц}{у}{ы}{}.
Список литературы содержит
\formbytotal{citenum}{наименован}{ие}{ия}{ий}.

В \textbf{первой} главе: особое внимание уделено процессам внутрипучкового рассеяния и наличию критической энергии, влияющие на динамику многозарядных тяжёлых ионов и лёгких ядер. С этой целью рассматривается дуальная магнитооптическая структура, способная адаптироваться для целей обоих типов экспериментов.
\par В случае тяжелых ионов зарядность выделяет проблему внутрипучкового рассеяния пучка на первый план. Разогрев пучка приводит к росту поперечного эмиттанса и продольного разброса по импульсам. Для предотвращения неконтролируемого роста фазового объёма применяются электронное и стохастическое охлаждение пучка.
\par Для легких частиц, таких как протоны, соотношение заряда к массе отличается почти в 2 раза по сравнению с тяжелыми ионами, таким образом пропорционально увеличивается и энергия эксперимента. При этом критическая энергия остается неизменной, поскольку является характеристикой конкретной установки и определяется магнитооптикой. Преодоление критической энергии является необходимым для обеспечения стабильности, в первую очередь, продольного движения. Таким образом, для тяжелых ионов такой проблемы не возникает, а в случае легких частиц, требуется принимать меры по преодолению критической энергии. Одним из таких методов может является создание резонансной структуры. 

Во \textbf{второй} главе проведён учёт влияния высших порядков разброса по импульсам и моделей продольных импедансов при пересечении критической энергии. Также рассмотрен метод скачка критической энергии для различных ускоряющих потенциалов с целью сохранения стабильности сгустка.

\par Существенное ограничение на параметры сгустка возникают из-за продольной микроволновой неустойчивости вблизи критической энергии. В конечном счёте это ограничивает величину светимости коллайдерного эксперимента. Для преодоления критической энергии классически используется процедура скачка критической энергии. Данная процедура осуществляется модулированием дисперсионной функции при приближении энергии пучка к критическому значению. Данные численного моделирования апробированы на экспериментальной установке У-70 в г. Протвино. Также рассмотрены эффекты влияния высших порядков коэффициента расширения орбиты и простейших моделей импедансов на динамику пучка.

В \textbf{третьей} главе рассматривается метод вариации критической энергии в резонансной структуре. Для этого может вводится как суперпериодическая модуляция градиентов квадрупольных линз для вариации дисперсионной функции, так и модуляция кривизны орбиты. Вследствие этого изменяется коэффициент уплотнения орбиты, который напрямую связан с критической энергий ускорителя.

\par Для регулярной магнитооптической структуры коллайдера NICA рассмотрены варианты модернизации для создания резонансной структуры с поднятой критической энергией. Поскольку установка рассматривалась как стационарная, то это возможно только путем модуляции градиентов в квадрупольных линзах. Для различных полученных структур представлены схемы расстановки секступолей.

В \textbf{четвёртой} главе рассматривается возможность изучения электрического дипольного момента легких заряженных частиц. Исследовано применение концепции квази-замороженного спина для накопительных колец. Рассматривается возможность модернизации колец с сохранением текущего предназначения  и расширением исследовательских возможностей установок. Изучена спиновая динамика в кольце с использованием электростатических, а также элементов с совмещенной функцией

\par Для проведения эксперимента по поиску ЭДМ становится необходимым использовать альтернативный метод управления спином, концепция квази-замороженного спина. В отличие от метода замороженного спина, спин-вектор больше не сохраняет ориентацию в течение всего периода обращения, а восстанавливает на прямолинейном участке. Это возможно благодаря использованию элементов как с электрическим, так и с магнитным полями, которые могут быть представлены либо электростатическим дефлектором с магнитным киккером, либо фильтром Вина, на прямом участке. Поворот спина в арке на определенный угол компенсируется соответствующим поворотом в компенсирующем элементе. Поля подбираются таким образом, чтобы создать нулевую силу Лоренца и не нарушить прямолинейность орбиты. Поляриметры, расположенные после компенсации, будут обнаруживать ту же ориентацию спин-вектора, и для них она будет 'заморожена'.

В \textbf{заключении} приведены результаты работы.