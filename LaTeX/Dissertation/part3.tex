%part 3

	\chapter{Прохождение критической энергии в регулярной магнитооптической структуре синхротрона NICA}\label{ch:transition_jump}

\par Данная глава посвящена рассмотрению процедуры скачка критической энергии в регулярной структуре спроектированного синхротрона.

\par Также процедура скачка была исследована для синхротрона У-70 при помощи численного моделирования и 

\par В статье исследуется влияние импедансов пространственного заряда, а также ВЧ на продольную динамику во время процедуры преодоления критической энергии скачком. Отличительной особенностью является использование ВЧ барьерного типа, в результате чего достигается специфическое распределение пучка в фазовом пространстве, отличное от классического, формируемого гармоническим ВЧ.

\par При рассмотрении продольного движения вводится понятие коэф\-фи\-ци\-ента
расширения орбиты (momentum compaction factor) \cite{lee}:

\begin{equation}
\alpha_c=\frac{1}{R_0} \frac{d R}{d \delta}=\alpha_0+2 \alpha_1 \delta+3 \alpha_2 \delta^2+\cdots \equiv \frac{1}{\gamma_T^2}
\label{alpha}
\end{equation}

и коэффициента скольжения (slip-factor):

\begin{equation}
\eta(\delta)=-\frac{1}{\omega_0} \frac{\Delta \omega}{\delta}=-\left(\eta_0+\eta_1 \delta+\eta_2 \delta^2+\cdots\right),
\label{eta}
\end{equation}

где $\delta$ – разброс по импульсам, $R_{0},R$ – усреднённый радиус референсной и отклоненной на $\delta$ частиц, $\omega, \omega_{0}$ – соответствующие частоты, $\alpha_n, \eta_n$ – n-ые члены разложения, $\gamma_{tr}$ – критическая энергия. Коэффициенты могут быть связаны соотношениями $\eta=\eta_{0}=\alpha_{0}-\frac{1}{\gamma_{0}^{2}}$, $\eta_{1}=\alpha_{1}-\frac{\eta_{0}}{\gamma_{0}^{2}}+\frac{3}{2} \frac{\beta^{2}}{\gamma^{2}}$ . Как видно при определённой энергии референсной частицы -- критической $\gamma = \gamma_{tr}$, коэффициент скольжения принимает нулевое значение $\eta = \eta_{0} = 0$.

	\section{Построение регулярной структуры на основе ячеек ФОДО, ФДО, ОДФДО}\label{sec:transition_jump/FODO_FDO}

	\section{Типы подавителей дисперсии в арках с регулярной структурой с краевыми пропущенными магнитами и/или квадруполями с варьируемыми градиентами}\label{sec:transition_jump/suppression}

\section{Прохождение критической энергии в У-70}\label{sec:transition_jump/U-70}
\par Прохождение критической энергии является актуальной задачей для протонного пучка в строящемся комплексе NICA (ОИЯИ г. Дубна). С целью изучения данной проблемы, исследована динамика продольного движения в окрестности критической энергии У-70 (НИЦ “Курчатовский институт” - ИФВЭ г. Протвино).

\par Увеличение скорости прохождения критической энергии уменьшает влияние факторов, возмущающих фазовое движение. Метод скачка критической энергии применяется на многих установках CERN [1], BNL [2], в том числе реализован на У-70. Сдвиг критической энергии обеспечивается искажением дисперсионной функции за счёт использования тонких квадрупольных линз [3].
\par Результаты данного исследования помогут осветить потенциальные последствия прохождения критической энергии и определить существенные параметры, влияющие на динамику фазового движения.

\par Вдали от критической энергии частота синхротронных колебаний слабо меняется со временем, движение адиабатично. Вблизи критической энергии нарушается условие адиабатичности синхротронного движения. Характерное время адиабатичности можно оценить, сравнивая синхротронную частоту с темпом изменения удерживающей сепаратрисы (рис. 1a) [5]:

\begin{equation}
\tau_{ad}=\left(\frac{\pi\beta^2mc^2\gamma_{tr}^4}{\dot{\gamma}\omega_0^2heV\left|\cos{\phi_s}\right|}\right)^{1/3}
\label{eq:adiabaticity}
\end{equation}

где $\gamma_{tr}$ – Лоренц-фактор, соответствующий критической энергии, $\dot{\gamma}$ – темп изменения энергии. Нелинейность продольного движения проявляется, когда $\eta_1\delta$ сравнимо с $\eta_0$, характерное время (рис. 1б):

\begin{equation}
\tau_{nl}=\frac{\eta_1\hat{\delta}}{\sfrac{2\dot{\gamma}}{{\gamma_{tr}}^3}}=\gamma_{tr}\frac{\sfrac{3}{2}\beta^2+\gamma_{tr}^2\alpha_1}{2\dot{\gamma}}\
\label{eq:nonliniarity}
\end{equation}

где $\hat{\delta}\approx{10}^{-2}-{10}^{-3}$ – абсолютное значение максимального отклонения импульса вблизи критической энергии, $\alpha_1$ – второй порядок коэффициента уплотнения орбиты. Для регулярной ФОДО структуры У-70 с скомпенсированной натуральной хроматичностью, получено $\alpha_1\simeq0.01$ [6].

\begin{figure}
   \includegraphics*[width=0.49\columnwidth]{3_adiabatic_time.png}
   \includegraphics*[width=0.49\columnwidth]{3_nonlin_time.png}
   \caption{а) Классическая синхротронная частота и тем изменения огибающей сепаратрисы в окресности критической энергии от номера оборота; б) изменение первого и второго порядка коэффициента проскальзывания $\eta_0$, $\eta_1\delta$ в окресности критической энергии от номера оборота.}
   \label{fig:3_adiabatic_time_nonlin_time}
\end{figure}

Кроме того, из ур.(1) следует условие стабильности синхротронных колебаний

\begin{equation}
\eta_0\cos\phi_s<0
\label{eq:long_stability}
\end{equation}

\noindent Видно, что для продольного согласования при прохождении критической энергии также должна быть сдвинута фаза $\phi_s$ ускоряющего поля ВЧ на $\pi-2\phi_s$.

\par Оценки для У-70, приведенные в табл. 1, отражают тот факт, что время адиабатичности (2) может быть сравнимо со временем нелинейности (3) $\tau_{ad}\sim\tau_{nl}$. При приближении энергии к критической, продольная длина пучка уменьшается, а разброс по импульсам увеличивается. На рис. 2 приведены результаты моделирования прохождения критической энергии при ускорении от $7.0$ до $13.0$ ГэВ для $\eta=\eta_0$ и $\eta=\eta_0+\eta_1\delta$ в различных моделях BLonD [7]. Влияние второго порядка коэффициента проскальзывания увеличивает продольный эмиттанс.

\begin{figure}
   \includegraphics*[width=0.32\columnwidth]{3_eta_beam_lenght.png}
   \includegraphics*[width=0.32\columnwidth]{3_eta_energy_spread.png}
   \includegraphics*[width=0.32\columnwidth]{3_eta_emit.png}
   \caption{Зависимость a) длины cгустка, б) разброса энергии внутри сгустка, в) продольного эмиттанса от номера оборота в окрестности критической энергии при изменении энергии от $7$ до $13$ ГэВ для трёх моделей без скачка и учёта импеданса. 
Синяя – учёт только первого порядка $\eta=\eta_0$, ‘simple’ solver, оранжевая – $\eta=\eta_0$, ‘exact’ solver, зеленая – $\eta=\eta_0+\eta_1\delta$, ‘exact’ solver.}
   \label{fig:3_eta}
\end{figure}

\textbf{ВЛИЯНИЕ ИНДУКТИВНОГО ИМПЕДАНСА}

\par На продольную динамику также оказывает влияние элементов ускорителя. Импеданс описывает взаимодействие пучка с элементами структуры ускорителя. Особенно важным для изучения динамики при прохождении критической энергии является продольный импеданс $Z_\parallel\left(\omega\right)$. Аналитический расчёт полного импеданса кольца является трудной задачей и в данной работе ограничен его индуктивной компонентой $\sfrac{Z_n}{n}=\pm i \cdot const$. Отрицательная индуктивность соответствует импедансу гладкой камеры, а положительная – продольному импедансу связи пикап-электродов, кикер-магнитов и сильфонов [3].
	На сеансе для У-70 наблюдалась интенсивность в импульсе порядка $N_{tot}=4\bullet{10}^{12}$ ppp (particles per period), соответственно, в сгустке – порядка $N_{beam}=4\bullet{10}^{11}$ (particles per beam). Моделирование продольной динамики при изменении энергии от 7 до 9 ГэВ показывает, что при малой интенсивности $N_{beam}=4\bullet{10}^{11}$ как для отрицательного, так и для положительного значений рассматриваемого импеданса пучок сохраняет стабильность. Для больших интенсивностей $N_{beam}=1\bullet{10}^{12}$ наблюдается существенное изменение симметрии фазового объёма и увеличение продольного эмиттанса (рис. 3, табл. 2). В соответствии с экспериментальными данными начальное значение длины сгустка $\tau_L=4t_\sigma\simeq20$ нс для $E_0=7$ ГэВ. Для гауссова распределения $\delta E_{0} = 4E \sigma = 52.7$ МэВ, $\varepsilon_{0 95\%}=1.23$ эВ$\cdot$с.
	
\begin{figure}
   \includegraphics*[width=0.32\columnwidth]{3_wo_jump_beam_lenght.png}
   \includegraphics*[width=0.32\columnwidth]{3_wo_jump_energy_spread.png}
   \includegraphics*[width=0.32\columnwidth]{3_wo_jump_emit.png}
   \caption{Зависимость a) длины cгустка, б) разброса энергии внутри сгустка, в) продольного эмиттанса от номера оборота в окресности критической энергии при изменении энергии от $7$ до $9$ ГэВ без скачка, с учётом различного вида импеданса и интенсивностей.}
   \label{fig:3_wo_jump}
\end{figure}

\textbf{СКАЧОК КРИТИЧЕСКОЙ ЭНЕРГИИ У-70}

\par Для сохранения стабильности продольного движения, продольный эмиттанс не должен расти при прохождении критической энергии. Для этой цели на У-70 используется метод скачка критической энергии [8]. Возрастает скорость прохождения критической энергии, при этом сам темп ускорения не меняется. Это достигается изменением параметров ускорителя, при котором изменяется $\alpha_0$. В общем случае, коэффициент расширения орбиты определяется как интеграл:

\begin{equation}
\alpha=\frac{1}{C} \int_0^{\mathrm{C}} \frac{D(s)}{\rho(s)} d s \cdot(9)
\label{eq:alpha_general}
\end{equation}

\noindent где $D\left(s\right)$ – дисперсионная функция, $\rho\left(s\right)$ – кривизна орбиты. Изменение коэффициента расширения орбиты возможно при модулировании дисперсионной функции, так как $rho\left(s\right)$ остается неизменной. Такая модуляция в синхротроне У-70 осуществляется квадруполями во $2$-ом и $8$-ом блоках каждого суперпериода [9]. На рис. 4 изображены параметры Твисса для одного суперпериода, состоящего из 10 магнитных блоков с совмещённой функцией как для регулярной структуры У-70, так и структуры с искаженной дисперсионной функцией [10].

\begin{figure}
   \includegraphics*[width=0.49\columnwidth]{3_twiss_U70_regular.png}
   \includegraphics*[width=0.49\columnwidth]{3_twiss_U70_modulated.png}
   \caption{Твисс-параметры $\beta_x,\beta_y, D_x$ для суперпериода У-70 a) регуляная структура; б) структура с модулированной дисперсией.}
   \label{fig:3_twiss_U70}
\end{figure}

\par Квадруполи расположены через полпериода $\Delta\nu_{x,y}=0.5\times0.5$ и имеют противоположные полярности. При такой модуляции дисперсии не происходит сдвига рабочей точки. В табл. 3 приведены значения рабочей точки в ходе процедуры поднятия критической энергии и скачка. Таким образом, поднятие критической энергии на переднем фронте происходит на $\delta \gamma_tr=0.9$ за $36$ мс, а сам скачок — за $1$ мс на заднем фронте. Принципиальная схема процедуры, а также соответствующее изменение первого порядка коэффициента скольжения приведены на рис. 5. Процедура скачка на сеансе У-70 приведена на рис. 6а, продольная линейная плотность сгустка относительно фазы ВЧ в момент скачка отражена на рис. 6б.

\begin{figure}
   \includegraphics*[width=0.49\columnwidth]{3_gamma_transition_jump_U70.png}
   \includegraphics*[width=0.49\columnwidth]{3_eta0_jump_U70.png}
   \caption{a) Поднятие критической энергии при процедуре скачка; б) соответствующее изменение первого порядка коэффициента скольжения $\eta_0$.}
   \label{fig:3_gamma_transition_jump_U70}
\end{figure}

\par Данные моделирования продольного движения соответствуют изменению длины сгустка в ходе ускорительного цикла на сеансе У-70 (рис. 7). Результаты моделирования продольного движения (Рис. 8 и Таблица 4) показаны для разных моделей при ускорении от $6.9-12.9$ ГэВ [9] для скачка критической энергии. А также для скачка с учётом импедансов вида $\sfrac{Z_n}{n}=\pm i\bullet const$ и разных интенсивностей при ускорении $6.9-8.9$ ГэВ (Рис. 9). Начальные значения $\tau_L=4t_\sigma\simeq20$ нс при $E_0=6.9$ ГэВ, $\delta E_{0} = 4E \sigma =49.3$ МэВ,  $\varepsilon_{0 95\%}=1.16$ эВ$\cdot$с. 

\begin{figure}
   \includegraphics*[width=0.49\columnwidth]{3_jump_U70_oscilogram.png}
   \includegraphics*[width=0.49\columnwidth]{3_jump_U70_beam_profile.png}
   \caption{а) Скачок критической энергии на сеансе У-70. Зелная линия – сигнал с фазового датчика, фиолетовая – градиент в обмотках дополниительных квадруполей, голубая – сигнал с пикового детектора; б) Продольная линейная плотность сгустка относительно фазы ВЧ в момент скачка.}
   \label{fig:3_jump_U70_oscilogram}
\end{figure}

\begin{figure}
   \includegraphics*[width=0.49\columnwidth]{3_jump_U70_beam_lenght.png}
   \caption{Изменение длины сгустка в ходе ускорительного цикла на сеансе У-70.}
   \label{fig:3_jump_U70_beam_lenght}
\end{figure}

\begin{figure}
   \includegraphics*[width=0.32\columnwidth]{3_jump_beam_lenght.png}
   \includegraphics*[width=0.32\columnwidth]{3_jump_energy_spread.png}
   \includegraphics*[width=0.32\columnwidth]{3_jump_emit.png}
   \caption{Зависимость a) длины cгустка, б) разброса энергии внутри сгустка, в) продольного эмиттанса от номера оборота в окрестности критической энергии при изменении энергии от $6.9$ до $12.9$ ГэВ для трёх моделей со скачком, без учёта импеданса. Синяя – учёт только первого порядка $\eta=\eta_0$, ‘simple’ solver, оранжевая – $\eta=\eta_0$, ‘exact’ solver, зеленая – $\eta=\eta_0+\eta_1\delta$, ‘exact’ solver.}
   \label{fig:3_jump}
\end{figure}

\begin{figure}
   \includegraphics*[width=0.32\columnwidth]{3_jump_imp_beam_lenght.png}
   \includegraphics*[width=0.32\columnwidth]{3_jump_imp_energy_spread.png}
   \includegraphics*[width=0.32\columnwidth]{3_jump_imp_emit.png}
   \caption{Зависимость a) длины cгустка, б) разброса энергии внутри сгустка, в) продольного эмиттанса от номера оборота в окресности критической энергии при изменении энергии от $6.9$ до $8.9$ ГэВ со скачком, с учётом различного вида импеданса и интенсивностей.}
   \label{fig:3_jump_imp}
\end{figure}

\par При сравнении двух рассмотренных способов прохождения критической энергии: без скачка критической энергии и со скачком, можно установить, что при скачке продольная длина сгустка сокращается меньше. Таким образом и рассмотренные импедансы меньше возмущают сгусток. Рост эмиттанса наблюдается только при рассмотрении интенсивного сгустка, где число частиц $N_{beam}=1\times{10}^{12}$ ppb (particles per beam)

\textbf{ЗАКЛЮЧЕНИЕ}
Прохождение критической энергии в гармоническом ВЧ как с использованием метода скачка, так и без него, было рассмотрено в сеансе на протонном синхротроне У-70. Проведено численное моделирование продольной динамики для различных импедансов и интенсивностей сгустков.

\par Показано, что темп ускорения играет ключевую роль при прохождении критической энергии. Для его увеличения используют метод скачка критической энергии. Изменение критической энергии осуществляется при помощи модуляции дисперсионной функции, что позволяет контролировать продольный эмиттанс сгустка в момент прохождения критической энергии.
\par Изученная динамика продольного движения вблизи критической энергии представляет интерес для дальнейшего изучения на комплексе NICA.

\section{Продольная микроволновая неустойчиввость}\label{sec:transition_jump/microwave_instab}

\par Проблема прохождения критической энергии в синхротроне NICA (ОИЯИ г. Дубна) актуальна для экспериментов с протонами при энергии пучка 13 ГэВ, поскольку может приводить к росту эмиттанса и в конечном счёте накладывает ограничения на конечную светимость. Для экспериментов с тяжелыми ионами при энергии $4.5$ ГэВ такой сложности не возникает, так как критическая энергия, характеристика кольца, $5.7$ ГэВ. 
\par Реализация скачкообразного прохождения критической энергии в NICA с сдвигом бетатронной частоты ограничивает величину скачка. Ограниченный темп изменения градиентов квадруполей влечет ограниченный темп изменения критической энергии. Подобная схема скачка рассмотрена для отличных по своему принципу работы ускоряющих ВЧ станций, барьерную и гармоническую. Кроме того, будет проведено сравнение с методикой прохождения скачком на У-70 (ИФВЭ г. Протвино).

	\textbf{СВЕТИМОСТЬ}

\par Для коллайдерного эксперимента светимость является ключевой величиной. В простейшем случае, столкновение симметричных сгустков, светимость дается формулой [1]:

\begin{equation}
L=\frac{n_{\mathrm{bunch}}N_1N_2f_0}{4\pi\sqrt{\varepsilon_x\varepsilon_y}B^\ast}\Phi_{HG}, \
\Phi_{HG}(\alpha)=\frac{2}{\sqrt\pi}\int_{0}^{\infty}\frac{e^{-u^2}du}{1+(\alpha u)^2},\
\alpha=\frac{\sigma_s}{B^\ast},
\label{eq:luminocity}
\end{equation}

\noindent где $n_{\mathrm{bunch\ }}$– количество сгустков, $N_1$, $N_2$ – количество частиц в сталкивающихся сгустках, $\varepsilon_x$, $\varepsilon_y$ – продольные эмиттансы, $f_0$ – частота обращения, $\Phi_{HG}$ – параметр песочных часов, $\sigma_s$ – гауссов параметр продольного размера, $B^\ast$ – бета-функция в точке столкновения. Как видно, данная формула отражает принципиальную зависимость от множества параметров как пучка, так и магнитооптики.

\begin{figure}
   \includegraphics*[width=0.49\columnwidth]{3_tunes_vs_g_tr.png}
   \caption{Зависимость бетатронной частоты в x, y – плоскости от $\gamma_{tr}$ при модуляции дисперсионной функции изменением градиента в фокусирующих линзах.}
   \label{fig:3_tunes_vs_g_tr.png}
\end{figure}

\par Прохождение через критическую энергию оказывает существенное влияние на продольную динамику. Светимость явно зависит от продольной длины пучка только в параметре песочных часов. $\Phi_{HG}(1)\cong0.76$, $\Phi_{HG}(2)\cong0.55$, $\Phi_{HG}(5)\cong0.29$, то есть при неизменных параметрах и увеличении только длины сгустка в $2$ раза, влияние эффекта песочных часов уменьшит исходную светимость на $30\%$ $L_2=0.7L_1$. Для NICA предполагается достичь $\alpha=1$, $\sigma_s=0.6$ м, бета-функция в точке встречи $B^\ast=0.6$ м. Таким образом учтена только явная зависимость от продольной длины. Неявно, светимость зависит от продольного эмиттанса сгустка так как накладывает ограничение на количество частиц.

	\textbf{УСКОРЕНИЕ}

\par Рассмотрим эволюцию продольного эмиттанса в процессе ускорения в барьерном ВЧ. 
Для достижения светимости порядка $2 \cdot 10^{32}$см-2с-1, конечный среднеквадратичный нормализованный продольный эмиттанс сгустка равен $\varepsilon_{\sin}^{exp}=n_{bunch}\gamma_{exp}\beta_{exp}\pi\sigma_s\sigma_p=0.9$ м ($\gamma_{exp}=14.3$, $n_{bunch}=22$, $\sigma_s=0.6$ м, $\sigma_p=1.5\cdot 10^{-3}$) при энергии порядка $13$ ГэВ. 
Формируется из эмиттанса равномерного сгустка в барьерном ВЧ $\varepsilon_{bb}^{fin}$, разделенного на $22$ сгустка $\varepsilon_{sin}^{exp}={D_{gym}\varepsilon}_{bb}^{fin}$ при помощи ВЧ гимнастики. 
Эмиттанс барьерного ВЧ подвержен влиянию критической энергии на эмиттанс охлажденного пучка после инжекции $\varepsilon_{bb}^{cool}$, $\varepsilon_{bb}^{fin}=D_{tr}\varepsilon_{bb}^{cool}$. 
Охлажденный пучок формируется после инжекции, накопления и электронного охлаждения на $2-3$ ГэВ$\varepsilon_{bb}^{cool}=D_{cool}\varepsilon_{bb}^{inj}$. 
Только охлаждение уменьшает эмиттанс $D_{cool}<1$, остальные эффекты, только раздувают эмиттанс $D_{gym}>1$, $D_{tr}>1$. 
Для гимнастики было принято $D_{gym}=1.3$, влияние $D_{tr}$ будет обсуждено далее.


	\textbf{СКАЧОК КРИТИЧЕСКОЙ ЭНЕРГИИ}
\par Метод скачка критической энергии применяется для сохранения фазового объема при переходе через критическую энергию. В NICA рассматривается скачок критической энергии за счёт сдвига бетатронных частот. Параметры скачка могут быть определены при рассмотрении магнитооптической структуры и возможностью изменения тока в квадрупольных линзах в поворотных арках. 

\par Изменение критической энергии достигается путем изменения коэффициента расширения орбиты

\begin{equation}
\alpha=\frac{1}{C} \int_0^{\mathrm{C}} \frac{D(s)}{\rho(s)} d s \cdot(9)
\label{eq:alpha_general}
\end{equation}

\noindent где $D\left(s\right)$ – дисперсионная функция, $\rho\left(s\right)$ – кривизна орбиты. При этом необходима модуляция дисперсионной функции. Магнитооптическая структура поворотных арок NICA состоит из $12$ ФОДО ячеек с подавленной на краях дисперсией (Рис. 1а). С помощью программ для численного моделирования движения пучка в магнитных системах ускорителей MADX [2] и OptiM [3] изучена зависимость изменения критической энергии от частоты бетатронных колебаний, при этом изменялся градиент в фокусирующих квадрупольных линзах. Именно в этих элементах расположен максимум $\beta_x$ и $D_x$. В имеющейся структуре $\Delta\gamma_{tr}=1.1\Delta q$ (Рис. 1б). Для обеспечения скачка порядка $\Delta\gamma_{tr}=0.09$ потребуется изменять частоту в пределах $\pm\Delta q=0.05$. Соответствующее суммарное изменение градиента $\Delta Kl=4\pi\Delta q\beta a=0.055$ м-1, где $\beta_a=11.5$ м – средняя бета-функция. Тогда максимальное изменение градиента в одном квадруполе $\Delta G= \Delta Kl(BR/N_Fl)=0.5$ Тл/м, где $N_F=24$ – количество фокусирующих линз, $BR=22$ Тл$\cdot$м – магнитная жесткость при кинетической энергии протонов $5.7$ ГэВ (критическая энергия), $l=0.47$ м – длина квадруполя. При этом ограничение скорости нарастания тока приводит к ограничению в изменении градиента квадрупольных линз. Темп изменения критической энергии $\sfrac{d\gamma_{tr}}{dt}=8.5$ c$^{-1}$.[4]

\par В протонном синхротроне НИЦ «Курчатовский институт» - ИФВЭ У-70 также используется методика скачка критической энергии [5]. Магнитооптическая структура У-70 является ФОДО-периодичной и состоит из $12$ суперпериодов с $10$ магнитными блоками  с совмещенной функцией. [6] Ускорение осуществляется гармоническим ВЧ с темпом $\left(\sfrac{d\gamma}{dt}\right)_{U-70}=40$ c$^{-1}$. Скачок достигается также искажением дисперсионной функции, однако без смещения рабочей точки. Дополнительные квадруполи, расположенные через полпериода $\Delta\nu_{x,y}=0.5\times0.5$ которые имеют противоположные полярности, модулируют дисперсионную функцию. Изменение критической энергии происходит на $\Delta\gamma_{tr}^{U-70}=0.9$ (Рис. 2а) за $1$ мс, то есть в $10$ раз больше и в $100$ раз быстрее $\left(\sfrac{d\gamma_{tr}}{dt}\right)_{U-70}=900$ c$^{-1}$, по сравнению с упомянутым скачком для NICA. Также показано соответствующее изменение $\eta_0=\alpha_0-\frac{1}{\gamma^2}$ (Рис. 2б), где $\gamma$ – Лоренц-фактор пучка. [7]

\par Темп ускорения непосредственно влияет на динамику продольного движения. В NICA имеется $3$ различные ВЧ станции: ВЧ-1 – барьерное, четыре ВЧ-2, восемь ВЧ-3 – гармонические с гармоническим числом $22$ и $66$ соответственно. Максимальное суммарное напряжение составляет порядка $\left(\sfrac{d\gamma}{dt}\right)_{RF2}=30$ c$^{-1}$, $\left(\sfrac{d\gamma}{dt}\right)_{RF3}=300$ c$^{-1}$ и значительно больше, чем для индукционного ускорения в барьерном $\left(\sfrac{d\gamma}{dt}\right)_{RF1}=0.2$ c$^{-1}$ [8].

	\textbf{ГАРМОНИЧЕСКОЕ ВЧ}
	
\begin{figure}[h!]
   \includegraphics*[width=0.49\columnwidth]{3_g_tr_harmonic.png}
   \includegraphics*[width=0.49\columnwidth]{3_eta_tr_harmonic.png}
   \caption{a) Принципиальная схема поднятия критической энергии на NICA в гармоническом ВЧ с темпом $\left(\sfrac{d\gamma}{dt}\right)_{RF2}=30$ c$^{-1}$ при процедуре скачка на $\mathrm{\Delta}\gamma_{tr}=0.09$ с темпом $\sfrac{d\gamma_{tr}}{dt}=8.5$ c$^{-1}$; б) соответствующее изменение первого порядка коэффициента скольжения $\eta_0=\pm1\times{10}^{-3}$.}
   \label{fig:3_g_tr_harmonic.png}
\end{figure}
	
\par Ускорение в гармоническом ВЧ-резонаторе достигается путем смещения фазы пучка относительно фазы ВЧ. Темп ускорения в гармоническом ВЧ-2 $\left(\sfrac{d\gamma}{dt}\right)_{RF2}=30$ c$^{-1}$ больше максимального темпа изменения критической энергии $\sfrac{d\gamma_{tr}}{dt}=8.5$ c$^{-1}$. На Рис.3а показана схема симметричного скачка от $\gamma_{tr}+\Delta\gamma_{tr}/2$ до $\gamma_{tr}-\Delta\gamma_{tr}/2$. При этом предварительное увеличение критической энергии и соответствующее восстановление до стационарного значение может происходить не с максимальным темпом изменения критической энергии, а медленнее. Таким образом, время нахождения вблизи нулевого значения $\eta$ сокращается. По сравнению со случаем скачка для У-70, коэффициент проскальзывания за время скачка изменяется медленно (Рис. 3б). Долгое пребывание вблизи около нулевого значения $\eta$ и является опасным для продольной динамики пучка. Именно поэтому и применяется процедура скачка (быстрого пересечения) критической энергии. В данном случае из-за ограниченной величины самого скачка $\Delta\gamma_{tr}=0.09$, а также ограниченного темпа изменения критической энергии $\sfrac{d\gamma_{tr}}{dt}=8.5$ c$^{-1}$, сам скачок оказывается незначительным.

	\textbf{БАРЬЕРНОЕ ВЧ}
	
\par Барьерное ВЧ-1 генерирует барьерные импульсы $5$ кВ для удержания пучка, ускорение достигается индукционно, меандром с напряжением $300$ В [8]. Темп ускорения $\left(\sfrac{d\gamma}{dt}\right)_{RF1}=0.2$ c$^{-1}$ значительно ниже по сравнению с гармоническим (Рис. 4а). Скачок происходит за тоже время ($10$ мс), что и для случая гармонического ВЧ. Продольная динамика в таком ВЧ отличается от случая гармонического. При этом достигает малого значения $\eta_0=-2.5\times{10}^{-4}$ и удерживается вблизи малого значения, в этом случае нелинейность $\eta=\eta_0+\eta_1\delta$, сказывается на частицах с большим $\delta$. Однако, для барьерного ВЧ это не опасно, так как нет дополнительного возбуждения, которое может вытолкнуть частицы вне сепаратрисы и может только исказить распределение между барьерами. Профиль пучка имеет ненулевой градиент только по краям, где частицы отражаются от барьера. После поднятия критической энергии происходит скачок критической энергии за $10$ мс в отсутствие барьеров. За это время фазовый портрет изменяется незначительного. А затем, захватывается барьерами с обратной полярностью.
\par Главным остается то, что ограничены 1) возможная величина скачка $\Delta\gamma_{tr}=0.09$; 2) темп изменения критической энергии $\sfrac{d\gamma_{tr}}{dt}=8.5$ c$^{-1}$. Ограничение на величину скачка приводит к ограничению на скачок коэффициента проскальзывания $\eta_0=\pm2.5\times{10}^{-4}$. Барьерное ВЧ подразумевает относительно долгое удержание пучка в окрестности около нулевого значения $\eta_0$ (Рис. 4б).

\begin{figure}
   \includegraphics*[width=0.49\columnwidth]{3_g_tr_BB.png}
   \includegraphics*[width=0.49\columnwidth]{3_eta_tr_BB.png}
   \caption{a) Принципиальная схема поднятия критической энергии на У-70 при процедуре скачка на $\Delta\gamma_{tr}^{U-70}=0.9$ с темпом $\left(\sfrac{d\gamma_{tr}}{dt}\right)_{U-70}=900$ c$^{-1}$; б) соответствующее изменение первого порядка коэффициента скольжения $\eta_0$.}
   \label{fig:3_g_tr_BB.png}
\end{figure}

	\textbf{ПРОДОЛЬНАЯ МИКРОВОЛНОВАЯ НЕУСТОЙЧИВОСТЬ}
	
\par Ограничение на порог микроволновой неустойчивости зависит от многих параметров и для равномерного распределения, характерного именно барьерному ВЧ определяется критерием Кейл-Шнель. В модифицированном виде этот критерий приведен в [9].

\begin{equation}
K_1K_2\frac{E_0}{\left(\left|Z_\parallel\right|/n\right)I}\frac{A_i}{Z_i}\gamma\beta^2|\eta|\sigma_p^2\geq1
\label{eq:microwave_instability}
\end{equation}

\par Ток $I=\frac{{e\beta cN}_pZ_i}{L_B}$, тут $L_B$ – длина пучка или для барьерного ВЧ это эквивалентно расстоянию между удерживающими барьерами (приближено, без учётов краевых эффектов). Отсюда видно, что возникает ограничение на количество частиц $N_p$ ($A, Z=1$ для протонов).

\begin{equation}
N_p\le K_1K_2\frac{E_0}{(\sfrac{Z_\parallel}{n})ec}\left|\eta\right|\gamma\beta\sigma_p^2L_B
\label{eq:microwave_instability_1}
\end{equation}

\noindent или, если учесть, что нормализованный эмиттанс для барьерного ВЧ $\varepsilon_{tr}=\gamma_{tr}\beta_{tr}\sqrt\pi\sigma_pL_B$ ($\sqrt\pi$ так как распределение по импульсам имеет гауссов вид, а продольный размер – равномерный), то справедливо для барьерного ВЧ

\begin{equation}
N_p\le K_1K_2\frac{E_0}{(\sfrac{Z_\parallel}{n})ec}\left|\eta\right|\frac{\varepsilon_{tr}^2}{{\pi\gamma\beta L}_B}
\label{eq:microwave_instability_2}
\end{equation}

\par Таким образом при нахождении вблизи малого значения $\left|\eta\right|$ количество частиц, ограничено длиной сгустка в барьерном ВЧ. При этом нормализованный эмиттанс определяется из необходимости иметь достаточную светимость $\varepsilon_{tr}=\varepsilon_{bb}^{fin}=\sfrac{\varepsilon_{sin}^{exp}}{D_{gym}}=0.7$ м. А длина сгустка может быть варьирована движением барьеров. 

Требуемое количество частиц для достижения светимости порядка $2\times10^{32}$см$^{-2}$с$^{-1}$ – $N_{exp}=1\times10^{12}$ для конечного сгустка, таким образом требуемое количество частиц в барьерном ВЧ как минимум должно быть $2.2\times10^{13}$. Для упомянутого скачка, энергия $E_0=E_{tr}=5.7$ ГэВ, $\gamma_{tr}=7.08$, $\beta=0.99$ вблизи $\left|\eta_0\right|=2.5\times10^{-4}$ для расчётов принято $\sfrac{Z_\parallel}{n}=20$ Ом, $K_1=1, K_2=5.4$.

\begin{equation}
N_p\le1\times5.45.7\times10^{9} eV 20 \Omega 1.6×10-19 K 3\times10^{8} ms 2.5\times10^{-4}0.7 m 27.08 \pi L_{B}
\label{eq:microwave_instability_example}
\end{equation}

Эта зависимость представлена на Рис.5. Таким образом ограничение для длины пучка $L_B=C_{ring}/2$ ограничение на количество частиц $N_p\le1.2\times10^{12}$, для $L_B=C_{ring}/10$, $N_p\le6\times10^{12}$.

\begin{figure}
   \includegraphics*[width=0.77\columnwidth]{3_microwave_inst_vs_lenght.png}
   \caption{Зависимость количества частиц в барьерном ВЧ и разброса по импульсам от длины между удерживающими барьерами с точки зрения продольной микроволновой неустойчивости.}
   \label{fig:3_microwave_inst_vs_lenght.png}
\end{figure}

\par Исходя из этих оценок, достичь конечного числа частиц $N_{exp}=1\times10^{12}$ для каждого из $22$ сгустков, представляется трудной задачей, вследствие возникновения продольной микроволновой неустойчивости вблизи критической энергии для интенсивного равномерного сгустка в барьерном ВЧ. 

	\textbf{Заключение}
\par В первую очередь коллайдерные эксперименты предъявляют требования к светимости. Которые дают ограничения, помимо прочего, и на продольный фазовый размер конченого сгустка. При ускорении, необходимо как преодолеть критическую энергию, так и разделить пучок на $22$ сгустка при помощи ВЧ гимнастики, при этом не раздуть фазовый объём. 
\par Рассмотрена возможная схема скачка критической энергии для NICA. Характерные величинами являются, величина скачка $\Delta\gamma_{tr}=0.09$ и темп изменения критической энергии $\sfrac{d\gamma_{tr}}{dt}=8.5$ c$^{-1}$. Для гармонического ВЧ предложенный скачок оказывает малое влияние на продольную динамику в силу малости скачка и его низкого темпа по сравнению с темпом ускорения. Для барьерного ВЧ, ограничение на величину скачка дает пороговое значение количества частиц в равномерном сгустке, вследствие микроволновой неустойчивости. И не позволяет достигнуть количества частиц $1\times10^{12}$ в конечном сгустке для достижения максимальной светимости.

	\section{Прохождение критической энергии в ускорителе NICA с регулярными арками}\label{sec:transition_jump/		regular}
	\subsection{Прохождение через критическую энергию за счет быстрого изменения бетатронной частоты}			\label{subsec:transition_jump/regular/gamma_jump}

\par Процедура скачка критической энергии применяется для преодоления критической энергии. Таким образом, удается сохранить устойчивое движение пучка в фазовом пространстве. Данный метод применялся на многих установках и описан в работах \cite{jump}, \cite{pip}.

\par Необходимость скачка можно понять, рассмотрев зависимость от $\eta(\delta)\\=\eta_{0}+\eta_{1} \delta+\cdots$, уравнений продольного движения, которые описывают эволюцию частиц в фазовом пространстве \cite{hans}:

\begin{equation}
\begin{aligned}
& \frac{d \tau}{d t}=\eta(\delta) \cdot \frac{h \cdot \Delta E}{\beta^2 \cdot E_0} \\
& \frac{d(\Delta E)}{d t}=\frac{V(\tau)}{T_0}
\end{aligned}
\label{long}
\end{equation}

\par При ускорении, значение коэффициента скольжения $\eta$ приближается к нулю для всех частиц, однако из-за ненулевого разброса по импульсам $\delta$, слагаемое $\eta_1\delta$ начинает быть сравнимо с $\eta_0$ и играет важную роль на динамику вблизи критической энергии. Если не предпринимать никаких мер, то для частиц, преодолевших критическую энергию, знак ко\-эф\-фи\-ци\-ента скольжения меняется. Исходя из Уравнений \ref{long}, видно, что движение в фазовой плоскости становится не устойчивым ведёт к потере пучка. Процедура скачка позволяет, во-первых, в течение поднятия критической энергии, удерживать пучок на расстоянии, достаточном, чтобы все час\-ти\-цы имели один и тот же знак коэффициента скольжения. Во-вторых, о\-бес\-пе\-чить быстрый переход к новому состоянию, где ко\-эф\-фи\-ци\-ент сколь\-же\-ния меняет знак, но для всех частиц снова имеет одинаковый знак. Стабильность обеспечивается сменой полярности у\-дер\-жи\-ва\-ющ\-их ВЧ-барьеров.

 Для коэффициента расширения орбиты может быть получено вы\-ра\-же\-ние \cite{resonant}:
 
\begin{equation}
\alpha=\frac{1}{C}\int_{0}^{C}{\frac{D\left(s\right)}{\rho\left(s\right)}ds}
\label{alpha_c}
\end{equation}

где $D\left(s\right)$– дисперсионная функция, $1/\rho\left(s\right)$ – кривизна орбиты. Для стационарной машины, возможно вариация дисперсионной функции для изменения значения $\alpha$, а соответственно и $\eta$. Например, для NICA, рассматривается возможность создания дополнительного градиента в квадрупольных линзах. Расчёты показывают, что возможно изменение критической энергии $\gamma_{tr}$ со скоростью $d\gamma_{tr}/\ dt\ =\ 8.5\ c^{-1}$ \cite{syresin}.

Можно выделить пять основных состояний продольной динамики, основанных на изменении критической энергии 
$\gamma_{tr}$ (Рисунок \ref{fig:jump_NICA}):

\begin{figure}[ht]
    \begin{minipage}[b][][b]{0.49\linewidth}\centering
        \includegraphics[width=0.9\linewidth]{3_Jump_NICA} \\ а)
    \end{minipage}
    \hfill
    \begin{minipage}[b][][b]{0.49\linewidth}\centering
        \includegraphics[width=0.9\linewidth]{3_Jump_NICA_2} \\ б)
    \end{minipage}
    \caption{Схема скачка критической энергии. Синяя линия – фактическая критическая энергия ускорителя $\gamma_{tr}$, красная линия – энергия референсной частицы.}
    \label{fig:jump_NICA}
\end{figure}

\begin{enumerate} 
  \item Ускорение от энергии инжекции $E_{inj}$ со стационарным значением $\gamma_{\ tr}^{stat}$;
  \item  Плавное увеличение $\gamma_{tr}$ параллельно энергии частиц до пикового значения, коэффициент скольжения $\eta_0$ приобретает минимально возможное значение, приближаясь к нулевому значению;
  \item Переход через стационарное значение критической энергии, при этом $\eta_0$ пересекает нулевое значение для всех частиц;
  \item Плавное восстановление $\gamma_{tr}$ до стационарного значения, также па\-рал\-лель\-но энергии частиц;
  \item Ускорение до энергии эксперимента со стационарным значением критической энергии $\gamma_{\ tr}^{stat}$.
  \end{enumerate}
 
Состояния 2-3-4 определяют процедуру преодоления $\gamma_{tr}$ скачком. Из\-ме\-не\-ние магнитооптики приводит к зависимости $\gamma_{tr}$, соответствующего смещения рабочей точки $\nu_{x,y}$ (Рисунок \ref{fig:tr}), а также высших порядков коэффициента расширения орбиты $\alpha_1, \alpha_2$ (Рисунок \ref{fig:alpha}).

\begin{figure}[!h]
   \includegraphics*[width=.49 \columnwidth]{3_gamma_tr_vs_dQF}
   \includegraphics*[width=.49 \columnwidth]{3_tunes_vs_dQF}
   \caption{Зависимость критической энергии и рабочей точки от возмущения градиента квадрупольных линз.}
   \label{fig:tr}
\end{figure}

\begin{figure}[!h]
   \includegraphics*[width=.49\columnwidth]{3_2nd_MCF_vs_g_tr.png}
   \includegraphics*[width=.49\columnwidth]{3_3nd_MCF_vs_g_tr.png}
   \caption{Зависимость высших порядков разложения коэффициента расширения орбиты от критической энергии.}
   \label{fig:alpha}
\end{figure}

	\subsection{Оптимизация динамической апертуры при прохождении критической энергии}\label{subsec:transition_jump/regular/optimization}
	\subsection{Обеспечение стабильности пучка с точки зрения динамической апертуры при процедуре скачка критической энергии}\label{subsec:transition_jump/regular/otimization_jump}
	\subsection{Динамика синхротронного движения с учетом зависимости MCF от квадратичного члена разброса по импульсам при прохождении критической энергии быстрым скачком в ВЧ типа Barrier Bucket}\label{subsec:transition_jump/regular/BB}

\par Для прохождения критической энергии, возможно использование ВЧ барьерного типа (Barrier Bucker RF) \cite{bb}, \cite{malyshev}. (Рисунок \ref{fig:rf})

\begin{figure}[!h]
  \centering
   \includegraphics*[width=.75\columnwidth]{3_BB}
   \caption{Нормализированная форма сигнала от ВЧ барьера.}
   \label{fig:rf}
\end{figure}

\begin{equation}
g(\phi)=\left\{\begin{array}{c}
-\operatorname{sign}(\eta),\quad -\pi / h_r \leq \phi \leq 0 \\
\operatorname{sign}(\eta),\quad 0<\phi \leq \pi / h_r \\
0, \quad \text { other }
\end{array}\right.
\label{sign}
\end{equation}

где $\eta$ – коэффициент скольжения (slip-factor), $h_r=\frac{\pi}{\phi_{r}}$ – гармоническое число для отражающего барьера и $\phi_{r}$ – соответствующая фаза.  В У\-рав\-не\-нии~$\ref{sign}$ учтено, что при прохождении через критическую энергию, знак $\eta$ меняется и, соответственно, полярность ВЧ барьеров. Для ускорения может быть также приложено дополнительное напряжение в виде меандра с напряжением $V_{acc}=300$ eV.

Коэффициенты Фурье-разложения для приведенного прямоугольного сигнала даются выражением \cite{bbcern}:

\begin{equation}
b_n=\operatorname{sign}{\left(\eta\right)}\frac{2}{n\pi}\left[1-\cos{\left(\frac{n}{h_r}\pi\right)}\right],
\label{b}
\end{equation}

где $n$ – номер гармоники. Для создания плавной формы сигнала, используется сигма-модуляция, сохраняющая симметрию сигнала:

\begin{equation}
\sigma_{m, n}={\text{sinc}}^m{\frac{n\pi}{2\left(N+1\right)}},
\label{sigma}
\end{equation}

где $N$ – количество членов гармонического разложения. Таким образом, напряжение n-ой гармоники:

\begin{equation}
V_n=V^{peak}b_n\sigma_{m, n}.
\label{Volt_n}
\end{equation}

На Рисунках \ref{fig:wave} представлены полученные формы сигнала и со\-от\-вет\-ству\-ющ\-ие напряжения для гармоник.

В зависимости от относительного смещения от референсной, частицы попадают под влияния ВЧ барьера – в области отражения и испытывают толчок энергии:

\begin{equation}
E_i^\prime=\Delta E_i+\sum_{j=1}^{N} V_j\sin{\left(\omega_{j}\mathrm{\Delta}t_i+\phi_j\right)} 
\label{dE}
\end{equation}

\begin{figure}[!h]
   \includegraphics*[width=.49\columnwidth]{3_BB_V}
   \includegraphics*[width=.49\columnwidth]{3_BB_fourier_coef}
   \caption{Разложение сигнала от ВЧ барьерного типа в ряд Фурье по синусоидальным гармоникам. Слева – форма 
   ВЧ барьеров, справа – амплитуды гармоник в зависимости от частоты для разной ширины отражающего барьера.}
   \label{fig:wave}
\end{figure}

\subsection{Учет импеданса, влияние пространственного заряда}\label{subsec:transition_jump/regular/SC}

\par Для учета влияния электромагнитного взаимодействия пучка с ок\-ру\-же\-ни\-ем вводится понятие импеданса. На продольную динамику основное влияние оказывает импеданс пространственного заряда \cite{laclare} (Рисунок \ref{fig:signal}) 

\begin{equation}
\frac{Z_{SC}}{n}=-\frac{Z_0}{2\beta\gamma^2}\left[1+2\ln{\left(\frac{b}{a}\right)}\right]
\label{sc}
\end{equation}

\begin{figure}[!h]
   \includegraphics*[width=.51\columnwidth]{3_Imp_vs_freq}
   \includegraphics*[width=.48\columnwidth]{3_induced_voltage}
   \caption{Слева – импеданс пространственного заряда; справа – Напряжение, создаваемое пространственным 
   зарядом вдоль профиля пучка в продольной плоскости. }
   \label{fig:signal}
\end{figure}

Для наглядности, приведём напряжение, индуцированное про\-стран\-стве\-нным зарядом, $V_{\mathrm{s.c.\ }}(\phi)$. Уравнение определяется производной от функции распределения $f(\phi)$ в пространстве \cite{sc}:

\begin{equation}
V_{\text{S.C.}}\left(\phi\right)=\frac{Z^2h^2g_0Z_0ce}{2R_0\gamma^2}\cdot\frac{\partial\left(N_0f\left(\phi\right)\right)}{\partial\phi}.
\label{V_sc}
\end{equation}

	\subsection{Проведение численного моделирования продольной динамики частиц с учетом высших порядков slip- factor}\label{subsec:transition_jump/regular/slip_factor}

\par Для ВЧ барьерного типа, как будет видно далее из Рисунков \ref{fig:2}, \ref{fig:3}, распределение внутри сепаратрисы равномерное непосредственно вне от\-ра\-жа\-ющего барьера. Таким образом, производная слабо отличается от нуля. Значительное напряжение может быть создано только на кра\-ях сепаратрисы, где наблюдается изменение градиента в профиле пучка.

\par Наиболее опасными с точки зрения разрушения пучка, являются со\-сто\-я\-ния 2-3-4, при которых изменяются параметры ускорителя. С точки зрения динамики, состояния 2 и 4 являются симметричными.
\par Профиль пучка в продольной плоскости равномерный, а э\-нер\-ге\-ти\-чес\-кий разброс гауссов. Состояние 2 и 4 характерны тем, что коэффициент скольжения для равновесной частицы остается неизменными, а кри\-ти\-чес\-кая энергия меняется синхронно с энергией пучка в течение порядка $2\times{10}^5$ оборотов. Таким образом, удержание пучка при стационарном значении критической энергии эквивалентно ускоренному движении пуч\-ка в структуре с меняющимися параметрами. Как видно на Рисунках 8 профиль пучка смещается к левому барьеру, это связано с тем, что для частиц с положительными $\delta>0$ коэффициент скольжения $\eta_{+\delta}$ больше, чем для частиц с отрицательным $\delta<0$ $\eta_{-\delta}: \eta_{+\delta}>\eta_{-\delta}$. Это видно из Уравнения \ref{eta} и того факта, что $\eta_1<0$. 

\begin{figure}
   \includegraphics*[width=.49\columnwidth]{3_BB_PS_near_tr}
   \includegraphics*[width=.49\columnwidth]{3_BB_PS_near_tr_2}
   \caption{Фазовая плоскость при удержании пучка внутри ВЧ-барьера. Слева – начальное распределение, справа – распределение после $2\times{10}^5 оборотов$.}
   \label{fig:2}
\end{figure}

\par Состояние 3 – быстрое изменение параметров в течение $6\times{10}^3$ о\-бо\-ро\-тов ($10~\text{ ms}$). ВЧ-барьеры выключены на время скачка, чтобы не разрушить пучок. Влияние пространственного заряда наиболее важно в отсутствие барь\-е\-ров, так как отсутствует внешняя удерживающая сила. Трекинг сделан с учетом описанного выше импеданса пространственного заряда.

\begin{figure}[!h]
   \includegraphics*[width=.49\columnwidth]{3_BB_PS_jump}
   \includegraphics*[width=.49\columnwidth]{3_BB_PS_jump_2}
   \caption{Фазовая плоскость при скачке, ВЧ-барьеры отключены. Слева – начальное распределение, справа – распределение после $6\times{10}^3$ оборотов.}
   \label{fig:3}
\end{figure}

\par За время скачка существенного изменения профиля пучка не про\-и\-зош\-ло. Моделирование выполнено в среде BLonD \cite{blond1}, \cite{blond}.

	\subsection{Сохранение поляризации при прохождении критической энергии}\label{subsec:transition_jump/regular/polarization}

\section{Ускорение и накопление пучка с использованием ВЧ резонаторов и barrier backet cavity}\label{subsec:transition_jump/regular/BB}

	\section{Заключение}
\par Изучена динамика продольного движения вблизи критической э\-нер\-гии в ВЧ барьерного типа, с учётом импеданса пространственного заряда. Процедура скачкообразного изменения параметров ускорителя является доступным вариантом преодоления критической энергии в барьерном ВЧ.

\FloatBarrier
