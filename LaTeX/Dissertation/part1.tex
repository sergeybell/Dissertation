%part 1

	\chapter{Особенности двойственной магнитооптической структуры для ускорения тяжелых ионов и легких частиц частиц}\label{ch:dual}

\par Независимо от назначения синхротрона, при наличии двух режимов, в которых ускоряются многозарядные тяжёлые частицы и одна или две лёгкие заряженные частицы, возникает задача определения оптимальной конфигурации магнитооптической структуры, которая обеспечит устойчивое движение обоих типов частиц. Очевидно, что многозарядные частицы, в отличие от лёгких, обладают более выраженным эффектом разогрева из-за внутрипучкового рассеяния \cite{trubnikov:cool}, а увеличивает вероятность прохождения лёгких частиц через энергию перехода. Эти эффекты имеют существенное значение для коллайдеров, где светимость играет ключевую роль. При разработке структуры, удовлетворяющей требованиям, предъявляемым к частицам с различным зарядом, принципиально важно создать перестраиваемую структуру без внесения конструктивных изменений. Мы назвали такую структуру дуальной.

	\section{Дуальность магнитооптической структуры для тяжелых ионов и легких ядер}\label{sec:ch:ions_light/duality}
	
\par Коллайдер NICA будет использован как для проведения коллайдерных экспериментов с тяжелыми ионами, так и легкими поляризованными ядрами. Различное соотношения заряда к массе является существенным при проектировании магнитооптики. Для достижения высокой светимости должно быть гарантировано достаточное время жизни пучка. Также должна быть решена проблема прохождения критической энергии.	

\par Высокое время жизни светимости пучка в коллайдерном эксперименте достигается за счет уменьшения эффекта внутрипучкового рассеяния в сочетании с использованием методов стохастического и электронного охлаждения. Этот подход особенно важен при работе с ионными пучками высокой интенсивности. Временная эволюция эмиттанса и разброса импульса при наличии процессов охлаждения определяется набором уравнений

\begin{equation}
\begin{aligned}
& \frac{d \varepsilon}{d t}=\underbrace{-\frac{1}{\tau_{t r}} \cdot \varepsilon}_{\text {cooling }}+\underbrace{\left(\frac{d \varepsilon}{d t}\right)_{I B S}}_{\text {heating }} \\
& \frac{d \delta^2}{d t}=\underbrace{-\frac{1}{\tau_{\text {long }}} \cdot \delta^2}_{\text {cooling }}+\underbrace{\left(\frac{d \delta^2}{d t}\right)_{\text {IBS }}}_{\text {heating }} \\
&
\end{aligned}
\end{equation}

\noindent где $\varepsilon$ – поперечный эмиттанс, $\tau_{tr}$ – поперечное время охлаждения, $\delta=\frac{\Delta p}{p}$ – разброс по импульсам, $\tau_{\mathrm{long\ }}$– продольное время охлаждения.
Для независимых от времени, стационарных значений, производные по времени становятся равными нулю, тогда

\begin{equation}
\begin{aligned}
& \varepsilon_{s t}=\left.\tau_{t r} \cdot\left(\frac{d \varepsilon}{d t}\right)_{I B S}\right|_{\mathcal{E}=\varepsilon_{s t}} \\
& \delta_{s t}^2=\left.\tau_{\text {long }} \cdot\left(\frac{d \delta^2}{d t}\right)_{I B S}\right|_{\delta^2=\delta_{s t}^2}
\end{aligned}
\end{equation}

Критерием применимости того, или иного метода охлаждения может быть сравнение характерных времен стохастического и электронного охлаждения со временем жизни с учетом ВПР во всем предполагаемом диапазоне энергий.

	\section{Оптимизация времени жизни пучка}
	
	\subsection{Стохастическое охлаждение}

\par Рассмотрим стохастическое охлаждение, пользуясь приближенной теорией D.Mohl \cite{mohl:stochastic, mohl:stochastic2}, . Следуя его основным выводам, скорость охлаждения определяется выражением		
	
\begin{equation}
\frac{1}{\tau_{t r, l}}=\frac{W}{N}[\underbrace{2 g \cos \theta\left(1-1 / M_{p k}^2\right)}_{\begin{array}{c}
\text { coherent } \\
\text { effect(cooling) }
\end{array}}-\underbrace{g^2\left(M_{k p}+U\right)}_{\begin{array}{c}
\text { incoherent } \\
\text { effect(heating) }
\end{array}}]
\end{equation}	

\noindent где $W=f_{max}-f_{min}$ – пропускная способность системы, $N$ – эффективное число частиц, пересчитанное через соотношение орбиты к длине сгустка с учетом его распределения, $g$ – fraction of observed sample error corrected per turn, $U=E({x_n}^2)/E({x_s}^2)$ – отношение шума к сигналу, $M_{pk}$, $M_{kp}$– факторы смешивания между пикапом – киккером и киккером – пикапом соответственно.

\noindent Уравнение (3) в отсутствии шума при $g=g_0={\frac{1-{M_{pk}}^2}{M_{kp}}}$ достигает максимум

\begin{equation}
\begin{aligned}
& \frac{1}{\tau_{t r}}=\frac{W}{N} \frac{\left(1-1 / M_{p k}^2\right)^2}{M_{k p}} \\
& \frac{1}{\tau_l}=2 \frac{W}{N} \frac{\left(1-1 / M_{p k}{ }^2\right)^2}{M_{k p}}
\end{aligned}
\end{equation}

\noindent коэффициенты смешивания определяются как

\begin{equation}
\begin{aligned}
M_{p k} & =\frac{1}{2\left(f_{\max }+f_{\min }\right) \eta_{p k} T_{p k} \frac{\Delta p}{p}}, \\
M_{k p} & =\frac{1}{2\left(f_{\max }-f_{\min }\right) \eta_{k p} T_{k p} \frac{\Delta p}{p}}
\end{aligned}
\end{equation}

\noindent где $\eta_{pk}T_{pk}\frac{\Delta p}{p}$, $\eta_{kp}T_{kp}\frac{\Delta p}{p}$– относительные времена смещения частиц (перемешивание),  $\eta_{pk}$, $\eta_{kp}$ – коэффициенты проскальзывания, в первом приближении $\eta_{pk}=\alpha_{pk}-\sfrac{1}{\gamma^2}$, $\eta_{kp}=\alpha_{\ kp}-\sfrac{1}{\gamma^2}$, $\alpha_{pk}$, $\alpha_{\ kp}$ – локальные факторы расширения орбиты первого порядка, $T_{pk}$, $T_{kp}$ – абсолютные времена пролета между пикапом-киккером и киккером-пикапом соответственно.

Времена стохастического охлаждения ур. (4-5) зависят от соотношения эффективной плотности частиц $N$ к полосе пропускания системы охлаждения $W$ и свойств магнитооптики, а именно локальных факторов расширения орбиты $\alpha_{pk}$, $\alpha_{\ kp}$.  
\noindent Максимальное значение полосы частот $f_{max}$ ограничено критерием неперекрытия “Schottky”-полос пучка. В простейшем случае это условие может быть записано:

\begin{equation}
f_{max}<\frac{1}{\eta_{pk}T_{pk}\frac{\Delta p}{p}}
\end{equation}	

\noindent при выполнении которого всегда фактор смешивания $M_{pk}>1$. В обратном случае, эффективность охлаждения становится нулевой. Таким образом, при заданном числе частиц желательно иметь полосу частот максимально возможной. С точки зрения электроники современные технологии позволяют реализовать полосу частот $10$ ГГц \cite{caspers:stochastic}, однако использование ее не всегда возможно из-за большой величины коэффициента проскальзывания $\eta_{pk}$ и разброса по импульсам $\frac{\Delta p}{p}$.

\noindent Уравнение (3) выведено для непрерывного (несгруппированного) пучка.  Эффективное число частиц, для случая сгустка, сформированного гармоническим одночастотным ВЧ резонатором, плотность частиц описывается распределением по Гауссу

\begin{equation}
\rho(s)=\frac{N_{bunch}}{\sigma_{bunch}\sqrt{2\pi}}\cdot e^{-\frac{s^2}{2\sigma_{bunch}^2}}\ \ \ 
\end{equation}	

\noindent где $s$ – расстояние от центра сгустка, $\sigma_{bunch}$ – дисперсия распределения частиц и $N_{bunch}$ – число частиц в сгустке. Если принять, что охлаждение определяется его минимальным значением в центре сгустка ($s=0$), то эффективное значение частиц на орбите длиной $C_{orb}$ равно:

\begin{equation}
N=\int_{0}^{C_{orb}}{\rho_{max}ds}=\frac{N_{bunch}}{\sqrt{2\pi}\sigma_{bunch}}\cdot C_{orb}
\end{equation}

\noindent Для сгустка, сформированного мультигармонической ВЧ системой барьерного типа («Barrier Bucket»), распределение частиц в сгустке близко к однородному с длиной сгустка $l_{bunch}=4\sigma_{bunch}$. Эффективное значение частиц определяется простым соотношением длины сгустка к общей длине орбиты:

\begin{equation}
N=\frac{N_{bunch}}{{4\sigma}_{bunch}}\cdot C_{orb}
\end{equation}

\noindent Подводя итог, можно сказать, что эффективное значение частиц зависит от распределения и определяется форм-фактором $F_{bunch}$, лежащим в пределах $F_{bunch}=\sqrt{2\pi}\div4$

\begin{equation}
N=N_{bunch}\cdot\frac{C_{orb}}{F_{bunch}\cdot\sigma_{bunch}}
\end{equation}

\noindent Для NICA  примем максимальный фактор $F_{bunch}=4$, и при ее ориентировочных параметрах $C_{orb}=503.04$ м, $\sigma_{bunch}=0.6$ м, $N_{bunch}=2.2\cdot{10}^9$. С учетом опыта работы FNAL \cite{church:stochastic} вполне реалистичные значения для полосы частот являются $f_{max}=8$ ГГц и $f_{min}=2$ ГГц. Для NICA выбрано $f_{max}=4$ ГГц и $f_{min}=2$ ГГц. При таких параметрах максимальная достижимая скорость охлаждения $\sfrac{1}{\tau_{tr}}=\sfrac{1}{230}$ c$^{-1}$.

Исходя из уравнений 6-7, видно, что может происходить асимптотический рост в двух случаях:
\begin{enumerate}
\item при приближении коэффициента проскальзывания к значению $\eta\rightarrow\frac{1}{2\left(f_{max}+f_{min}\right)T_{pk}\frac{\Delta p}{p}}$, Schottky-спектр пучка становится сплошным и $M_{pk}\rightarrow1$;
\item при приближении коэффициента проскальзывания к нулю, перемешивание на пути от киккера к пикапу не происходит и $M_{kp}\rightarrow\infty$.
\end{enumerate}

\noindent Эффективность стохастического охлаждения зависит от свойств магнитооптики. В классических “обычных” структурах энергия перехода передается через горизонтальную частоту, и коэффициент проскальзывания $\eta=1/\gamma_{tr}^2-1/\gamma^2$ может достигать нуля. Чтобы избежать асимптотического роста, необходимо изменять коэффициент проскальзывания, что означает вариацию $\gamma_{tr}$. Это возможно в “резонансной” структуре, где энергия перехода может быть увеличена или даже достигать комплексного значения \cite{senichev:resonant}. В более экзотическом случае может быть использована “комбинированная” структура, где $\eta_{pk}$ (пикап-кикер) с реальной критической энергией на одной арке

\begin{equation}
\eta_{pk}=1/\gamma_{tr}^2-1/\gamma^2
\end{equation}

\noindent компенсируется $\eta_{kp}$ (киккер-пикап) с комплексным значением в другой арке соответственно

\begin{equation}
\eta_{kp}=-1/\gamma_{tr}^2-1/\gamma^2\ \ \ 
\end{equation}

\noindent для всего кольца. При такой конструкции достигается требуемое соотношение факторов смешивания для максимальной скорости охлаждения, близкой к идеальной \cite{senichev:hesr}. Рассмотрим заявленные структуры более подробно.

\noindent Поведение $\beta$-функций и $D$ дисперсия вдоль всей “регулярной” структуры показаны на рисунке. 1. Прямые участки остаются неизменными во всех структурах, необходимы для анализа резонансных характеристик всей конструкции. Их расположение не влияет на внутрипучковое рассеяние и критическую энергию. Для подавления дисперсии в “регулярной” структуре с обеих сторон арок реализована технология 'missing magnet' (‘отсутствующих магнитов’).

\begin{figure}[!h]
  \centering
   \includegraphics*[width=1.0\columnwidth]{1_regular}
   \caption{Регулярная ФОДО структура.}
   \label{fig:1_regular}
\end{figure}

\noindent “Резонансная” структура основана на принципе резонансной модуляции дисперсионной функции \cite{senichev:construction} и может быть получена из "регулярной" структуры путем разделения фокусирующих квадруполей на 2 семейства с различными градиентами. Таким образом, критическая энергия может быть скорректирована таким образом, чтобы увеличить ее по сравнению с энергией эксперимента, избегая проблем с пересечением критической энергии. Для подавления дисперсии можно использовать либо два краевых фокусирующих квадруполя по обе стороны дуги, либо только два семейства фокусирующих квадруполей на дуге \cite{Kolokolchikov:2021trans}, когда достигается целое число бетатронных колебаний (рис. 2).

\begin{figure}[!h]
  \centering
   \includegraphics*[width=1.0\columnwidth]{1_resonant}
   \caption{"Резонансная" магнитооптическая структура с повышенной критической энергией.}
   \label{fig:1_resonant}
\end{figure}

\noindent В случае “комбинированной” конструкции одна дуга работает в обычном режиме, в то время как другая использует резонансную модуляцию (рис. 3). Такой выбор основан на принципе компенсации, описанном уравнениями 13 и 14, который требует большей глубины модуляции квадруполей, чем в чисто "резонансной" структуре с повышенной критической энергией.

\begin{figure}[!h]
  \centering
   \includegraphics*[width=1.0\columnwidth]{1_combined}
   \caption{"Резонансная" магнитооптическая структура с реальной и комплексной критической энергией в арках.}
   \label{fig:1_combined}
\end{figure}

\noindent Как показано на рисунке 4, "резонансная" оптика с увеличенной критической энергией, вторая асимптотика имеет более высокую энергию по сравнению с “регулярной” структурой. В “комбинированной” магнитооптике эффективность охлаждения близка к идеальному значению в широком диапазоне энергий от 2,5 до 4,5 ГэВ, в то время как в “обычной” оптике скорость охлаждения почти в два раза ниже в наиболее оптимальной точке ~3 ГэВ. Такое поведение объясняется отсутствием второй точки асимптотического роста.

\begin{figure}[!h]
  \centering
   \includegraphics*[width=0.495\columnwidth]{1_SC}
   \includegraphics*[width=0.495\columnwidth]{1_SC_wide}
   \caption{Зависимость времени охлаждения от энергии}
   \label{fig:1_SC}
\end{figure}

\subsection{Внутрипучковое рассеяние в регулярной, резонансной и комбинированной структурах}\label{sec:ions_light/IBS_res_reg}

\par  Как было уже сказано, внутрипучковое рассеяние является основным фактором, ограничивающим время жизни пучка в коллайдере. Поэтому критерием для использования того или иного способа охлаждения является сравнение их характерных времен с временем разогрева пучка из-за внутрипучкового рассеяния. Из общей теории этого явления следует:

\begin{equation}
\frac{1}{\tau_{IBS}}=\frac{\sqrt\pi}{4}\frac{cZ^2r_p^2L_C}{A}\cdot\frac{N}{C_{\mathrm{orb\ }}}\cdot\frac{\left\langle\beta_x\right\rangle}{\beta^3\gamma^3\varepsilon_x^{5/2}\left\langle\sqrt{\beta_x}\right\rangle}\left(\left\langle\frac{D_x^2+{\dot{D}}_x^2}{\beta_x^2}\right\rangle-\frac{1}{\gamma^2}\right)
\end{equation}

\noindent в отличии от стохастического охлаждения скорость разогрева из-за внутрипучкового рассеяния растет с уменьшением энергии как $1/\gamma^3$. Кроме того, выражение, стоящее в круглых скобках, пропорционально коэффициенту проскальзывания $\eta$. Поэтому следует ожидать, что в оптике со значением $\eta$ близким к нулю скорость разогрева должна падать. 

\begin{figure}[!h]
  \centering
   \includegraphics*[width=0.75\columnwidth]{1_IBS}
   \caption{Зависимость постоянной времени разогрева пучка из-за внутрипучкового рассеяния в регулярной, «резонансной» и комбинированной структурах от энергии пучка.}
   \label{fig:1_IBS}
\end{figure}

\noindent На рисунке 4 показаны зависимости постоянной времени нагрева в трех вышеупомянутых структурах, посчитанных с помощью программ MADX \cite{madx, antoniou:ibs} для параметров тяжелоионного пучка ${_{79}^{197}}Au$ коллайдера NICA c максимальной светимостью ${10}^{27}$ см$^{-2}$с$^{-1}$

\noindent Из сравнения времени разогрева со временем охлаждения (см. рис. 4) можно сделать заключение, что в регулярной структуре стохастическое охлаждение способно сбалансировать внутрипучковое рассеяние в диапазоне энергий $W\geq4.5$ ГэВ. Для применения стохастического охлаждения во всем диапазоне энергий очевидно, что мы должны пожертвовать светимостью пучка на низких энергиях посредством увеличения эмиттанса. В резонансных структурах, время внутрипучкового рассеяния значительно меньше. Это объясняется тем, что структура имеет большее соотношение между дисперсией и $\beta$-функцией пучка $\left\langle\frac{D_x^2+{\dot{D}}_x^2}{\beta_x^2}\right\rangle$, чем в случае регулярной. Таким образом, для тяжелоионной опции должна быть использована структура с максимально регулярным $\beta$-функцией и $D$ дисперсией (минимально модулированы). Для охлаждения пучка до $4.5$ ГэВ в регулярной структуре используется электронное охлаждение \cite{kostromin:stochastic}.

\section{Выбор критической энергии в магнитооптической структуре с учетом ускорения тяжелых ионов и легких частиц.}\label{sec:ch:ions_light/transition}

\par В случае легких ядер (протоны и дейтроны), время внутрипучкового рассеяния значительно вырастает, поскольку заряд становится меньше. Таким образом, проблема внутрипучкового рассеяния имеет значение для тяжелоионного сгустка с высокой зарядностью.

\subsection{Критическая энергия}\label{sec:ch:ions_light/transition/energy}
\par Поскольку понятия \textit{критическая энергия} (transition energy), \textit{коэффициент уплотнения орбиты} (momentum compaction factor) и \textit{коэффициент проскальзывания} (slip-factor) одни из ключевых и часто упоминаемых  в данной работе, поэтому уделим особое внимание при их определении.
\par Рассмотрим классическое уравнение продольного движение, описывающее эволюцию частицы отклоненной от референсной:

\begin{equation}
\begin{cases}
\begin{aligned}
& \frac{d \tau}{d n}=\eta(\delta) \cdot \frac{T_{0} \cdot h  \cdot \Delta E}{\beta^2 \cdot E_0} \\
& \frac{d(\Delta E)}{d n}=V(\tau)
\end{aligned}
\end{cases}
\label{eq:long_motion_eq_n}
\end{equation}

\noindent где $\tau$ -- временное отклонение рассматриваемой частицы от референсной, $\Delta E$ -- отклонение рассматриваемой частицы от референсной по энергии, $E_0$ -- энергия референсной частицы, $\omega_0=\sfrac{2\pi}{T_0}$ -- угловая частота и соответствующее время обращения референсной частицы, $\beta$ -- относительная скорость, $h$ -- гармоническое число, $V(\tau)$ -- функция определяющая амплитуду ВЧ для рассматриваемой частицы, $\eta$ -- коэффициент проскальзывания (в англоязычной терминологии 'slip-factor'). Индекс $0$ имеет значение референсной частицы.

\noindent Коэффициент проскальзывания является временным показателем запаздывания или опережения рассматриваемой частицы от референсной. Для его определения сначала рассмотрим зависимость удлинения орбиты от разброса по импульсам

\begin{equation}
C(\delta)=C_{0}(1+\alpha_{0}\delta+\alpha_{1}\delta^2+\cdots) = C_{0}(1+\alpha_{0}\delta+O(\delta^2)),
\label{eq:cdelta}
\end{equation} 

\noindent где также вводится понятие важное понятие коэф\-фи\-ци\-ента расширения орбиты (momentum compaction factor) \cite{lee}:

\begin{equation}
\alpha_c=\frac{1}{C_0} \frac{d C}{d \delta}=\alpha_0+2 \alpha_1 \delta+3 \alpha_2 \delta^2+\cdots \equiv \frac{1}{\gamma_{tr}^2},
\label{eq:alpha}
\end{equation}

\noindent тут $\gamma_{tr}$ иммет значение Лоренц-фактора при энергии пучка равной критическому значению или просто называется критической энергией. Таким образом, взяв во внимание, что $T=\frac{C}{v}=\frac{C}{\beta c}$, в первом приближении коэффициент проскальзывания может быть определен как

\begin{equation}
\frac{\Delta T}{T_{0}} = \frac{\Delta C}{C_{0}} - \frac{\Delta v}{v_{0}} = \eta \delta.
\label{eq:slip-factor_first}
\end{equation}

\noindent Однако, это справедливо только в первом приближении и приводит к выражению

\begin{equation}
\eta = \eta_{0} = \alpha_{0} - \frac{1}{\gamma_{0}^2}.
\label{eq:slip-factor_0}
\end{equation}

\noindent Отсюда из ур.\ref{eq:slip-factor_0} видно, что $\eta_{0}\rightarrow 0$ стремится к нулю при приближении $\gamma_{0}\rightarrow\gamma_{tr}$ правая часть уравнения ур.\ref{eq:long_motion_eq_n} также стремится к нулю. Возникает необsходимость обеспечения стабильности продольного движения при прохождении критической энергии. Поэтому при движении вблизи критической энергии также учитывается и влияние следующих порядков разложения, сравнимых по величине с первым. И определение \ref{eq:slip-factor_0} становится неточным. Для наиболее точного определения высших порядков коэффициента проскальзывания может быть использовано следующее соотношение \cite{ng}.

\begin{equation}
\frac{\Delta T_{n+1}}{T_{n+1}}=\eta_{n+1} \delta_{n+1},
\end{equation}

\noindent тут индекс $n+1$ отражает $n+1$-ое прохождение, а не порядок в разложении. Окончательно для коэффициента проскальзывания в зависимости от высших порядков разложения:

\begin{equation}
\eta(\delta)=\eta_0+\eta_1 \delta+\eta_2 \delta^2+\cdots,
\label{eq:eta}
\end{equation}

\noindent где $\eta_1=\frac{3\beta_0^2}{2\gamma_0^2}+\alpha_1-\alpha_0\eta_0, \eta_2=-\frac{\beta_0^2\left(5 \beta_0^2-1\right)}{2 \gamma_0^2}+\alpha_2-2 \alpha_0 \alpha_1+\frac{\alpha_1}{\gamma_0^2}+\alpha_0^2 \eta_0-\frac{3 \beta_0^2 \alpha_0}{2 \gamma_0^2}$

\subsection{Адаптация структуры для эксперимента с легкими поляризованными частицами}\label{sec:ch:ions_light/transition/energy}

\par Из-за соотношения заряда к массе, максимальная энергия протонного пучка становится порядка $13$ ГэВ. При этом, критическая энергия регулярной структуры, являющаяся характеристикой магнитооптической структуры ускорителя, составляет $5.7$ ГэВ. Таким образом, в регулярной структуре возникает необходимость преодоления критической энергии. Классическим способом является – скачок критической энергии \cite{Kolokolchikov:2024_bb_rupac}. Однако, в этом случае накладываются существенные ограничения на параметры сгустка \cite{Kolokolchikov:2024_bb_dspin}. Альтернативным способом является повышение критической энергии с использованием резонансной магнитооптической структуры. В этом случае происходит суперпериодическая модуляция дисперсионной функции, путем введения дополнительного семейства фокусирующих квадруполей.

\section*{Выводы}
\par Рассмотрены принципы реализации дуальной магнитооптическая структуры. 

\begin{enumerate}

\item  Для легких частиц из-за соотношения заряда к массе энергия эксперимента может превышать критическую энергию установки, которая является оптимальной для тяжелых ионов. При использовании дисперсионной модуляции критическая энергия увеличивается или даже достигает комплексного значения в резонансной или комбинированной магнитооптической структуре;

\item Показано, что вследствие модуляции $\beta$-функции и $D(s)$ дисперсии уменьшается время внутрипочкового рассеяния, что имеет решающее значение для многозарядных тяжелых частиц. По этой причине регулярная магнитооптическая структура с минимально модулированной $D(s)$ дисперсией и $\beta$-функцией оптимальна в режиме работы с тяжелыми ионами. Несмотря на то, что стохастическое охлаждение в регулярной структуре значительно слабее, чем в резонансной и комбинированной, оно может компенсировать эффект от внутрипучкового рассеяния;

\item Дуальная магнитооптическая структура предлагается для ускорения пучков как тяжелых ионов, так и легких частиц, что показано на примере установки NICA. Для преобразования регулярной структуры, оптимизированой для ускорения тяжелых ионов с точки зрения времени жизни пучка, в резонансную, с варьируемой критической энергией для легких частиц, не требуется никаких особых изменений, достаточно лишь внести отдельное семейство квадруполей.

\end{enumerate}

\FloatBarrier
