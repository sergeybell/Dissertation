%part 1

\chapter{Особенности двойственной магнитооптической структуры коллайдера NICA для ускорения тяжелых ионов и легких частиц частиц}\label{ch:ions_light}

Абстракт

	Коллайдер NICA будет использован как для проведения коллайдерных экспериментов с тяжелыми ионами, так и легкими поляризованными ядрами. Различное соотношения заряда к массе является существенным при проектировании магнитооптики. Для достижения высокой светимости должно быть гарантировано достаточное время жизни пучка. Также должна быть решена проблема прохождения критической энергии.	

\section{Дуальность магнитооптической структуры NICA для тяжелых ионов и легких ядер}\label{sec:ch:ions_light/duality}

Введение
	
	Для проведения коллайдерных экспериментов, необходимо гарантировать достаточное время жизни пучка [1]. Кроме того, для формирования конечного сгустка, удовлетворяющего требованиям высокой светимости, должна быть решена проблема прохождения критической энергии. [2] Оба этих, ключевых параметра, определяются спроектированной магнитооптической структурой.
Дуальная магнитооптическая структура предполагает возможность ускорения как тяжелых ионов (например, золото), так и легких частиц (протоны, дейтроны). Вследствие различного соотношения заряда к массе различается подход к проектированию магнитооптической структуры.


	\subsection{Оптимизация времени жизни пучка}

Высокое время жизни светимости пучка в коллайдерном эксперименте достигается путем минимизации эффекта внутрипучкового рассеяния, а также применения стохастического и электронного охлаждений пучка. Это особенно важно для тяжелоионных сгустков. Изменение эмиттанса и разброса по импульсам во времени при наличии охладителя описывается уравнениями

\begin{equation}
\begin{aligned}
& \frac{d \varepsilon}{d t}=\underbrace{-\frac{1}{\tau_{t r}} \cdot \varepsilon}_{\text {cooling }}+\underbrace{\left(\frac{d \varepsilon}{d t}\right)_{I B S}}_{\text {heating }} \\
& \frac{d \delta^2}{d t}=\underbrace{-\frac{1}{\tau_{\text {long }}} \cdot \delta^2}_{\text {cooling }}+\underbrace{\left(\frac{d \delta^2}{d t}\right)_{\text {IBS }}}_{\text {heating }} \\
&
\end{aligned}
\end{equation}

\noindent где $\varepsilon$ – поперечный эмиттанс, $\tau_{tr}$ – поперечное время охлаждения, $\delta=\frac{\Delta p}{p}$ – разброс по импульсам, $\tau_{\mathrm{long\ }}$– продольное время охлаждения.
Для независимых от времени, стационарных значений, производные по времени становятся равными нулю, тогда

\begin{equation}
\begin{aligned}
& \varepsilon_{s t}=\left.\tau_{t r} \cdot\left(\frac{d \varepsilon}{d t}\right)_{I B S}\right|_{\mathcal{E}=\varepsilon_{s t}} \\
& \delta_{s t}^2=\left.\tau_{\text {long }} \cdot\left(\frac{d \delta^2}{d t}\right)_{I B S}\right|_{\delta^2=\delta_{s t}^2}
\end{aligned}
\end{equation}

Критерием применимости того, или иного метода охлаждения может быть сравнение характерных времен стохастического и электронного охлаждения со временем жизни с учетом IBS во всем предполагаемом диапазоне энергий.
	
		\subsection{Стохастическое охлаждение}

\par Рассмотрим стохастическое охлаждение, пользуясь приближенной теорией D.Mohl [3,4]. Следуя его основным выводам, скорость охлаждения определяется выражением		
	
\begin{equation}
\frac{1}{\tau_{t r, l}}=\frac{W}{N}[\underbrace{2 g \cos \theta\left(1-1 / M_{p k}^2\right)}_{\begin{array}{c}
\text { coherent } \\
\text { effect(cooling) }
\end{array}}-\underbrace{g^2\left(M_{k p}+U\right)}_{\begin{array}{c}
\text { incoherent } \\
\text { effect(heating) }
\end{array}}]
\end{equation}	

\noindent где $W=f_{max}-f_{min}$ – пропускная способность системы, $N$ – эффективное число частиц, пересчитанное через соотношение орбиты к длине сгустка с учетом его распределения, $g$ – fraction of observed sample error corrected per turn, $U=E({x_n}^2)/E(xs2)$ – отношение шума к сигналу, $M_{pk}$, $M_{kp}$– факторы смешивания между пикапом – киккером и киккером – пикапом соответственно.

\noindent Уравнение (3) в отсутствии шума при $g=g_0={\frac{1-{M_{pk}}^2}{M_{kp}}}$ достигает максимум

\begin{equation}
\begin{aligned}
& \frac{1}{\tau_{t r}}=\frac{W}{N} \frac{\left(1-1 / M_{p k}^2\right)^2}{M_{k p}} \\
& \frac{1}{\tau_l}=2 \frac{W}{N} \frac{\left(1-1 / M_{p k}{ }^2\right)^2}{M_{k p}}
\end{aligned}
\end{equation}

\noindent коэффициенты смешивания определяются как

\begin{equation}
\begin{aligned}
M_{p k} & =\frac{1}{2\left(f_{\max }+f_{\min }\right) \eta_{p k} T_{p k} \frac{\Delta p}{p}}, \\
M_{k p} & =\frac{1}{2\left(f_{\max }-f_{\min }\right) \eta_{k p} T_{k p} \frac{\Delta p}{p}}
\end{aligned}
\end{equation}

\noindent где $\eta_{pk}T_{pk}\frac{\Delta p}{p}$, $\eta_{kp}T_{kp}\frac{\Delta p}{p}$– относительные времена смещения частиц (перемешивание),  $\eta_{pk}$, $\eta_{kp}$ – коэффициенты проскальзывания, в первом приближении $\eta_{pk}=\alpha_{pk}-\sfrac{1}{\gamma^2}$, $\eta_{kp}=\alpha_{\ kp}-\sfrac{1}{\gamma^2}$, $\alpha_{pk}$, $\alpha_{\ kp}$ – локальные факторы расширения орбиты первого порядка, $T_{pk}$, $T_{kp}$ – абсолютные времена пролета между пикапом-киккером и киккером-пикапом соответственно.

Времена стохастического охлаждения ур. (4-5) зависят от соотношения эффективной плотности частиц $N$ к полосе пропускания системы охлаждения $W$ и свойств магнитооптики, а именно локальных факторов расширения орбиты $\alpha_{pk}$, $\alpha_{\ kp}$.  
\noindent Максимальное значение полосы частот $f_{max}$ ограничено критерием неперекрытия “Schottky”-полос пучка. В простейшем случае это условие может быть записано:

\begin{equation}
f_{max}<\frac{1}{\eta_{pk}T_{pk}\frac{\Delta p}{p}}
\end{equation}	

\noindent при выполнении которого всегда фактор смешивания $M_{pk}>1$. В обратном случае, эффективность охлаждения становится нулевой. Таким образом, при заданном числе частиц желательно иметь полосу частот максимально возможной. С точки зрения электроники современные технологии позволяют реализовать полосу частот $10$ ГГц [5], однако использование ее не всегда возможно из-за большой величины коэффициента проскальзывания $\eta_{pk}$ и разброса по импульсам $\frac{\Delta p}{p}$.

\noindent Уравнение (3) выведено для непрерывного (несгруппированного) пучка.  Эффективное число частиц, для случая сгустка, сформированного гармоническим одночастотным ВЧ резонатором, плотность частиц описывается распределением по Гауссу

\begin{equation}
\rho(s)=\frac{N_{bunch}}{\sigma_{bunch}\sqrt{2\pi}}\bullet e^{-\frac{s^2}{2\sigma_{bunch}^2}}\ \ \ 
\end{equation}	

\noindent где $s$ – расстояние от центра сгустка, $\sigma_{bunch}$ – дисперсия распределения частиц и $N_{bunch}$ – число частиц в сгустке. Если принять, что охлаждение определяется его минимальным значением в центре сгустка ($s=0$), то эффективное значение частиц на орбите длиной $C_{orb}$ равно:

\begin{equation}
N=\int_{0}^{C_{orb}}{\rho_{max}ds}=\frac{N_{bunch}}{\sqrt{2\pi}\sigma_{bunch}}\bullet C_{orb}
\end{equation}

\noindent Для сгустка, сформированного мультигармонической ВЧ системой барьерного типа («Barrier Bucket»), распределение частиц в сгустке близко к однородному с длиной сгустка $l_{bunch}=4\sigma_{bunch}$. Эффективное значение частиц определяется простым соотношением длины сгустка к общей длине орбиты:

\begin{equation}
N=\frac{N_{bunch}}{{4\sigma}_{bunch}}\bullet C_{orb}
\end{equation}

\noindent Подводя итог, можно сказать, что эффективное значение частиц зависит от распределения и определяется форм-фактором $F_{bunch}$, лежащим в пределах $F_{bunch}=\sqrt{2\pi}\div4$

\begin{equation}
N=N_{bunch}\bullet\frac{C_{orb}}{F_{bunch}\bullet\sigma_{bunch}}
\end{equation}

\noindent Для NICA  примем максимальный фактор $F_{bunch}=4$, и при ее ориентировочных параметрах $C_{orb}=503.04$ м, $\sigma_{bunch}=0.6$ м, $N_{bunch}=2.2\bullet{10}^9$. С учетом опыта работы FNAL [6] вполне реалистичные значения для полосы частот являются $f_{max}=8$ ГГц и $f_{min}=2$ ГГц. Для NICA выбрано $f_{max}=4$ ГГц и $f_{min}=2$ ГГц. При таких параметрах максимальная достижимая скорость охлаждения $\sfrac{1}{\tau_{tr}}=\sfrac{1}{230}$ c$^{-1}$.

\noindent Однако эффективность стохастического охлаждения в значительной степени зависит от свойств магнитооптики. Рассмотрим 3 возможных вариации:
\begin{enumerate}
\item Регулярная ФОДО структура;
\item «Резонансная» магнитооптическая оптика с повышенной критической энергией;
\item «Резонансная» магнитооптическая с реальной и комплексной энергиями.
\end{enumerate}

\par Регулярная ФОДО структура
 \par В таких структурах $\gamma_{tr}\approx\nu_x$, критическая энергия набирается за счет горизонтальной частоты. На рис. 1 показано поведение $\beta$-функций и $D$ дисперсии на всем кольце.  Прямые участки необходимы для анализа резонансных свойств всей структуры, их устройство не влияет на внутрипучковое рассеяние и критическую энергию. Для подавления дисперсии в регулярной структуре используются два крайних фокусирующих магнита с обоих сторон арки.

\begin{figure}[!h]
  \centering
   \includegraphics*[width=1.0\columnwidth]{1_regular}
   \caption{Регулярная ФОДО структура.}
   \label{fig:1_regular}
\end{figure}


\par «Резонансная» структура
 \par Структура построена по принципу резонансной модуляции дисперсионной функции [7,8]. Таким образом может быть варьирована критическая энергия для её поднятия выше энергии эксперимента, можно избежать проблем с прохождением критической энергии. Для подавления дисперсии может быть использовано как два крайних фокусирующих магнита с обоих сторон арки, так и при помощи только двух семейств фокусирующих квадруполей на арке [9], при достижении целого числа бетатронных колебаний Рис. 2. Причем компенсация нелинейного вклада секступолей, подавление дисперсии на прямых участках, сохраняются во всем интервале значений критической энергии.

\begin{figure}[!h]
  \centering
   \includegraphics*[width=1.0\columnwidth]{1_resonant}
   \caption{"Резонансная" магнитооптическая структура с повышенной критической энергией.}
   \label{fig:1_resonant}
\end{figure}

\par «Резонансная» магнитооптическая с реальной и комплексной энергиями.
 \par С помощью резонансной модуляции дисперсионной функции может быть достигнуто и отрицательной значение первого порядка коэффициента уплотнения орбиты. Таким образом, критическая энергия может приобретать комплексное значение.  Если в первой и второй оптике в обеих арках коэффициенты проскальзывания имеют одинаковое значение, то во третей оптике создается арка с реальным значением критической энергии, коэффициент проскальзывания имеет минимальное значение

\begin{figure}[!h]
  \centering
   \includegraphics*[width=1.0\columnwidth]{1_combined}
   \caption{"Резонансная" магнитооптическая структура с реальной и комплексной критической энергией в арках.}
   \label{fig:1_combined}
\end{figure}

\begin{equation}
\eta_{pk}=1/\gamma_{tr}^2-1/\gamma^2
\end{equation}

в другой – комплексное значение соответственно и – максимальное значение коэффициента проскальзывания

\begin{equation}
\eta_{kp}=-1/\gamma_{tr}^2-1/\gamma^2\ \ \ 
\end{equation}

чем достигается требуемое соотношение факторов смешивания (ур. 6-7) для максимальной скорости охлаждения, близкой к идеальной [10].

Очевидно, что с учетом зависимости факторов смешивания от энергии, эффективность охлаждения также зависит от энергии. На рисунке 4 показана зависимость времени охлаждения от энергии пучка для трех оптик NICA. Асимптотический рост может происходить в двух случаях:

\begin{enumerate}
\item Исходя из уравнения 6 при приближении коэффициента проскальзывания к значению $\eta\rightarrow\frac{1}{2\left(f_{max}+f_{min}\right)T_{pk}\frac{\Delta p}{p}}$, Schottky-спектр пучка становится сплошным и $M_{pk}\rightarrow1$;
\item При приближении коэффициента проскальзывания к нулю, перемешивание на пути от киккера к пикапу не происходит и $M_{kp}\rightarrow\infty$.
\end{enumerate}

\begin{figure}[!h]
  \centering
   \includegraphics*[width=0.495\columnwidth]{1_SC}
   \includegraphics*[width=0.495\columnwidth]{1_SC_wide}
   \caption{Зависимость времени охлаждения от энергии}
   \label{fig:1_SC}
\end{figure}

Как мы видим в «резонансной» оптике с повышенной критической энергией вторая асимптотика при большей энергии по сравнению с регулярной структурой. В магнитооптике с реальной и комплексной энергиями эффективность охлаждения ближе к идеальному значению в большом диапазоне энергий от $2.5$ до $4.5$ ГэВ, в то время как в регулярной оптике скорость охлаждения почти в два раза ниже в самой оптимальной точке $\sim3$ ГэВ. Такое поведение объясняется отсутствием второй точки асимптотического роста в виду невозможности прохождения в «комбинированной» структуре с комплексной аркой через критическую энергию.

\subsection{Внутрипучковое рассеяние}

\par  Как было уже сказано, внутрипучковое рассеяние является основным фактором, ограничивающим время жизни пучка в коллайдере. Поэтому критерием для использования того или иного способа охлаждения является сравнение их характерных времен с временем разогрева пучка из-за внутрипучкового рассеяния. Из общей теории этого явления следует:

\begin{equation}
\frac{1}{\tau_{IBS}}=\frac{\sqrt\pi}{4}\frac{cZ^2r_p^2L_C}{A}\cdot\frac{N}{C_{\mathrm{orb\ }}}\cdot\frac{\left\langle\beta_x\right\rangle}{\beta^3\gamma^3\varepsilon_x^{5/2}\left\langle\sqrt{\beta_x}\right\rangle}\left(\left\langle\frac{D_x^2+{\dot{D}}_x^2}{\beta_x^2}\right\rangle-\frac{1}{\gamma^2}\right)
\end{equation}

в отличии от стохастического охлаждения скорость разогрева из-за внутрипучкового рассеяния растет с уменьшением энергии как $1/\gamma^3$. Кроме того, выражение, стоящее в круглых скобках, пропорционально коэффициенту проскальзывания $\eta$. Поэтому следует ожидать, что в оптике со значением $\eta$ близким к нулю скорость разогрева должна падать. 

\begin{figure}[!h]
  \centering
   \includegraphics*[width=0.75\columnwidth]{1_IBS}
   \caption{Зависимость постоянной времени разогрева пучка из-за внутрипучкового рассеяния в регулярной, «резонансной» и комбинированной структурах от энергии пучка.}
   \label{fig:1_IBS}
\end{figure}

На рисунке 4 показаны зависимости постоянной времени нагрева в трех вышеупомянутых структурах, посчитанных с помощью программ MADX [11] для параметров тяжелоионного пучка ${_{79}^{197}}Au$ коллайдера NICA c максимальной светимостью ${10}^{27}$ см$^{-2}$с$^{-1}$

Из сравнения времени разогрева со временем охлаждения (см. рис. 4) можно сделать заключение, что в регулярной структуре стохастическое охлаждение способно сбалансировать внутрипучковое рассеяние в диапазоне энергий $W\geq4.5$ ГэВ. Для применения стохастического охлаждения во всем диапазоне энергий очевидно, что мы должны пожертвовать светимостью пучка на низких энергиях посредством увеличения эмиттанса. В резонансных структурах, время внутрипучкового рассеяния значительно меньше. Это объясняется тем, что структура имеет большее соотношение между дисперсией и $\beta$-функцией пучка $\left\langle\frac{D_x^2+{\dot{D}}_x^2}{\beta_x^2}\right\rangle$, чем в случае регулярной. Таким образом, для тяжелоионной опции должна быть использована структура с максимально регулярным $\beta$-функцией и $D$ дисперсией (минимально модулированы). Для охлаждения пучка до $4.5$ ГэВ в регулярной структуре используется электронное охлаждение [12].

\subsection{Протонная мода}

В случае легких ядер (протоны и дейтроны), время внутрипучкового рассеяния значительно вырастает, поскольку заряд становится меньше. Таким образом, проблема внутрипучкового рассеяния имеет значение для тяжелоионного сгустка с высокой зарядностью.
Однако, из-за соотношения заряда к массе, максимальная энергия протонного пучка становится порядка $13$ ГэВ. При этом, критическая энергия регулярной структуры, являющаяся характеристикой магнитооптической структуры ускорителя, составляет $5.7$ ГэВ. Таким образом, в регулярной структуре возникает необходимость преодоления критической энергии. Классическим способом является – скачок критической энергии [13]. Однако, в этом случае накладываются существенные ограничения на параметры сгустка [14]. Альтернативным способом является повышение критической энергии с использованием резонансной магнитооптической структуры. В этом случае происходит Суперпериодическая модуляция дисперсионной функции, путем введения дополнительного семейства фокусирующих квадруполей.



\section{Выбор критической энергии в магнитооптической структуре с учетом ускорения тяжелых ионов и легких частиц.}\label{sec:ch:ions_light/transition}

\subsection{Критическая энергия}\label{sec:ch:ions_light/transition/energy}
\par Понятие критическая энергия одно из ключевых в данной работе, поэтому уделим внимание к его определению. 
\par Рассмотрим классическое уравнение продольного движение, описывающее эволюцию частицы отклоненной от референсной:

\begin{equation}
\begin{aligned}
& \frac{d \tau}{d t}=\eta(\delta) \cdot \frac{h \cdot \Delta E}{\beta^2 \cdot E_0} \\
& \frac{d(\Delta E)}{d t}=\frac{V(\tau)}{T_0}
\end{aligned}
\label{eq:long_motion_eq}
\end{equation}

\noindent где $\tau$ – временное отклонение рассматриваемой частицы от референсной, $\beta$ – относительная скорость, $\omega_0=\sfrac{2\pi}{T_0}$– угловая частота и соответствующее время обращения, $h$ – гармоническое число, $V$ – ВЧ, $\eta$ коэффициент проскальзывания (в англоязычной терминологии 'slip-factor') 

\par При рассмотрении продольного движения вводится понятие коэф\-фи\-ци\-ента
расширения орбиты (momentum compaction factor) \cite{lee}:

\begin{equation}
\alpha_c=\frac{1}{R_0} \frac{d R}{d \delta}=\alpha_0+2 \alpha_1 \delta+3 \alpha_2 \delta^2+\cdots \equiv \frac{1}{\gamma_T^2}
\label{eq:alpha}
\end{equation}

и коэффициента скольжения (slip-factor):

\begin{equation}
\eta(\delta)=\eta_0+\eta_1 \delta+\eta_2 \delta^2+\cdots,
\label{eq:eta}
\end{equation}

\noindent где $\eta_0=\alpha_0-\frac{1}{\gamma_0^2}$, $\eta_1=\frac{3\beta_0^2}{2\gamma_0^2}+\alpha_1-\alpha_0\eta_0$.

\par В ур.\ref{eq:long_motion_eq}, если энергия пучка приближается к критической $\gamma\rightarrow\gamma_{tr},$  то $\eta=\eta_0\rightarrow0$, правая часть уравнения обращается в ноль. Возникает необходимость обеспечения стабильности при прохождении критической энергии.

\par критическая энергия 

\section{Решение проблемы внутрипучкового рассеяния для тяжелых ионов и протонов в ускорителе NICA }\label{sec:ions_light/IBS}

\section{IBS в «резонансной» и регулярной структурах}\label{sec:ions_light/IBS_res_reg}

\section{Заключение}

Рассмотрена дуальная магнитооптическая структура коллайдера NICA. Показано, что время стохастического охлаждения в резонансной и комбинированной структурах значительно меньше по сравнению с регулярной. Однако, вследствие модуляции $\beta$-функции и дисперсии, падает время внутрипучкового рассеяния. По этой причине в тяжелоионной опции оптимальной является регулярная магнитооптическая структура с минимально модулированными дисперсией и $\beta$sk-функцией. В случае протонов, важным является проблема преодоления критической энергии, для этого может быть использована “резонансная” или “комбинированная” магнитооптическая структура. Последние могут быть получены путем модуляции градиента в фокусирующих квадрупольных линзах.

\FloatBarrier
