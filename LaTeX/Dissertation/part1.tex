%part 1

\chapter{Особенности двойственной магнитооптической структуры коллайдера NICA для ускорения тяжелых ионов и легких частиц частиц}\label{ch:ions_light}

\section{Дуальность магнитооптической структуры NICA для тяжелых ионов и легких ядер}\label{sec:ch:ions_light/duality}

\section{Выбор критической энергии в магнитооптической структуре с учетом ускорения тяжелых ионов и легких частиц.}\label{sec:ch:ions_light/transition}

\subsection{Критическая энергия}\label{sec:ch:ions_light/transition/energy}
\par Понятие критическая энергия одно из ключевых в данной работе, поэтому уделим внимание к его определению. 
\par Рассмотрим классическое уравнение продольного движение, описывающее эволюцию частицы отклоненной от референсной:

\begin{equation}
\begin{aligned}
& \frac{d \tau}{d t}=\eta(\delta) \cdot \frac{h \cdot \Delta E}{\beta^2 \cdot E_0} \\
& \frac{d(\Delta E)}{d t}=\frac{V(\tau)}{T_0}
\end{aligned}
\label{eq:long_motion_eq}
\end{equation}

\noindent где $\tau$ – временное отклонение рассматриваемой частицы от референсной, $\beta$ – относительная скорость, $\omega_0=\sfrac{2\pi}{T_0}$– угловая частота и соответствующее время обращения, $h$ – гармоническое число, $V$ – ВЧ, $\eta$ коэффициент проскальзывания (в англоязычной терминологии 'slip-factor') 

\par При рассмотрении продольного движения вводится понятие коэф\-фи\-ци\-ента
расширения орбиты (momentum compaction factor) \cite{lee}:

\begin{equation}
\alpha_c=\frac{1}{R_0} \frac{d R}{d \delta}=\alpha_0+2 \alpha_1 \delta+3 \alpha_2 \delta^2+\cdots \equiv \frac{1}{\gamma_T^2}
\label{eq:alpha}
\end{equation}

и коэффициента скольжения (slip-factor):

\begin{equation}
\eta(\delta)=\eta_0+\eta_1 \delta+\eta_2 \delta^2+\cdots,
\label{eq:eta}
\end{equation}

\noindent где $\eta_0=\alpha_0-\frac{1}{\gamma_0^2}$, $\eta_1=\frac{3\beta_0^2}{2\gamma_0^2}+\alpha_1-\alpha_0\eta_0$.

\par В ур.\ref{eq:long_motion_eq}, если энергия пучка приближается к критической $\gamma\rightarrow\gamma_{tr},$  то $\eta=\eta_0\rightarrow0$, правая часть уравнения обращается в ноль. Возникает необходимость обеспечения стабильности при прохождении критической энергии.

\par критическая энергия 

\section{Решение проблемы внутрипучкового рассеяния для тяжелых ионов и протонов в ускорителе NICA }\label{sec:ions_light/IBS}

\section{IBS в «резонансной» и регулярной структурах}\label{sec:ions_light/IBS_res_reg}

\FloatBarrier
