%part 1

	\chapter{Особенности дуальной магнитооптической структуры}\label{ch:dual}

\par Независимо от назначения синхротрона, при наличии двух режимов, в которых ускоряются многозарядные тяжёлые частицы и одна или две лёгкие заряженные частицы, возникает задача определения оптимальной конфигурации магнитооптической структуры, которая обеспечит устойчивое движение обоих типов частиц. Многозарядные частицы, в отличие от лёгких, обладают более выраженным эффектом разогрева из-за внутрипучкового рассеяния \cite{trubnikov:cool}. А в случае лёгких частиц определяющим фактором является влияние критической энергии на динамику сгустка. Эти эффекты имеют существенное значение для коллайдеров, где светимость играет ключевую роль. При разработке структуры, удовлетворяющей требованиям, предъявляемым к частицам с различным зарядом, принципиально важно создать перестраиваемую структуру без внесения конструктивных изменений. Мы назвали такой подход -- проектированием дуальной структуры.

\par В комплексе NICA дуальная магнитооптическая структура открывает перспективу ускорения как тяжелых ионов, таких как золото, так и легких частиц, таких как протоны и дейтроны. Различное соотношения заряда к массе является существенным при проектировании магнитооптики.

	\section{Выбор критической энергии в магнитооптической структуре с учетом ускорения тяжелых ионов и легких частиц.}\label{sec:ch:ions_light/transition}

\par В классической регулярной структуре критическая энергия приблизительно равна горизонтальной бетатронной частоте $\gamma_{\text{tr}}\simeq\nu_{x}$. При одинаковой магнитной жесткости $B\rho$ максимальная энергия для легких частиц выше, чем для тяжелых ионов, из-за их соотношения заряд-масса. Это означает, что структура для тяжелых ионов, оптимизированная для работы до определенной критической энергии, потребует преодоления этой энергии для работы с легкими частицами. По этой причине можно рассмотреть структуру с изменяющейся критической энергией.

\subsection{Критическая энергия}\label{sec:ch:ions_light/transition/energy}
\par Поскольку понятия \textit{критическая энергия} (transition energy), \textit{коэффициент уплотнения орбиты} (momentum compaction factor) и \textit{коэффициент проскальзывания} (slip-factor) одни из ключевых и часто упоминаемых в данной работе, поэтому уделим особое внимание при их определении.
\par Рассмотрим классическое уравнение продольного движение, описывающее эволюцию частицы отклоненной от референсной \cite{lee}:

\begin{equation}
	\begin{cases}
		\begin{aligned}
			& \dv{\tau}{n}=\eta(\delta) \cdot \frac{T_{0} \cdot h  \cdot \Delta E}{\beta^2 \cdot E_0}, \\
			& \dv{(\Delta E)}{n}=V(\tau),
		\end{aligned}
	\end{cases}
	\label{eq:long_motion_eq_n}
\end{equation}

\noindent где $\tau$, $\Delta E$ -- временное и энергетическое отклонение рассматриваемой частицы от референсной по энергии, $E_0$ -- энергия референсной частицы, $n$ -- номер оборота, $\omega_0=\frac{2\pi}{T_0}$ -- угловая частота и соответствующее время обращения частицы, $\beta$ -- относительная скорость, $h$ -- гармоническое число, $V(\tau)$ -- функция определяющая амплитуду ВЧ для рассматриваемой частицы, $\eta$ -- коэффициент проскальзывания. Индекс $0$ имеет значение референсной частицы.

\par Коэффициент проскальзывания является временным показателем запаздывания или опережения рассматриваемой частицы от референсной. Для его определения сначала рассмотрим зависимость удлинения орбиты от разброса по импульсам

\begin{equation}
	C(\delta)=C_{0}(1+\alpha_{0}\delta+\alpha_{1}\delta^2+\cdots) = C_{0}(1+\alpha_{0}\delta+O(\delta^2)),
	\label{eq:cdelta}
\end{equation} 

\noindent где также вводится понятие важное понятие коэф\-фи\-ци\-ента расширения орбиты (momentum compaction factor) \cite{lee}:

\begin{equation}
	\alpha_c=\frac{1}{C_0} \dv{C}{\delta}=\alpha_0+2 \alpha_1 \delta+3 \alpha_2 \delta^2+\cdots \equiv \frac{1}{\gamma_{\textrm{tr}}^2},
	\label{eq:alpha}
\end{equation}

\noindent тут $\gamma_{\textrm{tr}}$ имеет значение Лоренц-фактора при энергии пучка равной критическому значению или просто называется критической энергией. Таким образом, взяв во внимание, что $T=\frac{C}{v}=\frac{C}{\beta c}$, в первом приближении коэффициент проскальзывания может быть определен как

\begin{equation}
	\frac{\Delta T}{T_{0}} = \frac{\Delta C}{C_{0}} - \frac{\Delta v}{v_{0}} = \eta \delta.
	\label{eq:slip-factor_first}
\end{equation}

\noindent Однако, это справедливо только в первом приближении и приводит к выражению

\begin{equation}
	\eta = \eta_{0} = \alpha_{0} - \frac{1}{\gamma_{0}^2}.
	\label{eq:slip-factor_0}
\end{equation}

\noindent Отсюда из ур. \ref{eq:slip-factor_0} видно, что $\eta_{0}\rightarrow 0$ стремится к нулю при приближении $\gamma_{0}\rightarrow\gamma_{\textrm{tr}}$ правая часть уравнения ур. \ref{eq:long_motion_eq_n} также стремится к нулю. Возникает необходимость обеспечения стабильности продольного движения при прохождении критической энергии. Поэтому при движении вблизи критической энергии также учитывается и влияние следующих порядков разложения, сравнимых по величине с первым. И определение \ref{eq:slip-factor_0} становится неточным. Для наиболее точного определения высших порядков коэффициента проскальзывания может быть использовано следующее соотношение \cite{ng}

\begin{equation}
	\frac{\Delta T_{n+1}}{T_{n+1}}=\eta_{n+1} \delta_{n+1},
\end{equation}

\noindent тут индекс $n+1$ отражает $n+1$-ое прохождение, а не порядок в разложении. Окончательно для коэффициента проскальзывания в зависимости от высших порядков разложения:

\begin{equation}
	\eta(\delta)=\eta_0+\eta_1 \delta+\eta_2 \delta^2+\cdots,
	\label{eq:eta}
\end{equation}

\noindent где $\eta_1=\frac{3\beta_0^2}{2\gamma_0^2}+\alpha_1-\alpha_0\eta_0,\quad\eta_2=-\frac{\beta_0^2\left(5 \beta_0^2-1\right)}{2 \gamma_0^2}+\alpha_2-2 \alpha_0 \alpha_1+\frac{\alpha_1}{\gamma_0^2}+\alpha_0^2 \eta_0-\frac{3 \beta_0^2 \alpha_0}{2 \gamma_0^2}$.

\subsection{Адаптация структуры для эксперимента с легкими поляризованными частицами}\label{sec:ch:ions_light/transition/energy}

\par Из-за соотношения заряда к массе, максимальная энергия протонного пучка становится порядка $13$ ГэВ. При этом критическая энергия регулярной структуры, являющаяся характеристикой магнитооптической структуры ускорителя, составляет $5.7$ ГэВ. Таким образом, в регулярной структуре возникает необходимость преодоления критической энергии. Классическим способом является скачок критической энергии \cite{Kolokolchikov:2024_bb_rupac}, что будет показано в Главе \ref{ch:transition}. Однако, в этом случае накладываются существенные ограничения на параметры сгустка \cite{Kolokolchikov:2024_bb_dspin}. Альтернативным способом является повышение критической энергии с использованием резонансной магнитооптической структуры, что будет показано в Главе \ref{ch:resonant}. В этом случае происходит суперпериодическая модуляция дисперсионной функции, путем введения дополнительного семейства фокусирующих квадруполей.

	\section{Оптимизация времени жизни пучка}
	
\par Достижение высокой светимости пучка в коллайдерных экспериментах требует обеспечения достаточного времени жизни пучка. Это достигается за счет уменьшения эффектов внутрипучкового рассеяния в сочетании с применением методов стохастического и электронного охлаждения. Данный подход особенно важен при работе с ионными пучками высокой интенсивности. Временная эволюция эмиттанса и разброса импульса в присутствии процессов охлаждения описывается набором уравнений

\begin{equation}
\begin{aligned}
& \dv{\varepsilon}{t} =\underbrace{-\frac{1}{\tau_{\textrm{tr}}} \cdot \varepsilon}_{\text {cooling}}+\underbrace{\left(\dv{\varepsilon}{t}\right)_{\textrm{IBS}}}_{\text {heating}}, \\
& \dv{\delta^2}{t} =\underbrace{-\frac{1}{\tau_{\text {long }}} \cdot \delta^2}_{\text {cooling }}+\underbrace{\left(\dv{\delta^2}{t}\right)_{\text {IBS }}}_{\text {heating }},\\
&
\end{aligned}
\end{equation}

\noindent где $\varepsilon$ -- поперечный эмиттанс, $\tau_{\textrm{tr}}$ -- поперечное время охлаждения, $\delta=\frac{\Delta p}{p}$ -- разброс по импульсам, $\tau_{\mathrm{long\ }}$ -- продольное время охлаждения.
Для независимых от времени, стационарных значений, производные по времени становятся равными нулю, тогда

\begin{equation}
\begin{aligned}
& \varepsilon_{\textrm{st}}=\left.\tau_{\textrm{tr}} \cdot\left(\dv{\varepsilon}{t}\right)_{\textrm{IBS}}\right|_{\varepsilon=\varepsilon_{\textrm{st}}}, \\
& \delta_{s t}^2=\left.\tau_{\text {long }} \cdot\left(\dv{\delta^2}{t}\right)_{\textrm{IBS}}\right|_{\delta^2=\delta_{\textrm{st}}^2}.
\end{aligned}
\end{equation}

\noindent Критерием применимости того, или иного метода охлаждения может быть сравнение характерных времен стохастического и электронного охлаждения со временем жизни с учетом ВПР во всем предполагаемом диапазоне энергий.
	
	\subsection{Стохастическое охлаждение}

\par Рассмотрим стохастическое охлаждение, пользуясь приближенной теорией D.Mohl \cite{mohl:stochastic, mohl:stochastic2}. Следуя его основным выводам, скорость охлаждения определяется выражением		
	
\begin{equation} 
\frac{1}{\tau_{\text{tr, l}}}=\frac{W}{N}[\underbrace{2 g \cos \theta\left(1-1 / M_{\textrm{pk}}^2\right)}_{\begin{array}{c}
\text {coherent} \\
\text {effect(cooling)}
\end{array}}-\underbrace{g^2\left(M_{\textrm{kp}}+U\right)}_{\begin{array}{c}
\text {incoherent} \\
\text {effect(heating)}
\end{array}}],
\label{eq:stochastic_rate}
\end{equation}	

\noindent где $W=f_{\text{max}}-f_{\text{min}}$ -- пропускная способность системы, $N$ -- эффективное число частиц, пересчитанное через соотношение орбиты к длине сгустка с учетом его распределения, $g$ -- доля наблюдаемой ошибки выборки, скорректированная за оборот, $U=E({x_n}^2)/E({x_s}^2)$ -- отношение шума к сигналу, $M_{\textrm{pk}}$, $M_{\textrm{kp}}$ -- факторы смешивания между пикапом -- киккером и киккером -- пикапом соответственно.

\noindent Уравнение \ref{eq:stochastic_rate} в отсутствии шума при $g=g_0={\frac{1-{M_{\textrm{pk}}}^2}{M_{\textrm{kp}}}}$ достигает максимум

\begin{equation}
\begin{aligned}
& \frac{1}{\tau_{\textrm{tr}}}=\frac{W}{N} \frac{\left(1-1 / {M_{\textrm{pk}}}^2\right)^2}{M_{\textrm{kp}}}, \\
& \frac{1}{\tau_{\textrm{l}}}=2 \frac{W}{N} \frac{\left(1-1 / {M_{\textrm{pk}}}^2\right)^2}{M_{\textrm{kp}}}.
\end{aligned} 
\label{eq:cooling_rate}
\end{equation}

\noindent Коэффициенты смешивания определяются как

\begin{equation}
\begin{aligned}
M_{\textrm{pk}} & =\frac{1}{2\left(f_{\max }+f_{\min }\right) \eta_{\textrm{pk}} T_{\textrm{pk}} \frac{\Delta p}{p}}, \\
M_{\textrm{kp}} & =\frac{1}{2\left(f_{\max }-f_{\min }\right) \eta_{\textrm{kp}} T_{\textrm{kp}} \frac{\Delta p}{p}},
\end{aligned}
\label{eq:mixing_coeff}
\end{equation}

\noindent где $\eta_{\textrm{pk}}T_{\textrm{pk}}\frac{\Delta p}{p}$, $\eta_{\textrm{kp}}T_{\textrm{kp}}\frac{\Delta p}{p}$ -- относительные времена смещения частиц (перемешивание),  $\eta_{\textrm{pk}}$, $\eta_{\textrm{kp}}$ -- коэффициенты проскальзывания, в первом приближении $\eta_{\textrm{pk}}=\alpha_{\textrm{pk}}-\sfrac{1}{\gamma^2}$, $\eta_{\textrm{kp}}=\alpha_{\textrm{kp}}-\sfrac{1}{\gamma^2}$, $\alpha_{\textrm{pk}}$, $\alpha_{\textrm{kp}}$ -- локальные факторы расширения орбиты первого порядка, $T_{\textrm{pk}}$, $T_{\textrm{kp}}$ -- абсолютные времена пролета между пикапом-киккером и киккером-пикапом соответственно.

\par Времена стохастического охлаждения ур. \ref{eq:cooling_rate} зависят от соотношения эффективной плотности частиц $N$ к полосе пропускания системы охлаждения $W$ и свойств магнитооптики, а именно локальных факторов расширения орбиты $\alpha_{\textrm{pk}}$, $\alpha_{\textrm{kp}}$.  
\noindent Максимальное значение полосы частот $f_{\textrm{max}}$ ограничено критерием неперекрытия Schottky-полос пучка. В простейшем случае это условие может быть записано как

\begin{equation}
f_{\textrm{max}}<\frac{1}{\eta_{\textrm{pk}}T_{\textrm{pk}}\frac{\Delta p}{p}},
\end{equation}	

\noindent при выполнении которого всегда фактор смешивания $M_{\textrm{pk}}>1$. В обратном случае, эффективность охлаждения становится нулевой. Таким образом, при заданном числе частиц желательно иметь полосу частот максимально возможной. С точки зрения электроники современные технологии позволяют реализовать полосу частот $10$ ГГц \cite{caspers:stochastic}, однако использование ее не всегда возможно из-за большой величины коэффициента проскальзывания $\eta_{\textrm{pk}}$ и разброса по импульсам $\frac{\Delta p}{p}$.

\noindent Уравнение \ref{eq:stochastic_rate} выведено для непрерывного (несгруппированного) пучка.  Эффективное число частиц, для случая сгустка, сформированного гармоническим одночастотным ВЧ резонатором, плотность частиц описывается распределением по Гауссу

\begin{equation}
\rho(s)=\frac{N_{\textrm{bunch}}}{\sigma_{\textrm{bunch}}\sqrt{2\pi}}\cdot e^{-\frac{s^2}{2\sigma_{\textrm{bunch}}^2}},\ \ \ 
\end{equation}	

\noindent где $s$ – расстояние от центра сгустка, $\sigma_{\textrm{bunch}}$ – дисперсия распределения частиц и $N_{\textrm{bunch}}$ – число частиц в сгустке. Если принять, что охлаждение определяется его минимальным значением в центре сгустка ($s=0$), то эффективное значение частиц на орбите длиной $C_{\textrm{orb}}$ равно

\begin{equation}
N=\int_{0}^{C_{\textrm{orb}}}{\rho_{\textrm{max}}ds}=\frac{N_{\textrm{bunch}}}{\sqrt{2\pi}\sigma_{\textrm{bunch}}}\cdot C_{\textrm{orb}}.
\end{equation}

\noindent Для сгустка, сформированного мульти-гармонической ВЧ системой барьерного типа (Barrier Bucket), распределение частиц в сгустке близко к однородному с длиной сгустка $l_{\textrm{bunch}}=4\cdot\sigma_{\textrm{bunch}}$. Эффективное значение частиц определяется простым соотношением длины сгустка к общей длине орбиты

\begin{equation}
N=\frac{N_{\textrm{bunch}}}{{4\sigma}_{\textrm{bunch}}}\cdot C_{\textrm{orb}}.
\end{equation}

\noindent Подводя итог, можно сказать, что эффективное значение частиц зависит от распределения и определяется форм-фактором $F_{\textrm{bunch}}$, лежащим в пределах $F_{\textrm{bunch}}=\sqrt{2\pi}\divisionsymbol4$, тогда

\begin{equation}
N=N_{\textrm{bunch}}\cdot\frac{C_{\textrm{orb}}}{F_{\textrm{bunch}}\cdot\sigma_{\textrm{bunch}}}.
\end{equation}

\noindent Для NICA  примем максимальный фактор $F_{\textrm{bunch}}=4$, и при ее ориентировочных параметрах $C_{\textrm{orb}}=503.04$ м, $\sigma_{\textrm{bunch}}=0.6$ м, $N_{\textrm{bunch}}=2.2\cdot{10}^9$. С учетом опыта работы FNAL \cite{church:stochastic} вполне реалистичные значения для полосы частот являются $f_{\textrm{max}}=8$ ГГц и $f_{\textrm{min}}=2$ ГГц. Для NICA выбрано $f_{\textrm{max}}=4$ ГГц и $f_{\textrm{min}}=2$ ГГц. При таких параметрах максимальная достижимая скорость охлаждения $\frac{1}{\tau_{\textrm{tr}}}=\frac{1}{230}$ $\text{c}^{-1}$.

Исходя из уравнений \ref{eq:mixing_coeff}, видно, что может происходить асимптотический рост в двух случаях:
\begin{enumerate}
\item при приближении коэффициента проскальзывания к значению $\eta~\rightarrow~\frac{1}{2\left(f_{\textrm{max}}+f_{\textrm{min}}\right)T_{\textrm{pk}}\frac{\Delta p}{p}}$, Schottky-спектр пучка становится сплошным и $M_{\textrm{pk}}~\rightarrow~1$;
\item при приближении коэффициента проскальзывания к нулю, перемешивание на пути от киккера к пикапу не происходит и $M_{\textrm{kp}}\rightarrow\infty$.
\end{enumerate}

\noindent Эффективность стохастического охлаждения зависит от свойств магнитооптики. В классических, регулярных, структурах критическая энергия пропорциональная горизонтальной частоте бетатронных колебаний $\gamma_{\text{tr}}\approx\nu_x$ и коэффициент проскальзывания $\eta=1/\gamma_{\textrm{tr}}^2-1/\gamma^2$ может достигать нуля. Чтобы избежать асимптотического роста, необходимо изменять коэффициент проскальзывания, что означает вариацию $\gamma_{\textrm{tr}}$. Это возможно в 'резонансной' структуре, где критическая энергия может быть увеличена или даже достигать комплексного значения \cite{senichev:resonant}. В более экзотическом случае может быть использована 'комбинированная' структура, где $\eta_{\textrm{pk}}$ (пикап-киккер) с реальной критической энергией на одной арке

\begin{equation} \label{eq:eta_pk}
\eta_{\textrm{pk}}=\frac{1}{\gamma_{\textrm{tr}}^2}-\frac{1}{\gamma^2},
\end{equation}

\noindent компенсируется $\eta_{\textrm{kp}}$ (киккер-пикап) с комплексным значением в другой арке соответственно

\begin{equation} \label{eq:eta_kp}
\eta_{\textrm{kp}}=-\frac{1}{\gamma_{\textrm{tr}}^2}-\frac{1}{\gamma^2},\ \ \ 
\end{equation}

\noindent для всего кольца. При такой конструкции достигается требуемое соотношение факторов смешивания для максимальной скорости охлаждения, близкой к идеальной \cite{senichev:hesr}. Рассмотрим заявленные структуры более подробно.

\par Поведение $\beta$-функций и $D$ дисперсии вдоль всей регулярной структуры с $\gamma_{\text{tr}}=7$ показаны на рисунке \ref{fig:1_regular}. Прямые участки, которые остаются неизменными во всех структурах, необходимы для анализа резонансных характеристик всего кольца в целом. Их расположение не влияет на внутрипучковое рассеяние и критическую энергию. Для подавления дисперсии в регулярной структуре с обеих сторон арки реализуется технология ‘отсутствующих магнитов’ ('missing magnet').

\begin{figure}[!h]
  \centering
   \includegraphics*[width=1.0\columnwidth]{1_regular}
   \caption{Регулярная ФОДО структура коллайдера NICA.}
   \label{fig:1_regular}
\end{figure}

\noindent Резонансная структура основана на принципе резонансной модуляции дисперсионной функции \cite{senichev:construction} и может быть получена из регулярной структуры путем разделения фокусирующих квадруполей на 2 семейства с различными градиентами, что будет подробно рассмотрено в Главе \ref{ch:resonant}. Таким образом, критическая энергия может быть скорректирована и увеличена по сравнению с энергией эксперимента, избегая проблем с пересечением критической энергии. Для подавления дисперсии можно использовать либо два краевых фокусирующих квадруполя с обоих сторон поворотной арки, либо только два семейства фокусирующих квадруполей на арке \cite{Kolokolchikov:2021trans}, когда достигается целое число бетатронных колебаний (рис. \ref{fig:1_resonant}).

\begin{figure}[!h]
  \centering
   \includegraphics*[width=1.0\columnwidth]{1_resonant}
   \caption{Резонансная магнитооптическая структура коллайдера NICA с повышенной критической энергией.}
   \label{fig:1_resonant}
\end{figure}

\noindent В случае комбинированной структуры, одна арка функционирует в регулярном режиме, в то время как другая использует резонансную модуляцию (рис. \ref{fig:1_combined}). Такой выбор основан на принципе компенсации, описанном уравнениями \ref{eq:eta_pk} и \ref{eq:eta_kp}, который требует большей глубины модуляции квадруполей, чем в чисто резонансной структуре с повышенной критической энергией.

\begin{figure}[!h]
  \centering
   \includegraphics*[width=1.0\columnwidth]{1_combined}
   \caption{Резонансная магнитооптическая структура коллайдера NICA с реальной и комплексной критической энергией в арках.}
   \label{fig:1_combined}
\end{figure}

\noindent Как показано на рисунке \ref{fig:1_SC} для резонансной магнитооптике с увеличенной критической энергией, вторая асимптотика соответствует более высокой энергию по сравнению с регулярной структурой. В комбинированной магнитооптике эффективность охлаждения близка к идеальному значению в широком диапазоне энергий от $2.5$ до $4.5$ ГэВ/нуклон, в то время как в регулярной оптике скорость охлаждения почти в два раза ниже в наиболее оптимальной точке~$\sim3$~ГэВ/нуклон. Такое поведение объясняется отсутствием второй точки асимптотического роста.

\begin{figure}[!h]
  \centering
   \includegraphics*[width=0.495\columnwidth]{1_SC}
   \includegraphics*[width=0.495\columnwidth]{1_SC_wide}
   \caption{Зависимость времени стохастического охлаждения от энергии для разных структур.}
   \label{fig:1_SC}
\end{figure}

\newpage
\subsection{Внутрипучковое рассеяние в регулярной, резонансной и комбинированной структурах}\label{sec:ions_light/IBS_res_reg}

\par Как было уже сказано, внутрипучковое рассеяние является основным фактором, ограничивающим время жизни пучка в коллайдере. Поэтому критерием для использования того или иного способа охлаждения является сравнение их характерных времен с временем разогрева пучка из-за внутрипучкового рассеяния. Из общей теории этого явления следует:

\begin{equation}
\frac{1}{\tau_{\textrm{IBS}}}=\frac{\sqrt\pi}{4}\frac{cZ^2r_p^2L_C}{A}\cdot\frac{N}{C_{\mathrm{orb\ }}}\cdot\frac{\left\langle\beta_x\right\rangle}{\beta^3\gamma^3\varepsilon_x^{5/2}\left\langle\sqrt{\beta_x}\right\rangle}\left(\left\langle\frac{D_x^2+{\dot{D}}_x^2}{\beta_x^2}\right\rangle-\frac{1}{\gamma^2}\right).
\label{eq:IBS}
\end{equation}

\noindent В отличие от стохастического охлаждения скорость разогрева из-за внутрипучкового рассеяния растет с уменьшением энергии как $1/\gamma^3$. Кроме того, выражение, стоящее в круглых скобках, пропорционально коэффициенту проскальзывания $\eta$. Поэтому следует ожидать, что в оптике со значением $\eta$ близким к нулю скорость разогрева должна падать. На рисунке \ref{fig:1_IBS} показаны зависимости постоянной времени нагрева в трех вышеупомянутых структурах, посчитанных с помощью программы MADX \cite{madx, antoniou:ibs} для параметров тяжелоионного пучка ${_{79}^{197}}\textrm{Au}$ коллайдера NICA c максимальной светимостью ${10}^{27}$ $\text{см}^{-2}\cdot\text{с}^{-1}$, интенсивностью $N_{\text{heavy}} = 2.2\times10^9$ ppb (particles per bunch -- количество частиц в пучке) и количеством пучков $n_{\text{bunch}}=22$. В случае легких ядер, таких как протоны и дейтроны, время ВПР значительно увеличивается по мере уменьшения заряда даже для интенсивного сгустка ${10}^{31}$ $\text{см}^{-2}\cdot\text{с}^{-1}$, $N_{\text{light}} = 1\times10^{12}$~ppb. В таблице \ref{tab:dual} приведены расчёты, которые показываю, что при энергии эксперимента времена ВПР различаются примерно в 10 раз, поэтому проблема внутрипучкового рассеяния имеет значение для тяжелоионного сгустка с высокой зарядностью.

\begin{figure}[!h]
  \centering
   \includegraphics*[width=0.75\columnwidth]{1_IBS}
   \caption{Зависимость постоянной времени разогрева пучка из-за внутрипучкового рассеяния в регулярной, резонансной и комбинированной структурах от энергии пучка.}
   \label{fig:1_IBS}
\end{figure}

\noindent Из сравнения времени внутрипучкового рассеяния со временем охлаждения можно сделать заключение, что в регулярной структуре стохастическое охлаждение способно сбалансировать внутрипучковое рассеяние в диапазоне энергий $W\geq4.5$ ГэВ/нуклон. Для применения стохастического охлаждения во всем диапазоне энергий требуется пожертвовать светимостью пучка на низких энергиях посредством увеличения эмиттанса. В резонансных структурах, время ВПР значительно меньше. Это объясняется тем, что структура имеет большее соотношение между дисперсией и $\beta$-функцией пучка $\left\langle\frac{D_x^2+{\dot{D}}_x^2}{\beta_x^2}\right\rangle$, чем в случае регулярной. Таким образом, для тяжелоионной опции должна быть использована структура с максимально регулярным $\beta$-функцией и $D$ дисперсией -- то есть минимально модулированы. Для охлаждения пучка до $4.5$ ГэВ/нуклон в регулярной структуре также используется электронное охлаждение \cite{kostromin:stochastic}.

\begin{table}[h!]
	\centering
	\begin{tabular}{|l|c|c|c|}
		\hline
		\textbf{Структура} & \textbf{Регулярная} & \textbf{Резонансная} & \textbf{Комбинир.} \\
		\hline
		Энергия, ГэВ/нуклон & 4.5 & 12.6 & 12.6 \\
		\hline
		Критическая энергия $\gamma_{\text{tr}}$ & 7 & 15 & 150 \\
		\hline
		Глубина модуляции & -- & 25\% & 45\% \\
		\hline
		Время охл. при 4.5 ГэВ/н, с & 2500 & 1500 & 800 \\
		\hline
		\parbox{7cm}{Время ВПР, с (для тяжелых ионов при 4.5 ГэВ/нуклон)} & 2500 & 400 & 250 \\
		\hline
		\parbox{7cm}{Время ВПР, с (для протонов при 12.6 ГэВ/нуклон)} & $1.8 \times 10^4$ & $4.5 \times 10^3$ & $7.9 \times 10^3$ \\
		\hline
		Рабочая точка & 9.44/9.44 & 9.44/9.44 & 9.44/9.44 \\
		\hline
	\end{tabular}
	\caption{Основные параметры магнитооптических структур коллайдера NICA.}
	\label{tab:dual}
\end{table}

\section*{Выводы}
\par Рассмотрены основные принципы реализации дуальной магнитооптическая структуры. 
В таблице \ref{tab:dual} приведены основные параметры рассмотренных структур.

\begin{enumerate}

\item  Для легких частиц из-за соотношения заряда к массе энергия эксперимента может превышать критическую энергию установки, которая является оптимальной для тяжелых ионов. При использовании дисперсионной модуляции критическая энергия увеличивается или даже достигает комплексного значения в резонансной или комбинированной магнитооптической структуре;

\item Показано, что вследствие модуляции $\beta$-функции и $D(s)$ дисперсии уменьшается время внутрипучкового рассеяния, что имеет решающее значение для многозарядных тяжелых частиц. По этой причине регулярная магнитооптическая структура с минимально модулированной $D(s)$ дисперсией и $\beta$-функцией оптимальна в режиме работы с тяжелыми ионами. Несмотря на то, что стохастическое охлаждение в регулярной структуре значительно слабее, чем в резонансной и комбинированной, оно может компенсировать эффект от внутрипучкового рассеяния;

\item Дуальная магнитооптическая структура предлагается для ускорения пучков как тяжелых ионов, так и легких частиц, что показано на примере установки NICA. Для преобразования регулярной структуры, оптимизированной для ускорения тяжелых ионов с точки зрения времени жизни пучка, в резонансную, с варьируемой критической энергией для легких частиц, не требуется никаких особых изменений, достаточно лишь внести отдельное семейство квадруполей.

\end{enumerate}

\FloatBarrier
