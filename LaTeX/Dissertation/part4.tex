%part 4

	\chapter{Изучение особенностей магнитооптической структуры синхротронов NICA и Nuclotron с учетом ускорения поляризованных пучков и модернизация магнитооптической структуры Nuclotron с учётом возможности изучения ЭДМ}\label{ch:EDM}

\par Помимо самого коллайдера NICA, в ускорительный комплекс также входит Nuclotron. Данный синхротрон предназначен как для самостоятельных экспериментов на выведенной мишени BM@N, так и для использования в качестве инжектора поляризованного пучка протонов и дейтронов в коллайдер NICA. Однако, установка была введена в эксплуатацию в 70-е годы и может быть модернизирована с использованием новых современных магнитооптических элементов, производимых непосредственно в ОИЯИ г. Дубна.

\par Для расширения возможностей Nuclotron в качестве самостоятельной машины рассматривается возможность изучения ЭДМ заряженных частиц. Изучение ЭДМ заряженных частиц возможно на кольцевом ускорителе, работающем в режиме накопителя. Моделирование структуры представляет отдельную задачу исследования спиновой динамики. Спин – квантовая величина, однако её поведение в ансамбле описывается классически, что было показано Tompson, Telegdi, Bargmann, Michel []. Поведение спина частицы описывается одноименным уравнением T-БМТ. 

\begin{align} \label{eq:T-BMT}
\frac{{d \vec{S}}}{d t} &=\vec{S} \times\left(\vec{\Omega}_{MDM}+\vec{\Omega}_{EDM}\right), \nonumber\\
\vec{\Omega}_{MDM}&=\frac{q}{m \gamma}\left\{(\gamma G+1)\vec{B}_{\perp}+(G+1)\vec{B}_{\parallel}-\left(\gamma G+\frac{\gamma}{\gamma+1}\right) \frac{\vec{\beta} \times \vec{E}}{c}\right\}, \\
\vec{\Omega}_{EDM}&=\frac{q \eta}{2 m}\left(\vec{\beta} \times \vec{B}+\frac{\vec{E}}{c}\right), \quad G=\frac{g-2}{2},\nonumber
\end{align}

\noindent где $\vec{\Omega}_{MDM}, \vec{\Omega}_{EDM}$ -- угловые частоты обусловленные наличием магнитного дипольного момента (МДМ) \& элекстического дипольного момента (ЭДМ); $q, m, G$ -- заряд, масса и магнитная аномалия; $\beta$ -- нормализованная скорость; $\gamma$ -- Лоренц-фактор; $d =~\eta \frac{q}{2mc}s$, $d$ -- ЭДМ фактор, $s$ -- спин. Уравнение содержит 2 слагаемых, одно обусловлено наличием магнитного дипольного момента (МДМ) частицы, другое – электрического дипольного момента соответственно. Для непосредственного измерения ЭДМ-компоненты, влияние МДМ на спин должно быть нивелировано.

Таким образом, при модернизации кольца Nuclotron должны быть учтены несколько факторов:
1. Использование Nuclotron в качестве бустера в NICA
2. Возможность изучения ЭДМ заряженных частиц

	\section{Прохождение критической энергии ускорителя NICA в синхротроне Nuclotron}\label{sec:EDM/nuclotron}

\par Как было показано ранее в Главах , прохождение критической энергии для протонного поляризованного пучка протонов является необходимым требованием коллайдерного эксперимента на SPD детекторе коллайдера NICA для достижения требуемой светимости. Помимо, повышения критической энергии резонансным методом, также возможен подход, при котором, ускорение выше критической энергии происходит в кольце Nuclotron с последующей инжекцией и накоплением уже в кольце коллайдера. Однако, такой способ, требует эффективного охлаждения выше критической энергии. Критическая энергия коллайдера NICA составляет $5.7$ ГэВ, а энергия эксперимента порядка $13$ ГэВ. Таким образом, эффективная энергия для охлаждения может составлять порядка $7-10$ ГэВ. При такой энергии движение происходит строго выше критической энергии. Текущий электронный охладитель рассчитан на энергию $2-3$ ГэВ [?]. Для реализации такого подхода требуется как разработка нового электронного охладителя, так и в случае проектирования нового модернизированного Nuclotron, он должен быть рассчитан на максимальную энергию $7-10$ ГэВ.

\par В Главе 2 Был рассмотрен вариант модернизации критической энергии с повышенной критической энергией. При таком подходе проблем не возникает, поскольку Модернизация NICA для резонансного метода

	\section{Требвания к магнитооптической структуре синхротронов NICA-Nuclotron в задаче исследования электрического дипольного момента легких ядер}\label{sec:EDM/requirements}

\par Текущая структура Nuclotron не предполагает проведения экспериментов по исследованию ЭДМ. Рассмотрим возможные способы реализации такой программы.

Рассмотрим необходимые требования с точке зрения спиновой динамики
Стоит отметить, что поскольку Nuclotron должен быть использован в качестве бустера в NICA, то необходимо реализовать структуру, работающую как для ускорения поляризованного протонного пучка до энергии порядка единицы ГэВ, так и низкоэнергетическую при энергии порядка сотен МэВ с возможностью изучения ЭДМ. Таким образом, ведущим полем должно выступать магнитное поле в поворотных магнитах, поскольку электрическое поле неспособно ускорить до энергии порядка единиц ГэВ. 

Наличие чисто магнитных арок приводит к необходимости использовать метод квазизаможенного спина для компенсации МДМ вращения спина. Такой метод, в отличие от замороженного спина [BNL] предполагает пространственное разделение электрического и магнитных полей и последовательную компенсацию МДМ-компоненты. Компенсация может быть осуществлена на прямых участках с необходимостью использовать электрическое поле. Могут быть рассмотрены как чисто электрические дефлекторы, так и фильтры Вина с перпендикулярными электрическим и магнитным полем.

	\subsection{Электостатический дефлектор}\label{sec:EDM/requirements/deflector}
	
	
\par Для орбитального вращения в электрическом поле угловая скорость имеет вид
\begin{equation}
\vec{\Omega}_p^E=\frac{q}{m c} \frac{\vec{\beta} \times \vec{E}_{\perp}}{\gamma(\vec{\beta}, \vec{\beta})}=\frac{q}{m \gamma} \frac{\vec{\beta} \times \vec{E}_{\perp}}{c \beta^2}
\end{equation}

Рассмотрим простейший случай одного периода. На рис. изображено поведения спин-вектора для дейтрона при последовательном действии сначала магнитной арки, киккера, электростатической арки с отрицательной кривизной и киккера. В отсутствии электростатических дефлекторов, вращение импульса в магнитной арке происходит на $\Phi_p^{arc}=\sfrac{2\pi}{N}$. При необходимости введения электростатической арки с отрицательной кривизной $\Phi_{pE}^{def}$, магнитные арки должны дополнительно поворачивать на угол $\Phi_{pB}^{kick}$, при помощи киккеров $\Phi_{pB}^{kick}$, который будет в последствии скомпенсирован поворотом в электростатической арке. Окончательно, импульс, после прохождения периода, будет повернут на
\begin{equation}
\Phi_p^{arc}+\Phi_{pB}^{kick}+\Phi_{pE}^{def}=\frac{2\pi}{N}\ \ \
\end{equation}
или с учетом поворота в магнитной арке
\begin{equation}
\Phi_{pB}^{kick}={-\Phi}_{pE}^{def}\ \ \ 
\end{equation}

Спин-вектор в магнитной арке совершит отклонение в магнитном поле $\Phi_{sB}^{arc+kick}=\nu_s^B\left(\Phi_p^{arc}+\Phi_{pB}^{kick}\right)$. В электростатической арке $\Phi_{sE}^{\ def}=\nu_s^E\Phi_{pE}^{def}$. 
Отдельно для спинового движения, в киккерах и дефлекторе
\begin{equation}
\begin{aligned}
\Phi_s^{k i c k+d e f}  & = \nu_s^B \Phi_{p B}^{k i c k}+\nu_s^E \Phi_{p E}^{\text {def }}= \\
				& = \Phi_{p B}^{k i c k}\left[\gamma G-\beta^2 \gamma\left(G-\frac{1}{\gamma^2-1}\right)\right]= \\
				& = \Phi_{p B}^{k i c k}\left[\frac{G+1}{\gamma}\right]
\end{aligned}
\end{equation}

Условие сохранения ориентации спин-вектора, то есть условие «квази-замороженности» можно записать в виде
\begin{equation}
\Phi_s^{arc}+\Phi_s^{kick+def}=0\ \ \
\end{equation}
Тогда из (6)-(7)
\begin{equation}
\Phi_{pB}^{kick}=-\Phi_p^{arc}\frac{\gamma^2G}{G+1}=-\frac{2\pi}{N}\frac{\gamma^2G}{G+1}\ \
\end{equation}

	
	\subsection{Прямой фильтр Вина}\label{sec:EDM/requirements/wien}
\par Рассмотрим движение в прямом фильтре Вина. Ключевое условие – равенство нулю силе Лоренца, таким образом фильтр Вина не отклоняет орбитальное движение. 
\begin{equation}
q\left(c\vec{\beta}\times{\vec{B}}_\bot+{\vec{E}}_\bot\right)=0\ \ \	
\end{equation}
Стоит отметить, при этом выполняется равенство магнитного и электростатического абсолютных радиусов $\left|R_E\right|=\left|R_B\right|$ и для углов поворота
\begin{equation}
\Phi_{pB}^{WF}+\Phi_{pE}^{WF}=0\ \ \ 
\Phi_{pB}^{WF}=-\Phi_{pE}^{WF}=\Phi_p^{WF}\ \ \
\end{equation}

Поскольку импульс в фильтре Вина остается неизменным, то результирующее вращение спина может быть удобно рассмотрено как в абсолютной системе, так и относительной. При этом необходимо подавить вращение от магнитной арки $\Phi_s^{arc}=\gamma G\bullet\Phi_p^{arc}$.
Таким образом условие «квази-замороженности» может быть записано аналогично ур. (7) как 
\begin{equation}
\Phi_s^{WF}=-\Phi_s^{arc}\ \ \
\end{equation}

Для сравнения, сделаем это в обоих.
Для спинового движения в фильтре Вина под действием МДМ выполняется
\begin{equation}
{{\vec{\Omega}}_{MDM}^{B_\bot}=-\frac{q}{m\gamma}\left(\gamma G+1\right){\vec{B}}_\bot\atop{\vec{\Omega}}_{MDM}^E=\frac{q}{m\gamma}\left(\gamma G+\frac{\gamma}{\gamma+1}\right)\frac{\vec{\beta}\times\vec{E}}{c}}\ \ \ 
\end{equation}

\begin{equation}
\begin{aligned}
\Phi_s^{W F} & =(\gamma G+1) \Phi_{p B}^{W F}+\beta^2\left(\gamma G+\frac{\gamma}{\gamma+1}\right) \Phi_{p E}^{W F}= \\
& =\Phi_p^{W F}\left[(\gamma G+1)-\beta^2\left(\gamma G+\frac{\gamma}{\gamma+1}\right)\right]= \\
& =\Phi_p^{W F}\left[\frac{G+1}{\gamma}\right]
\end{aligned}
\end{equation}

Также спиновое движение может быть рассмотрено и относительно. Тогда

\begin{equation}
\begin{aligned}
 \Phi_s^{W F} & =  v_s^B \Phi_{p B}^{W F}+v_s^E \Phi_{p E}^{W F}= \\
			& =  \Phi_p^{W F}\left[\gamma G-\beta^2 \gamma\left(G-\frac{1}{\gamma^2-1}\right)\right]= \\
			& =  \Phi_p^{W F}\left[\frac{G+1}{\gamma}\right]
\end{aligned}
\end{equation}

Выражения (6) и (15) дают идентичный поворот. Более того, совпадают в уравнением 
А значит и для угла поворота в фильтре Вина.
\begin{equation}
\Phi_p^{WF}=-\Phi_p^{arc}\frac{\gamma^2G}{G+1}=-\frac{2\pi}{N}\frac{\gamma^2G}{G+1}
\end{equation}

	\subsection{Длина элемента}\label{sec:EDM/requirements/length}
\par Радиус кривизны элемента с электрическим и магнитным полем может быть найден как
\begin{equation}
\begin{gathered}
\frac{1}{R}  = \frac{1}{R_B}+\frac{1}{R_E}\ \ 
	R_B  = \frac{B\rho}{B}\ \ 
	R_E  = \frac{\kappa}{E}\ \ 
\end{gathered}
\end{equation}
Где $B\rho=\sfrac{p_0}{e}$ – магнитная жесткость,$ p_0=\gamma m\beta c$, $\kappa=\sfrac{p_0\beta c}{e}$ – электрическая жесткость.
Поскольку для фильтра Вина $R=0$, то и радиусы кривизны связаны $R_B=-R_E$. И для выбора радиуса достаточно определить либо магнитное поле, либо электрическое. Более строгое ограничение дается на электрическое поле $E_{max}=10\div13$ МВ/м.
Для нахождения длины
\begin{equation}
L={\Phi_p^{WF}R}_E=-\Phi_p^{arc}\frac{\gamma^2G}{G+1}R_E\ \ \
\end{equation}
Для минимальной длины в периодической структуре
\begin{equation}
L_{min}=-\frac{2\pi}{N}\frac{\gamma^2G}{G+1}\frac{\kappa}{E_{max}}\ \ \
\end{equation}

	\subsection{Сорт частиц}\label{sec:EDM/requirements/particles}
	
\par В зависимости от сорта исследуемых частиц аномальный магнитный момент для протона $G=1.79$ и для дейтрона $G=-0.142$. Отличаются как абсолютное значение, так и знак. Если рассмотреть вывод формул, то всюду учитывалось, что введенные углы могут иметь как положительный, так и отрицательный знак. Таким образом уравнения могут быть использованы как для рассмотрения дейтрона, так и протона. 

\par  Как видно из рассмотренных структур, изучение одновременно ЭДМ дейтрона и протона в структуре с электростатическими дефлекторами не целесообразно по сравнению со структурой с использованием фильтров Вина. Во-первых, требуемая длина дефлекторов равна длине фильтров Вина, но в первом случае необходимы дополнительные киккеры. Во-вторых, кривизна дефлекторов для протонов и дейтронов имеет различный знак. В тоже время, фильтры Вина устанавливаются на прямой участок и не требуют альтернативного канала. А для изучения протонов фильтры Вина могут быть повернуты на 180 градусов относительно оси.

	\section{Адаптация существующей структуры синхротрона NICA для реализации режима «квази-замороженного спина»}\label{sec:EDM/QFS}

\par Как было показано в Главе \ref{sec:EDM/requirements}, для исследования ЭДМ в накопительном кольце racetrack-формы может быть применён только метод квази-замороженного спина с использованием фильтра Вина на прямом участке, для структуры с несколькими поворотными арками (8-16) квази-замороженный спин может быть реализован с применением электрических дефлекторов с отрицательной кривизной.

	\section{Метод введения ByPass c расположенными на них Wien filters}\label{sec:EDM/Wien_filter}

\par В экспериментах по измерению электрического дипольного момента (ЭДМ) ключевым является обеспечение высокого показателя времени когерентности (SCT — Spin Coherence Time) порядка 1000 с. В течение такого времени когерентный поляризованный пучок удерживается на орбите. Для этого возможно использовать главное кольцо NICA в качестве накопителя, а не в режиме коллайдера. Поэтому, предлагается установить дополнительные отводные каналы ByPass. Таким образом, можно создать совершенно новую регулярную структуру с альтернативными прямыми секциями, параллельными имеющимся. Создание отводных каналов является большим преимуществом, не требующим значительной перестройки комплекса и затрат, при всём при этом, позволяющим задействовать NICA в различных экспериментах.

\subsection{Предпосылки модернизации главного кольца NICA}\label{sec:EDM/Wien_filter/modernization}

\par Исходя из уравнений эволюции вектора спина, видно, что определяющими факторами являются энергия эксперимента, а также сорт частиц.

\par Эксперимент по исследованию ЭДМ не требует специального детектора, необходимо только наличие поляриметра. Сечение рассеяния на поляриметре с углеродной мишенью принимает наибольшее значение для протонов и дейтронов при энергии $270$ МэВ. Это требование устанавливает энергию эксперимента и определяется потребностями поляриметрии.
Кроме того, требуется обеспечить стабильность спинового движения. Прямым следствием Т-БМТ уравнения является концепция «замороженного» спина [2]. Такой метод предполагает равенство нулю члена, связанного с магнитным дипольным моментом (МДМ) в течение всего времени удержания пучка. Этот способ справедлив как для протонов, так и для дейтронов, но имеет существенные отличия. Аномалия магнитного момента дейтрона отрицательна $G_{d}=-0.1429$, и на порядок меньше по абсолютному значению, чем у протона $G_{p}=1.7928$. Для протонов определенной энергии, называемой «магической», МДМ член принимает нулевой значение в чисто электрическом кольце, без магнитных элементов. В то время как для дейтронов, из-за отрицательного значения магнитной аномалии, необходимая магнитооптическая структура предполагает использование поворотных дефлекторов как с электрическим, так и с магнитным полем. При этом орбитальное вращение спина в магнитном поле компенсируется электрическим в одном элементе. Таким образом, вращение сохраняет свою ориентацию в течение всего времени вращения в кольце. Однако, диполи в арках главного кольца NICA имеют только магнитную составляющую поля. Таким образом, реализация концепции «замороженного спина» в кольце NICA невозможна без соответствующей значительной модернизации и реконструкции.

\begin{figure}[!h]
  \centering
   \includegraphics*[width=1.0\columnwidth]{4_NICA_bypass.png}
   \caption{Принципиальная схема обходных каналов ByPass в существующем комплексе NICA.}
   \label{fig:4_NICA_bypass}
\end{figure}

\par Для проведения эксперимента по поиску ЭДМ становится необходимым использовать альтернативный метод управления спином — концепция «квази-замороженного» спина [3]. В отличие от метода «замороженного» спина, спин больше не сохраняет ориентацию в течение всего периода обращения, а восстанавливает ориентацию на прямолинейном участке. Это возможно благодаря использованию на прямом участке элементов как с электрическим, так и с магнитным полями, которые называются фильтрами Вина. Поворот вращения в арке на определенный угол компенсируется соответствующим поворотом в фильтре Вина. Также можно выбрать поля, создающие нулевую силу Лоренца и не нарушающие орбиту. По этой причине поляриметрия может устанавливаться на прямых участках. Таким образом, поляриметры, расположенные после фильтров Вина, будут обнаруживать ту же ориентацию спин-вектора, и для них она будет «замороженной»

\par Есть две основные причины для модернизации магнитооптической структуры. Во-первых, нехватка места для фильтров Вина в уже существующих прямых секциях. Во-вторых, доступная магнитооптика предполагает использование кольца NICA в режиме коллайдера. Но эксперименты по поиску ЭДМ предполагают длительное удержание и удерживание поляризованного когерентного пучка в течение примерно $T_{SC} \sim 1000 с$. Поэтому была предложена модернизация коллайдера NICA путем введения обходных каналов для создания альтернативного прямого участка, параллельного первоначальному. Таким образом, NICA можно использовать в качестве накопительного кольца (Storage Ring mode). Такие кольца позволяют проводить эксперименты по исследованию ЭДМ с поляризованными дейтронами и поиска аксиона в режиме «квази-замороженного» спина.

\subsection{Магнитооптическая структура с отводными каналами ByPass}\label{sec:EDM/Wien_filter/ByPass}

\par Приведенные в предыдущем разделе особенности являются решающими при выборе энергии эксперимента и сорта частиц. В будущем вся предлагаемая магнитооптика будет рассмотрена для дейтронов с энергией $240 МэВ$. Стоит отметить, что расчеты показывают основные параметры магнитного поля диполей $B_{dip}=0,132\ T$, а также магнитную жесткость $B\rho=3,252\ T\bullet m$. (Таблица 1)
\par Проектируя накопительное кольцо NICA с отводными секциями ByPass, планируется, оставить геометрию арок неизменной. Возможно лишь изменение полей в уже установленных элементах. Так что NICA можно использовать для различных экспериментов.
\par В кольце NICA, арка имеет ненулевую дисперсию. По краям как дисперсия, так и ее производная сведены к нулю. Прямой участок имеет нулевую дисперсию по всему периметру. Общая длина оригинального кольца NICA $L_{acc}=503,04 м$. Длина одной арки равна $L_{arc}=142,15 м$. Итак, доступно $\left(L_{acc}-2L_{arc}\right)/2=109.6 м$. 
\par ByPass – это канал для альтернативного прохождения кольца, не содержащее место встречи. Дипольные магниты выбраны таким образом, чтобы обеспечить отклонение на угол $\alpha=\ 9$. Сила диполя $B_{BP}=1 Т$ при длине $L_{dip}^{BP}=50 см$. Альтернативный прямой участок находится на расстоянии $1$ метра от исходного, поэтому длина обводного участка $L_{BP}=1\mathrm{м/sin\alpha}\sim6.4 м$. Принципиальная схема обходных каналов показана на Рисунке 1.
	Отклоняющие магниты искажают дисперсионную функцию. Таким образом, необходимо было использовать по меньшей мере 2 фокусирующих квадруполя на обходном канале для подавления дисперсия на выходе. Это поможет обеспечить нулевую дисперсию на всем прямолинейному участке. Чтобы обеспечить периодичность и симметрию бета-функций, можно использовать или один или три дефокусирующих квадруполя.
	Будут рассмотрены два случая, с адаптированными прямыми участками, идентичным поворотным аркам, но без магнитов. Это сделано для простоты моделирования в регулярной идеальной структуре. Наконец, мы рассмотрим реальный случай магнитооптики с полностью регулярной ФОДО прямой секцией.
	
\subsubsection{Первичная схема с 3 квадруполями}\label{sec:EDM/Wien_filter/ByPass/3quad}

\par В этом случае байпас состоит из минимально возможных 3 квадруполей: 2 фокусирующих QBP1 и 1 дефокусирующий QBP2 (Рисунок 2). Согласование арки с каналом ByPass обеспечивается тремя квадруполями QM1, QM2, QM3 (секция согласователя Matching M1). А согласование ByPass с прямым участком также симметрично осуществляется такими же квадруполями QM1, QM2, QM3. Это возможно в силу изначально заложенной симметрии между аркой и прямым участком. Длины упомянутых секций приведены на Рисунке 2, а также в таблице 1. Тогда общая длина всего ускорителя составит $L{_3quad^{acc}}=503,46 м$.
\par На Рисунке 3 приведены Твисс-функции, черными линиями указаны границами канала ByPass. Максимум бета-функции $\beta_y$ расположен в центре канала ByPass. И может принимать большое значение, по сравнению с $\beta_{\ x}$. По этой причине можно рассмотреть случай с 5 квадруполями в отводном канале.

\begin{figure}[!h]
  \centering
   \includegraphics*[width=1.0\columnwidth]{4_bypass_3scheme.png}
   \caption{Принципиальная схема ByPass с 3 квадруполями.}
   \label{fig:4_bypass_3scheme}
\end{figure}

\begin{figure}[!h]
  \centering
   \includegraphics*[width=1.0\columnwidth]{4_bypass_3quad}
   \caption{Twiss-параметры для ByPass с 3 квадруполями. Черными линиями показано расположение дефлекторов.}
   \label{fig:4_bypass_3quad}
\end{figure}

\subsubsection{Схема ByPass с 5 квадруполями}\label{sec:EDM/Wien_filter/ByPass/5quad}

По сравнению с предыдущим случаем, обводной канал состоит из 5 квадруполей, которые представлены 2 семействами: фокусирующим QBP1 и дефокусирующим QBP2. Он становится длиннее $L_{5quad}^{BP}=9,35 м$ и отклоняется на $1.46 м$ (Рисунок. 4). Теперь секции согласования M1 и M2 по-прежнему идентичны, но представлены двумя квадруполями QM1 и QM2 для обеспечения регулярности Твисс-функций. Однако, полная длина ускорителя становится больше, NICA $L_{5quad}^{acc}=510,02 м$. На Рисунке 5 показаны, что максимум $\beta_y$ становится меньше в центре. Стоит отметить, что максимум дисперсионной функции стал увеличился от $D_x^{3\ quad}\ ~\ 0.2\ м$ до $D_x^{5\ quad}\ ~\ 0.5\ м$. Таким образом, этот случай должен быть адаптирован к реальному.

\begin{figure}[!h]
  \centering
   \includegraphics*[width=1.0\columnwidth]{4_bypass_5scheme.png}
   \caption{Принципиальная схема ByPass с 5 квадруполями.}
   \label{fig:4_bypass_5scheme}
\end{figure}

\begin{figure}[!h]
  \centering
   \includegraphics*[width=1.0\columnwidth]{4_bypass_5quad}
   \caption{Twiss-параметры для ByPass с 5 квадруполями. Черными линиями показано расположение дефлекторов.}
   \label{fig:4_bypass_5quad}
\end{figure}

\subsubsection{Окончательный вариант}\label{sec:EDM/Wien_filter/ByPass/final}

Основываясь на рассмотренных примерах, мы, наконец, можем получить структуру, максимально адаптированную к реальности. Теперь рассмотрим полностью регулярный прямой участок, который стал короче $L_{SS}^{BP}=80,71 м$ (Рисунок 6). Байпас состоит из 5 квадруполей и отклоняет пучок на $1.46 м$. Но для согласования использовались разные секции M1 и M2, чтобы компенсировать не симметрию между поворотной аркой и прямым участком. Наконец, Твисс-функция половины байпасного NICA, представлена на Рисунке 7. В центре прямой секции расположены фильтры Вина. Все расчеты выполнены при помощи программ OptiM [4] и COSY Infinity [5].

\par Для экспериментов с EDM необходимо использовать NICA в качестве накопительного кольца. По этой причине была рассмотрена модернизация путем создания альтернативных прямых участков, параллельных исходным, с использованием каналов ByPass. Также на прямых участках есть возможность разместить специальные элементы – фильтры Wien для компенсации вращения спина от МДМ компоненты в поворотных арках. Поскольку арки остаются неизменными, это позволяет использовать NICA в различных экспериментах.
\par Рассмотрены 2 принципиальные схемы обходного канала. И, наконец, получили наиболее реалистичный случай, когда прямой участок полностью регулярный. Конечная конструкция удовлетворяет всем необходимым требованиям к магнитооптике. Исследование спин-орбитальной динамики с оптимизированными фильтрами Вина показывают, спин восстанавливает ориентацию на прямом участке и метод «Квази-Замороженного Спина» может быть реализован в ByPass NICA.

\begin{figure}[!h]
  \centering
   \includegraphics*[width=1.0\columnwidth]{4_bypass_real_scheme.png}
   \caption{Принципиальная схема адаптированной структуры кольца NICA c ByPass.}
   \label{fig:4_bypass_real_scheme}
\end{figure}

\begin{figure}[!h]
  \centering
   \includegraphics*[width=1.0\columnwidth]{4_bypass_twiss_ring.png}
   \caption{Twiss-функции для половины адаптированной структуры кольца NICA c ByPass. Фильтры Вина, расположенные на прямом участке.}
   \label{fig:4_bypass_twiss_ring}
\end{figure}

\newpage
\section{Спин-орбитальная динамика пучка в Wien filters, спин-орбитальный трекинг в магнитном кольце со скрещенными E+B элементами}\label{sec:EDM/Wien_filter_tracking}

\par Возможность управления спином для эксперимента с электрическим дипольным моментом (ЭДМ) может быть реализована путем установки фильтров Вина в прямолинейных секциях, которые гарантируют, что вращение вектора спина частиц сохраняет направление в соответствии с режимом «Квази-Замороженного Спина». Однако спин разных частиц, из-за их движения в трехмерном пространстве, в любом случае, прецессирует со слегка отличающимися частотами вокруг инвариантной оси. Таким образом, нарушает спиновую когерентность. Для обеспечения спиновой когерентности необходимо использовать нелинейные элементы, секступоли, расположенные в местах с ненулевой дисперсией, на поворотных арках. Так как секступоли также влияют и на бетатронную хроматичность, мы рассматриваем возможность одновременного подавления обоих эффектов.

\subsection{Режим «квази-замороженного спина»}\label{sec:EDM/Wien_filter_tracking/QFS}

\par Как видно из Уравнения (1) для поиска ЭДМ необходимо нивелировать влияние магнитного дипольного момента (МДМ). Но кольцо ускорителя NICA имеет чисто магнитные арки, без электрической компоненты. Таким образом, невозможно использовать метод «Замороженного Спина», впервые предложенного в БНЛ [1]. Фильтры Вина, расположенные на прямом участке, компенсируют поворот спина, возникающий от МДМ на поворотной арке, и реализуют условие «Квази-Замороженного Спина» для дейтронов [2]. Для этой цели NICA нуждается в модернизации, чтобы функционировать в качестве накопительного кольца с альтернативными прямыми участками за счет использования обходных каналов ByPass [3].

\subsection{Декогеренция спина}\label{sec:EDM/Wien_filter_tracking/decoherence}

Следствием уравнения Т-БМТ (1) является частота вращения спина в электрическом и магнитном полях и задаются выражениями:

\begin{equation}
\begin{aligned}
\nu_{s}^{B} &= \gamma G \\
\nu_{s}^{E} &= \frac{G+1}{\gamma}-G \gamma
\end{aligned}
\end{equation}


\par Равновесный уровень энергии частицы

\par Разные частицы имеют различный импульс, и существует необходимость в использовании понятия эффективной энергии:

\begin{equation}
\gamma_{eff}=\gamma_s+\beta_s^2\gamma_s\Delta\delta_{eq}
\end{equation}

\par Распределение равновесного импульса из-за бетатронного движения и ненулевого коэффициента уплотнения импульса второго порядка основано на синхронном принципе [4] и определяется с помощью:

\begin{equation}
\Delta\delta_{eq}=\frac{\gamma_s^2}{\gamma_s^2\alpha_0-1}\left[\frac{\delta_0^2}{2}\left(\alpha_1+\frac{3}{2}\frac{\beta_s^2}{\gamma_s^2}-\frac{\alpha_0}{\gamma_s^2}+\frac{1}{\gamma_s^4}\right)+\left(\frac{\Delta L}{L}\right)_\beta\right]
\end{equation}

для определения удлинения орбиты из-за бетатронных колебаний:

\begin{equation}
\left(\frac{\Delta L}{L}\right)_\beta=-\frac{\pi}{L_0}\left[\epsilon_x\nu_x+\epsilon_y\nu_y\right],
\end{equation}

где индекс s означает синхронную частицу, $\epsilon_x$, $\epsilon_y$ – эмиттансы, $\nu_x$, $\nu_y$ – частота бетатронных колебаний, $\delta_0$ – относительный разброс импульса, $\alpha_0$, $\alpha_1$ – два первых порядка коэффициента уплотнения импульса. Уравнение 2 вместе с Уравнениями (3-5) показывают, что разброс спиновой частоты зависит от равновесного уровня энергии частицы.

\textbf{Удлинение орбиты и бетатронная хроматичность}

Более формальная теория подразумевает воздействие внешнего (секступольного) поля. Принимая во внимание выражение для полного удлинения орбиты из [5]:

\begin{equation}
\Delta C_\Sigma=-2\pi\left(J_x\xi_x+J_y\xi_y\right)+\delta_0\left(\alpha_0+\alpha_1\delta_0+\alpha_2\delta_0^2+\ldots\right),
\end{equation}

где $\xi_x$, $\xi_y$ – хроматичности. Если мы сравним Уравнение 6 с Уравнениями 4, 5 можно заметить, что длина орбиты тесно связана с равновесным уровнем энергии.

\subsection{Секступольная коррекция}\label{sec:EDM/Wien_filter_tracking/sextupole_correction}

\par В результате Уравнения 4, 6 показывают, что использование секступолей может влиять на частоту прецессии спина $\nu_s$ и в конечном счёте позволяют достигнуть спиновой когерентности. Такие эксперименты были проведены на ускорителе COSY в Юлихе, Германия, чтобы получить время когерентности (Spin Coherence Time) SCT на уровне $1000$ секунд [6]. Секступоли располагаются в местах с ненулевой дисперсией на поворотных арках. В минимумах и максимумах дисперсионной $D_{x,y}$ и бета $\beta_{x,y}$ функциях оказывают наибольшее воздействие и физически располагаются рядом с квадрупольными линзами. Твисс-функции арки NICA являются регулярными и показаны на Рисунке 1 [7]. Диспепсия же подавляется методом “missing magnet” по краям.

\begin{figure}[!h]
  \centering
   \includegraphics*[width=1.0\columnwidth]{4_NICA_arc.png}
   \caption{Twiss-параметры ByPass NICA для дейтронного режима в OptiM. Также показано расположение секступольных семейств.}
   \label{fig:4_NICA_arc}
\end{figure}

\textbf{Бетатронная хроматичность}
Для коррекции бетатронной хроматичности используется только 2 семейства секступолей: одно вблизи фокусирующих квадруполей, другое – рядом с дефокусирующими. Натуральная хроматичность накопительного кольца ByPass NICA равна $\nu_{x,y}=-17/-17$. После оптимизации можно отслеживать частоту прецессии спина на Рисунке 2: красная линия показывает натуральную хроматичность, синяя – скорректированную, подавленную до нуля. Для этого случая также был осуществлен спин-трекинг в течение $3\times{10}^6$ оборотов для частиц с различным начальным отклонением в координатах $x, y, d$ с начальной ориентацией спина ${\vec{S}}_0$ под углом $45$ градусов в плоскости $y-z$, что показано на Рисунке 3 [8].

\begin{figure}[!h]
  \centering
   \includegraphics*[width=1.0\columnwidth]{4_spin_decoherence.png}
   \caption{Спиновый трексинг частиц с различным начальным отклонением в координатах x, y, d с использованием 2 семейств секступолей для получения нулевой бетатронной хроматичности.}
   \label{fig:4_spin_decoherence}
\end{figure}

\textbf{Спиновая когерентность}
\par Чтобы достигнуть спиновой когерентности, рассмотрим чисто частоту прецессии спина. COSY Infinity [8] не может работать вблизи нулевого значения частоты прецессии спина. Так как это может привести к ошибке из-за резонанса, по этой причине отстраиваемся от резонанса до уровня $\nu_s~{10}^{-4}$. Но к частицам предъявляется требование прецессировать синхронно — когерентно. 
Основным параметром является частота вращения спина, которая в общем случае зависит от координат и энергии. Можно видеть, что доминирующим компонентом является квадратичный член в разложении частоты спиновой прецессии. Это видно на Рисунке 2 для обоих случаев – как для натуральной хроматичности, так и скорректированной хроматичности. По этой причине секступоли могут быть выбраны другим способом, чтобы просто достигнуть спиновой когерентности.
Как мы можем видеть из Уравнений 4, 6, недостаточно использовать 2 семейства, таким образом, третье семейство должно быть использовано для подавления зависимости от энергетической компоненты. Но в регулярных структурах бета и дисперсионные $\beta$, $D$ - функции не позволяют использовать 3 линейных независимых семейства. На Рисунке 1 показано расположение секступольных семейств: SF1, SF2, SD. В этом методе мы не влияем на хроматичность, просто отслеживаем её значение $\nu_{x,y}=-13/-18$. Этого недостаточно для обеспечения стабильного орбитального движения. В этом случае можно видеть, что спиновая когерентность достигнута – нет зависимости частоты спиновой прецессии от координат и энергии (Рисунок 2: зеленая линия). Результаты спинового трекинга частиц подтверждают это утверждение. На Рисунке 4, частота вращения спина $\nu_s~{10}^{-7}$, количество оборотов $3\times{10}^6$ оборотов или $3$ секунды. Частицы с различным начальным отклонением прецессируют с одинаковой спиновой частотой. Но в этом случае максимум секступольного коэффициента принимает большое значение, что может вызвать нелинейные эффекты (Таблица 1).

\begin{figure}[!h]
  \centering
   \includegraphics*[width=1.0\columnwidth]{4_spin_coherence.png}
   \caption{Спиновый трексинг частиц с различным начальным отклонением в координатах x, y, d с использованием 3 семейств секступолей для получения спиновой когерентности.}
   \label{fig:4_spin_coherence}
\end{figure}

\subsection{Коррекция $\alpha_{1}, \eta_{1}$}\label{sec:EDM/Wien_filter_tracking/correction}
\par Как мы можем видеть, чистая коррекция бетатронной хроматичности не позволила нам получить нулевой разброс частоты вращения спина. Одновременно, получение спиновой когерентности, путем подавления квадратичного члена частоты спиновой прецессии, не подавляет хроматичность. Это возвращает нас к Уравнению 6. Значение $\delta_0\alpha_0$ может быть усреднено с использованием RF для смешивания. Таким образом, чтобы гарантировать нулевое удлинение орбиты, хроматичности должны быть подавлены $\xi_x,\xi_y$ вместе со значением $\alpha_0$ до нуля. Это также возможно при использовании 3-х семейств секступолей. Но все равно не позволяет добиться спиновой когерентности. На Рисунке 2 (фиолетовая линия) показана ненулевая зависимость частоты прецессии спина от координат. То же самое происходит, если мы следуем Уравнению 4 и подавляем значение $\eta_1$ вместе с коррекцией хроматичности (Рисунок 2). Кроме того, максимальное значение секступольного градиента слишком велико и не может быть реализована (Таблица 1).

\begin{figure}[!h]
  \centering
   \includegraphics*[width=0.32\columnwidth]{4_spin_dependance_x.png}
   \includegraphics*[width=0.32\columnwidth]{4_spin_dependance_y.png}
   \includegraphics*[width=0.32\columnwidth]{4_spin_dependance_d.png}
   \caption{Зависимость частоты прецессии спина от координат x, y, d для различных случаев оптимизации. NC – натуральная хроматичность (красная линия); BC – нулевая (бетатронная) хроматичность (синяя пунктирная линия); SC – спиновая когерентность (зеленая линия); $BC_{\alpha}$ – нулевая хроматичность и $\alpha_1=0$ (фиолетовая линия); $BC_{eta}$ – нулевая хроматичность и ноль $\eta_1=0$ (светло-голубая линия).}
   \label{fig:4_spin_dependance}
\end{figure}


\subsection{Вывод}\label{sec:EDM/Wien_filter_tracking/final}
Подводя итог, рассмотрено явление спиновой декогеренции одновременно с бетатронной хроматичностью на накопительном кольце ByPass NICA. При этом, кольцо, функционирует в режиме "Квази-Замороженного спина" и может быть использован для экспериментов по исследованию ЭДМ дейтрона. 
Были рассмотрены различные случаи оптимизации секступолями. Квадратичные члены в разложении по частоте спиновой прецессии являются наиболее важными и представляют зависимость от координат и энергии. Все основные параметры, которые подвергались мониторингу, приведены в Таблице 1. Исследование показывает, что невозможно использовать $3$ семейства секступолей в регулярной структуре для достижения как бетатронной хроматичности, так и спиновой когерентности. Более того, максимальное значение коэффициента секступолей неудовлетворительно и может привести к нелинейным неустойчивостям. Стоит отметить, что регулярная дисперсионная функция на поворотной арке не позволяет найти $3$ линейных независимых семейства, так как они располагаются в одних и тех же минимумах/максимумах бета и дисперсионных $\beta$, $D$ - функциях. Однако, возможно промодулировать дисперсионную функцию таким образом, чтобы получить $3$ линейных независимых семейства секступолей. Также одним из возможных решений проблемы является использование охлажденного пучка на уровне $\sfrac{dp}{p}\approx{10}^{-5}$. Это может помочь свести к минимуму $\gamma$–эффективное и, наконец, обеспечить спиновую когерентность одновременно с подавленной бетатронной хроматичностью.


\section{Исследование поведения поляризации пучка при совершении процедуры скачка критической энергии}\label{sec:EDM/jump}

\section{Результаты математического 3D моделирования динамики пучка в ускорительном комплексе NICA-Nuclotron}\label{sec:EDM/tracking}

\FloatBarrier
