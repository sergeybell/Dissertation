%part 2

	\chapter{Регулирование критической энергии методом резонансной вариации дисперсионной функции в коллайдере NICA }\label{ch:transition_variation}

\par Эта глава посвящена адаптации структуры коллайдера NICA с варьируемой критической энергией для возможности проведения коллайдерных экспериментов с поляризованными пучками протонов и дейтронов на SPD детекторе.

\par Изначально структура коллайдера NICA проектировалась как дуальная для работы в двух модах: для экспериментов с тяжелыми ионами $^{\mathbf{79}}{\mathbf{Au}}_{\mathbf{197}}$ и для экспериментов с поляризованными протонами/дейтронами $\mathbf{p, d}$. В эксперименте по столкновению тяжелых ионов золота c максимальной энергией $E_{exp}=4.5$ ГэВ/нуклон критическая энергия магнитооптической структуры коллайдера составляет $E_{tr}^{Au-Au}=5.7$\ ГэВ ($\gamma_{tr}^{Au-Au}=7.1$). Такое значение критической энергии было достигнуто выбором частоты бетатронных колебаний в горизонтальной плоскости $\nu_x\approx\gamma_{tr}^{Au}>\gamma_{max}^{Au}\approx7.1$, которая  при условии регулярности структуры арок, состоящих из одинаковых ячеек ФОДО, должна быть больше максимального значения фактора Лоренца во всем интервале энергий.В этом случае проблем с прохождением критической энергии не возникает, что было изначально учтено при проектировании. Однако, спроектированная и построенная регулярная структура, выбранная для тяжелоионной программы имеет фиксированное значение критической энергии и является характеристикой конкретной установки, то есть не отличается для разного сорта частиц.

\par Как было показано в Главе 2, при прохождении через критическую энергию $\gamma_{tr}$ развивается продольная неустойчивость. Пороговый ток ее развития $I_{th}~\eta$ пропорционален коэффициенту расширения орбиты $\eta$, который равен нулю при  $\gamma=\gamma_{tr}=7.1$. Темп ускорения протонов при прохождении через критическую энергию при использовании индукционного ускорения ВЧ1 станции составляет $d\gamma/dt=0.2c^{-1}$. Этот темп слишком мал, чтобы избежать развития неустойчивости при приближении релятивистского фактора к $\gamma_{tr}$. Чтобы исключить при ускорении протонов прохождение через критическую энергию, для протонной моды должна быть реализована новая оптическая структура колец вместо оптической структуры ионной моды. В этой оптической структуре критическая энергия должна быть выше максимальной энергии протонов при работе коллайдера на эксперимент.

\par Максимальная магнитная жесткость поворотных магнитов постоянна $R_{arc}\cdot B_{bend}=\frac{A\cdot m c\gamma\beta}{eZ}\approx45$. В силу того, что отношения массы к заряду $\left(A/Z\right)_{p}=1/1=1$, для ионов золота $\left(A/Z\right)_{p}=197/79=2.2$. Тем самым, определяется максимально возможная энергия для протонов $E_{max}^p=12,4$\ ГэВ ($\gamma_{p}=14,3$), следовательно критическая энергия должна быть, что заведомо выше критической энергии для ионной регулярной структуры $E_{tr}^{Au-Au}=5,7$\ ГэВ ( $\gamma_{tr}^{Au-Au}=7,1$).

\par Как уже было показано в Главе 1, характер поведения тяжелых ионов и легких частиц, существенно отличается с точки зрения внутрипучкового рассеяниия. Для протонного пучка с интенсивностью $2\times10^{13}$ время внутрипучкового нагрева возрастает примерно в 30 раз по сравнению с пучками ионов золота с интенсивностью $6.6\times10^{10}$. Поэтому критическая энергия может подниматься за счет вариации дисперсии без опасения влияния внутрипучкового рассеяния. За счет резонансной модуляции дисперсионной функции коэффициент диффузии для внутрипучкового рассеяния возрастает в 2-3 раза, что не критично как при охлаждении протонов во время накопления, так и при их группировке на энергии эксперимента.

\par Теоретический метод был развит в работе ... . Его применение для современных ускорительных установок рассмотрено в работах ... .

\par В данной главе будут приведен краткое теоретической обоснование применения метода резонансной структуры. Применение метода для адаптации регулярной магнитооптической структуры NICA к резонансной. А также анализ полученной структуры с точки зрения компенсации нелинейных эффектов, компенсации хроматичности и  динамической апертуре.
  
\section{Краткий обзор теории построения «Резонансной» магнитооптической структуры}\label{sec:transition_variation/methods/resonant}

Здесь будет приведено кратное теоретическое обоснование метода резонансных структур, применяемых для обеспечения вариации критической энергии.

В общем смысле коэффициент расширения орбиты (momentum compaction factor) определяется как 

\begin{equation}
\alpha=\frac{1}{{\gamma_{tr}}^2}=\frac{1}{C}\int_{0}^{C}\frac{D\left(s\right)}{\rho\left(s\right)}ds,
\label{eq:alpha}
\end{equation}

\noindent где $C$ – длина замкнутой равновесной орбиты, $D(s)$ – горизонтальная дисперсионная функция, $\rho(s)$ – радиус кривизны равновесной орбиты. Коэффициент $\alpha$ может быть варьирован как при помощи модуляции дисперсионной функции, так и кривизны независимо.

Уравнение для дисперсионной функции с бипериодической переменной переменной фокусировкой описывает поведение дисперсионной функции под влиянием градиента в квадрупольных линзах [?]

\begin{equation}
\frac{d^2D}{ds^2}+\left[K(s)+\varepsilon k(s)\right]D=\frac{1}{\rho(s)} ,
\label{eq:disp_eq}
\end{equation}

\noindent где $K\left(s\right)=\frac{e}{p}G\left(s\right), \varepsilon k\left(s\right)=\frac{e}{p}\Delta G\left(s\right), G\left(s\right)$ – градиент магнитооптических линз, $\Delta G\left(s\right)$ – суперпериодическая модуляция градиента, $p=\beta\gamma m_0 c$ – импульс частицы. Суперпериод определяется как совокупность нескольких периодов. Функция $K\left(s\right)$ имеет периодичность одного периода фокусирующей ячейки $L_{c}$, $k(s)$ и $\rho(s)$ имеет периодичность суперпериода $L_s = n L_c$.

Разложение в ряд Фурье для зеркального суперпериода может быть осуществлено только по косинусам

\begin{equation}
\varepsilon k\left(s\right)=\sum_{k=0}^{\infty}g_{k}\cos(k\phi)
\label{eq:superperiodicity_fourier}
\end{equation}

\noindent где $\phi=\frac{2\pi}{L_s}s$ -- угловая продольная координата, $k$-ая гармоника:
\begin{equation}
g_k=\frac{1}{B\bar{R}} \frac{1}{\pi} \int_{-\pi}^\pi \Delta G \cos k \phi d \phi,
\end{equation}
\noindent $B\bar{R}$ -- магнитная жесткость.

\noindent Радиус кривизны орбиты также может быть разложен в ряд Фурье по косинусам

\begin{equation}
\frac{1}{\rho(\phi)}=\frac{1}{\bar{R}}\left(1+\sum_{n=1}^{\infty} r_n \cos n \phi\right),
\end{equation}

\noindent где $n$-ая гармоника

\begin{equation}
r_n=\frac{\bar{R}}{\pi} \int_\pi^{-\pi} \frac{\cos n \phi}{\rho(\phi)} d \phi.
\end{equation}

\noindent Полученные разложения могут быть подставлены в уравнение (?), из которых получено необходимое точное решение для дисперсионной функции $D(s)$. Таким образом, окончательно для коэффициента уплотнения орбиты одного суперпериода получено выражение в общем виде, разложенное до второго порядка

\begin{equation}
\begin{gathered}
\alpha_s=\frac{1}{\nu^2}\left\{1+\frac{1}{4}\left(\frac{\bar{R}}{\nu}\right)^4 \sum_{k=-\infty}^{\infty}\right.
\frac{g_k^2}{(1-k S / \nu)\left[1-(1-k S / \nu)^2\right]^2}+ \\
+\frac{1}{4} \sum_{k=-\infty}^{\infty} \frac{r_k^2}{1-k S / \nu} -\frac{1}{2}\left(\frac{\bar{R}}{\nu}\right)^2 \sum_{k=-\infty}^{\infty} \frac{r_k g_k}{(1-k S / \nu)\left[1-(1-k S / \nu)^2\right]} - \\
-\frac{1}{2}\left(\frac{\bar{R}}{\nu}\right)^2 \sum_{k=-\infty}^{\infty} \frac{r_k g_k}{1-(1-k S / \nu)^2}
\left.+O\left(g_k^i, r_k^j, i+j \geq 3\right)\right\},
\end{gathered}
\label{eq:alpha_general}
\end{equation}

\noindent где ${\overline{R}}_{arc}$ -- усредненное значение кривизны, $\nu_{x}$ -- количество горизонтальных бетатронных колебаний на длине арке, $S$ -- количество суперпериодов на длине арки.

\noindent В случает отсутствия суперпериодической модуляции и модуляции кривизны орбиты $g_k=0, r_n=0, \  \forall k,n$, формула \ref{eq:alpha_general} принимает вид $\alpha_s=\frac{1}{{\gamma_{tr}}^2}=\frac{1}{{\nu_x}^2}$, что соответствует случаю регулярной структуры и окончательно  $\gamma_{tr}\sim\nu_x$.

\par Для поднятия критической энергии необходимо уменьшить коэффицент $\alpha_s=\frac{1}{{\gamma_{tr}}^2}$, а значит выражение под знаком суммы в ур.\ref{eq:alpha_general} должно быть отрицательным, это реализуемо при условии $kS/\nu_{x}>1$.

Ранее все формулы были приведены для арки, а не для всего кольца коллайдера. Введение прямых участков уменьшает степень модуляции дисперсионной функции. Усреднение дисперсии по более длинной орбите автоматически уменьшает ее значение, а значит уменьшает коэффициент уплотнения орбиты для всего ускорителя, результирующее значение критической энергии $\gamma_{tr}^{total}$ увеличивается и определяется выражением:

\begin{equation}
\gamma_{tr}^{total}=\gamma_{tr}^{arc}\sqrt{\frac{S\cdot L_s+L_{str}}{S\cdot L_s}}.
\label {eq:gamma_tr_modulated}
\end{equation}

\noindent где $L_s$ -- длина суперпериода, $L_{str}$ -- длина бесдисперсионных прямых секций.

	\section{Построение резонансной структуры на основе ячеек ФОДО, ФДО, ОДФДО}\label{sec:transition_variation/methods/FODO_FDO}
	
\par Принцип построения резонансных магнитооптических структур рассмотрен в работе ... .
Суперпериод может быть образован на основе синглетных ФОДО ячейках, дублетных ФДО ячейках, а также триплетных ОДФДО (рис.\ref{fig:fodo_fdo_odfdo} a,б,в). Рассмотрим структуры поворотных арок на 180 градусов без модуляции кривизны (рис.\ref{fig:fodo_fdo_odfdo} г, д, е), образованных из соответствующих ячеек. Из полученных суперпериодов также образуется резонансная магнитооптическая структура путем только модуляции градиента (рис.\ref{fig:fodo_fdo_odfdo} ж, з, и). Резонансная структура образуется путем вариации параметров регулярной структуры.

\begin{figure} [h!]

   \includegraphics*[width=.32\columnwidth]{2_twiss_fodo_cell.pdf}
   \includegraphics*[width=.32\columnwidth]{2_twiss_fdo_cell.pdf}
   \includegraphics*[width=.32\columnwidth]{2_twiss_odfdo_cell.pdf}

   \includegraphics*[width=.32\columnwidth]{2_twiss_fodo_regular.pdf}
   \includegraphics*[width=.32\columnwidth]{2_twiss_fdo_regular.pdf}
   \includegraphics*[width=.32\columnwidth]{2_twiss_odfdo_regular.pdf}

   \includegraphics*[width=.32\columnwidth]{2_twiss_fodo_resonant.pdf}
   \includegraphics*[width=.32\columnwidth]{2_twiss_fdo_resonant.pdf}
   \includegraphics*[width=.32\columnwidth]{2_twiss_odfdo_resonant.pdf}

   \caption{Твисс-параметры $\beta_{x,y}$, $D_{x}$. Сверху -- для ячеек для сигнлетной ФОДО, дублетной ФДО, триплетной ОДФДО ячеек; посредине -- регулярная структура; снизу -- резонансная.}
   \label{fig:fodo_fdo_odfdo}
\end{figure}

	\section{Оптимизация магнитооптической структуры коллайдера}\label{sec:transition_variation/methods/optimization}
\par Качественное отличие в пространственном распределении Твисс-параметров $\beta_{x}$, $\beta_{y}, D_{x}$ является ключевым для соответствующей оптимизации структуры.
Как видно из приведенных структур, суперпериод, основанный на синглетных ФОДО ячейках может иметь ряд преимуществ.

\par Хроматичность

\begin{equation}
\begin{aligned}
& C_x=\frac{-1}{4 \pi} \oint \beta_x\left[K_x(s)-S(s) D(s)\right] d s \\
& C_z=\frac{-1}{4 \pi} \oint \beta_z\left[K_z(s)+S(s) D(s)\right] d s .
\end{aligned}
\end{equation}

Во-первых, для подавления хроматических эффектов, требуются меньшие градиенты в секступольных линзах.
Во-вторых, более простой способ коррекции и тонкой настройки набега бетатронных частот в обеих плоскостях, а также коэффициента расширения орбиты.
Таким образом, является более предпочтительной по сравнению с аналогичными.


	\section{Регулярная ФОДО структура с суперпериодической модуляцией градиентов линз}\label{sec:transition_variation/methods/FODO}

\par В структуре NICA регулярная расстановка отклоняющих магнитов исключает возможность модуляции кривизны орбиты  $r_n=0, \ \forall n$. Тогда для одного суперпериода коэффициент расширения орбиты (\ref{eq:alpha_general}) в первом приближении ($k=1$) определяется по формуле:

\begin{equation}
\begin{aligned}
\alpha_s= & \frac{1}{\nu^2}\left\{1+\frac{1}{4(1-k S / \nu)}\left(\frac{\bar{R}}{\nu}\right)^4 \frac{g_k^2}{\left[1-(1-k S / \nu)^2\right]^2}\right\},
\end{aligned}
\label{eq:alpha_gradient}
\end{equation}

\noindent Таким образом, доступным средством вариации критической энергии, является только модуляция градиента квадрупольных линз по длине суперпериода.

\begin{figure} [h!]
   \includegraphics*[width=1.0\columnwidth]{2_superperiod.png}
   \caption{Суперпериод, состоящий из 3-х ФОДО ячеек. QF1, QF2 – фокусирующие квадруполи, QD –  дефокусирующие квадруполи, B – поворотный магнит.}
   \label{fig:superperiod_3FODO}
\end{figure}

\par Квадрупольная фокусирующая структура поворотных арок коллайдера NICA состоит из ФОДО ячеек. Одинаковые элементы, расположенные в различных местах арки объединяют в одно семейство. На Рис.\ref{fig:superperiod_3FODO} изображен один суперпериод, который состоит из 3-х ФОДО ячеек, c двумя семействами фокусирующих квадруполей (QF1 и QF2) и одним семейством дефокусирующих (QD).

\begin{figure}
   \includegraphics*[width=.49\columnwidth]{2_twiss_3FODO_regular}
   \includegraphics*[width=.49\columnwidth]{2_twiss_3FODO_modulated}
   \caption{Twiss-параметры 3-х ячеек. Слева – регулярная структура без модуляции, справа – модулированная c введением суперпериодичности, глубина модуляции 24\%.}
   \label{fig:twiss_3FODO}
\end{figure}

На Рис.\ref{fig:twiss_3FODO} приведены 3 ФОДО ячейки, первая – используется в регулярной тяжелоионной структуре, в этом случае модуляция отсутствует, вторая – модулированная структура,  которая и образует один суперпериод. В обоих случаях частота бетатронных колебаний $\nu_{x,y\ S}=0,75$, таким образом для 4-х суперпериодов частота $\nu_{x,y\ arc}=3$, что удовлетворяет ранее рассмотренному условию $S=4, \nu_x=3$.

Глубина модуляции определяется соотношением градиентов двух различных фокусирующих семейств. Для приведенного на правом рисунке Рис.\ref{fig:twiss_3FODO} $G_{\textrm{QF1}}=27.7$ T/m, $G_{\textrm{QF2}}=21.0$ T/m. Таким образом глубина модуляции:

\begin{equation}
H=\frac{G_{\textrm{QF1}}-G_{\textrm{QF2}}}{G_{\textrm{QF1}}}=24\%
\label {eq:gamma_tr_modulated}
\end{equation}

\par Первая гармоника является определяющей и для 12 ФОДО ячеек реализуемо условие $S=4, \nu_x=3$, где 3 ФОДО ячейки объединены в один суперпериод. Таким образом, благодаря набегу бетатронных колебаний кратному $2\pi$, в нашем случае $6\pi$, арка имеет свойства ахромата первого порядка. В дальнейшем это свойство будет использовано для подавления дисперсии.

\par Для арки, составленной из 4-х одинаковых суперпериодов с критической энергией  на арке $\gamma_{tr}^{arc}=\ 6$, по формуле (5) для всего кольца получаем $\gamma_{tot}^{arc}\ \approx\ 11.3$. Однако, конечная арка будет нерегулярной в силу необходимости подавления дисперсии на краях арки. А значит и значение критической энергии будет несколько отличаться.

\section{Подавление дисперсии на краях поворотных арок}\label{sec:transition_variation/disp_supperssion}

\par Важным требованием при проектировании магнитооптической структуры является обеспечение нулевого значения дисперсии на прямых участках для обеспечения движения частиц вдоль равновесной орбиты на этих участках.

При введении прямых участков необходимо подавить дисперсию на конце каждой арки.

Рассматривается 4 варианта подавления дисперсии:

\begin{enumerate} 
\item	Полносью регулярная арка. \ Регулярная арка состоящая из 4-х одинаковых суперпериодов. При набеге фазы на арке кратному $2\pi$ подавление дисперсии происходит в силу регулярности;
\item Регулярная арка с missing magnet на краях.
\item	Подавление дисперсии при помощи крайних суперпериодов.\
Регулярная арка также состоящая из 4-х одинаковых суперпериодов. При набеге фазы на арке не кратному $2\pi$ необходимо дополнительно добавить подавители дисперсии на краях арки.\
А именно двух крайних ФОДО ячеек. На Рисунке 2 и 3 сверху (Edge Suppressor – ES) изображена принципиальная схема данной магнитооптической структуры. Как видно две крайние ФОДО ячейки отличаются наличием missing-magnet и в этих ячейках квадруполи QFE1 и QFE2 также имеют отличные градиенты от основных квадруполей арки и подбираются таким образом, чтобы подавить дисперсию.
\item	Подавление дисперсии всей аркой, при помощи выбора градиентов квадруполей двух семейств.\
Арка состоящая из 4-х суперпериодов, крайние суперпериоды выполняют роль подавителей дисперсии, а именно 2 крайние ячейки.\
На Рисунке 1 снизу (Arc Suppressor – AS) изображена принципиальная схема данной магнитооптической структуры. Этот случай отличается тем, что все квадруполи арки принадлежат первому, либо второму семейству и подавление дисперсии также обеспечивается только 2-мя семействами.
\end{enumerate} 

\noindent Дефокусирующие же квадруполи во всех случаях принадлежат только одному семейству QD.	

\subsection{Полностью регулярная магнитооптическая структура}\label{subsec:transition_variation/disp_supperssion/regular}

\par Требование подавления легко реализуемо в случае создания регулярных поворотных арок, составленных из одинаковых суперпериодов. В этом случае, обеспечив нулевое значение дисперсии $D=0$ (а также производной дисперсии $D\prime=0$) на входе в арку, в силу регулярности на выходе из арке также будут нулевые значения дисперсии и её производной, а следовательно и на всем прямом участке. Однако, учитывая особенность структуры коллайдера NICA, наличие missing-магнитов на двух крайних cell’aх не дает возможность создать полностью регулярную арку из 4-х одинаковых суперпериодов.

\begin{figure} [h!]
   \center
   \includegraphics*[width=1.\columnwidth]{2_supp_scheme_regular.png}
   \includegraphics*[width=1.\columnwidth]{2_supp_Twiss_regular.png}
   \includegraphics*[width=1.\columnwidth]{2_supp_elem_regular.png}
   \caption{Подавление дисперсии в регулярной структуре.}
   \label{fig:2_disp_supp_full_regular}
\end{figure}
	
\subsection{Подавление дисперсии при помощи крайних суперпериодов}\label{subsec:transition_variation/disp_supperssion/ES}

\par	
\begin{figure} [h!]
   \center
   \includegraphics*[width=1.\columnwidth]{2_supp_scheme_ES.png}
   \includegraphics*[width=1.\columnwidth]{2_supp_Twiss_ES.png}
   \includegraphics*[width=1.\columnwidth]{2_supp_elem_ES.png}
   \caption{Подавление дисперсии в тяжелоионной структуре.}
   \label{fig:2_disp_supp_ES}
\end{figure}	

\begin{figure} [h!]
   \center
   \includegraphics*[width=1.\columnwidth]{2_supp_scheme_AS.png}
   \includegraphics*[width=1.\columnwidth]{2_supp_Twiss_AS.png}
   \includegraphics*[width=1.\columnwidth]{2_supp_elem_AS.png}
   \caption{Подавление дисперсии в тяжелоионной структуре.}
   \label{fig:2_disp_supp_AS}
\end{figure}	

\par Выбор значения градиентов квадруполей арки определяется двумя факторами:
	Получение необходимого значения критической энергии на всем кольце коллайдера, что соответствует $\gamma_{tr}~15-16$;
	Обеспечить количество бетатронных колебаний на арке $\nu_{arc}=3$ в обоих плоскостях, тем самым удовлетворив резонансному условию при количестве суперпериодов $S=4$.
	
\par Исходя из этих условий модулируем суперпериод с набегом фазы на суперпериоде $\nu_S=0.75$ в обоих плоскостях (Рисунок 4).

\par Коллайдер также состоит из 2-х арок и 2-х прямых участках, соединяющих арки. Посередине прямых участков имеются точки столкновения, где нужно обеспечить малое значение бета-функции для достижения требуемой светимости. На Рисунке 6 изображены параметры Твисса всего кольца коллайдера без введения крайних квадруполей QFE1 и QFE2, наглядно видно, что дисперсия не подавлена на прямых участках. Крайний суперпериод имеет missing magnet в 2-х cell’ах, тем самым делая арки коллайдера не регулярными и возникает необходимость подавления дисперсии на прямых участках при помощи введения 2-х дополнительных семейств квадруполей QFE1 и QFE2 на краю арки, параметры Твисса изображены на Рисунке 5.
В результате значение критической энергии подобрано таким образом, что $\gamma_{tr}=15.6$, а количество колебаний на арке: $\nu_{x\ arc}=3.01,\ \nu_{y\ arc}=3.01$.


\subsection{Подавление дисперсии всей аркой, при помощи выбора градиентов квадруполей двух семейств.}\label{subsec:transition_variation/methods/disp_supperssion_AS}	

Данный способ показывает возможность подавления дисперсии на прямых участках при помощи только двух семейств фокусирующих квадруполей. 
Тут важно учесть, как и в первом случае выполнить:
	Получение необходимого значения критической энергии на всем кольце коллайдера, что соответствует $\gamma_{tr}~15-16$;
	Только при помощи квадруполями двух семейств подавить дисперсию на прямых участках
	
\par Изначально выбирается суперпериод, как и в первом случае с набегом на суперпериоде $\nu_s=0.75$. Тем самым получаем значения квадруполей QF1 и QF2 для всей арки, в том числе и на краях.
\par Однако, получается, что дисперсия на прямых участках оказывается не подавленной. Для подавления значения градиентов квадруполей изменяется, но в таком случае набег фазы на арке становится равен $\nu_{x\ arc}=2.72178$,\ $\nu_{y\ arc}=2.99884$, то есть в x–плоскости кратен $2\pi$.

\par В этом случае для достижения требуемого значения критической энергии необходимо обеспечить большую модуляцию градиентов квадруполей, чем в случае подавления дисперсии крайними суперпериодами.
	
\begin{figure} [h!]
   \center
   \includegraphics*[width=.5\columnwidth]{2_disp_supp}
   \caption{Подавление дисперсии в тяжелоионной структуре.}
   \label{fig:2_disp_supp}
\end{figure}

\par Расстановка поворотных магнитов в тяжелоионной структуре на краях арки не позволяет реализовать подавление дисперсии как в случае а), так и в случае б). На рисунке 4 изображены 3 ФОДО ячейки, подавляющие дисперсию на краю арки для регулярной тяжелоионной структуры. Отличительной особенностью является отсутствие 2-х поворотных магнитов. Это связано с тем, что в этих местах происходит инжекция частиц в коллайдер из нуклотрона.

\begin{figure}
   \center
   \includegraphics*[width=.5\columnwidth]{2_disp_supp_modulated}
   \caption{Слева – таблица со значением градиента в квадруполях. Справа – подавленная дисперсия в арке.}
   \label{fig:2_disp_supp_modulated}
\end{figure}
	
\par Подавление дисперсии также осуществляется с помощью 2-х семейств фокусирующих QFE1 и QFE2 квадруполей в начале и на конце арки.  Таким образом для прямых участков дисперсионная функция будет нулевой. Для полученной нерегулярной арки значение критической энергии $\gamma_{tr}^{arc}=\ 10$.

	\section{Подавление натуральной хроматичности и компенсация нелинейных эффектов хроматических секступолей}\label{sec:transition_variation/methods/chromaticity}

\par Добавление секступолей, подавляющих хроматичность внутри арки делает арку ахроматом второго порядка, что убирает зависимость бетатронных колебаний от импульса и способствует сохранению динамической апертуры в большом диапазоне энергий. Выбор нечетного значения частоты на арке $\nu_{x,\ arc}=3$ и четного значения суперпериодичности арки $S=4$ замечателен еще и тем, что позволяет компенсировать нелинейный вклад секступолей внутри арки. В этом случае набег фазы радиальных колебаний между ячейками, расположенными в разных половинках арки и разделенных $S/2$ числом суперпериодов равен:

\begin{figure}
   \includegraphics*[width=1.0\columnwidth]{2_modulated_ring}
   \caption{Twiss-параметры протонной опции коллайдера NICA с $\gamma_{tr}^{tot}\approx13$.}
   \label{fig:2_modulated_ring}
\end{figure}

\begin{equation}
2\pi\cdot\frac{\nu_{arc}}{S_{arc}}\cdot\frac{S_{arc}}{2}=2\pi\cdot\frac{\nu_{arc}}{2}=\pi+2\pi n,\
\label{eq:chrom_period}
\end{equation}

что соответствует условию компенсации нелинейного влияния секступолей в первом приближении во всей арке. Это замечательное свойство также относится к высшим мультиполям в квадруполях и отклоняющих магнитах. Эта связь через число суперпериодов $S_{arc}/2$ будет называться длинной связью.

\section{Заключение}

\par Рассмотрена методика вариации критической энергии методом модуляции градиента квадрупольных линз на арках в применении к ускорительному комплексу NICA. Такой случай предполагает раздельное питание квадруполей. Также учтена необходимость подавления дисперсии на краях арки в имеющейся структуре и подавление хроматичности на всем кольце коллайдера.

\section{Оптимизация динамической апертуры и выбор рабочей точки}\label{sec:transition_variation/methods/DA_optimization}

\section{Исследование динамической апертуры в синхротроне NICA с учетом требуемой модуляции дисперсионной функции для повышения критической энергии}

\begin{figure} [h!]
   \center
   \includegraphics*[width=0.49\columnwidth]{2_ES_x_dpp0.png}
   \includegraphics*[width=0.49\columnwidth]{2_ES_y_dpp0.png}
   \includegraphics*[width=0.49\columnwidth]{2_ES_x_dpp0,3.png}
   \includegraphics*[width=0.49\columnwidth]{2_ES_y_dpp0,3.png}
   \includegraphics*[width=0.49\columnwidth]{2_ES_x_dpp0,5.png}
   \includegraphics*[width=0.49\columnwidth]{2_ES_y_dpp0,5.png}
   \caption{Динамическая аппретура для случая плавления дисперсии крайними квадруполями. 
Слева – x-плоскость; справа – y-плоскость.}
   \label{fig:DA_ES_dpp}
\end{figure}	

\begin{figure} [h!]
   \center
   \includegraphics*[width=0.49\columnwidth]{2_AS_x_dpp0.png}
   \includegraphics*[width=0.49\columnwidth]{2_AS_y_dpp0.png}
   \includegraphics*[width=0.49\columnwidth]{2_AS_x_dpp0,3.png}
   \includegraphics*[width=0.49\columnwidth]{2_AS_y_dpp0,3.png}
   \includegraphics*[width=0.49\columnwidth]{2_AS_x_dpp0,5.png}
   \includegraphics*[width=0.49\columnwidth]{2_AS_y_dpp0,5.png}
   \caption{Динамическая апертура в случае подавление дисперсии двумя семействами квадруполей.
Слева – x-плоскость; справа – y-плоскость.}
   \label{fig:DA_AS_dpp}
\end{figure}	

\FloatBarrier