\chapter*{Заключение}                       % Заголовок
\addcontentsline{toc}{chapter}{Заключение}  % Добавляем его в оглавление

%% Согласно ГОСТ Р 7.0.11-2011:
%% 5.3.3 В заключении диссертации излагают итоги выполненного исследования, рекомендации, перспективы дальнейшей разработки темы.
%% 9.2.3 В заключении автореферата диссертации излагают итоги данного исследования, рекомендации и перспективы дальнейшей разработки темы.
%% Поэтому имеет смысл сделать эту часть общей и загрузить из одного файла в автореферат и в диссертацию:

Основные результаты работы заключаются в следующем.
%% Согласно ГОСТ Р 7.0.11-2011:
%% 5.3.3 В заключении диссертации излагают итоги выполненного исследования, рекомендации, перспективы дальнейшей разработки темы.
%% 9.2.3 В заключении автореферата диссертации излагают итоги данного исследования, рекомендации и перспективы дальнейшей разработки темы.
\begin{enumerate}
  \item На основе анализа внутрипучкового рассеяния, а также стохастического охлаждения показано, что использование метода 'резонансной' структуры способно увеличить эффективность стохастического охлаждения. Особенно эффективным может быть использование 'комбинированной' структуры. Однако, эффекты ВПР для приведенных структуры оказались в несколько раз большими и в конечном счёте недостаточными, делая предпочтительной 'регулярную' структуру для тяжелоионного эксперимента.
  \item Для коллайдерных экспериментов с протонами рассмотрена 'резонансная' структура в варьированой критическая энергия, что использовано для рассмотрения адаптированной структуры коллайдера NICA.
  \item Численные исследования показали, что прохождение критической энергии может вызывать нестабильность продольного фазового движения. Использование процедуры скачка критической энергии может быть использовано для преодоления этой проблемы. Получены экспериментальные данные процедуры скачка критической с синхротрона У-70, которые находятся в соответствии с проведенным численными оценками с учетом высших порядков разложения коэффициента уплотнения орбиты и импедансов для различных интенсивностей сгустка.
  \item Использование процедуры скачка для коллайдера NICA ограничено величиной скачка критической энергии, а также для гармонического ВЧ темпом изменения критической энергии по сравнению с темпом ускорения пучка. Что делает невозможным использование процедуры для этого типа ВЧ. Для барьерного ВЧ приведены оценки продольной микроволновой неустойчивости, показывающие существенное ограничение на параметры конечного сгустка.
  \item Для исследования спиновой динамики и реализации "квази-замороженного" спина в коллайдере NICA рассмотрено введение обводных каналов bypass. На прямых участках предлагается расположение фильтров Вина для компенсации поворота спина под действием МДМ в магнитной арке.
  \item Рассмотрена модернизированная структура синхротрона Nuclotron с сохранением функции бустера поляризованного пучка в коллайдер NICA. В предложенных 8/16-периодичных структурах возможно проведение независимых прецезионных экспериментов по исследованию ЭДМ дейтрона и протона, а также осуществлению поика аксиона в режиме сканирующей антенны.
\end{enumerate}


В заключение автор
выражает благодарность и большую признательность научному руководителю
Сеничеву~Ю.\,В. за поддержку, помощь, обсуждение результатов и~научное
руководство. 
Также автор благодарит коллег Аксентьева~А.\,Е.,~Мельникова~А.\,А. и Паламарчук П.И. за помощь в регулярных обсуждениях. Cотрудников ОИЯИ Лебедева~В.\,A. за плодотворные дискуссии, Сыресина~E.\,М. и Ладыгина~В.\,П. за поддержку в изучении установки Nuclotron-NICA.

Автор также благодарит академика РАН Иванова~С.\,В. за возможность участия в сеансе на синхротроне У-70, а также сотрудников НИЦ "Курчатовский институт" - ИФВЭ Калинина~В.\,А., Пашкова~П.\,Т., Ермолаева~А.\,Д. за всестороннюю помощь в проведении экспериментальных наблюдений.