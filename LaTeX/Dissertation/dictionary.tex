\chapter*{Словарь терминов}             % Заголовок
\addcontentsline{toc}{chapter}{Словарь терминов}  % Добавляем его в оглавление

\textbf{Критическая энергия} : (\textit{aнгл.} transition energy) ;\\
\textbf{Коэффициент уплотнения орбиты} : (\textit{aнгл.} momentum compaction factor) показатель зависимости относительного удлинения обриты от разброса по импульсу нереференсной частицы;\\ 
\textbf{Коэффициент проскальзывания} :  (\textit{aнгл.} slip-factor) показатель зависимости относительного изменения периода обращения от разброса по импульсу нереференсной частицы;\\
\textbf{Регулярная структура} : магнитооптическая ;\\
\textbf{Суперпериод} :  ;\\
\textbf{Резонансная структура} : ;\\
\textbf{Комбинированая структура} : ;\\
\textbf{Дуальная структура} : ;\\
\textbf{NICA} : (аббревиатура от \textit{aнгл.} Nuclotron-based Ion Collider fAcility) название коллайдера, расположенного в Объединенном Институте Ядерных Исследований (ОИЯИ) в г. Дубна, Россия;\\
\textbf{Nuclotron} : независимая установка для ускорения частиц на эксперимент BM@N и изучения управления поляризацией, в будущем – осуществлять функцию бустера поляризованных частиц в коллайдер NICA;\\
\textbf{МДМ} : магнитный дипольный момент элементарных частиц, обусловленный ;\\
\textbf{Аномальный магнитный момент} : отклонение величины магнитного момента элементарных частиц от предсказания релятивистской квантовомеханической теорией;\\
\textbf{ЭДМ} : электрический дипольный момент элементарных частиц, обусловленный неоднородностью распределения заряда;\\
\textbf{Поляризованные пучки} : ;\\
\textbf{Поляризация} : ;\\
\textbf{Декогеренция спина} : ;\\
\textbf{Прямые фильтры Вина} : ;\\
\textbf{Обводные каналы bypass} : ;\\
