\par Данная работа посвящена исследованию стабильности динамики тяжелых ионов, а также поляризованных пучков в ускорителях и накопителях. Также будут разобраны вопросы проектирования современных ускорительных установок.

\par Использование ускорительных установок для проведения разнообразных экспериментов представляет собой значительное преимущество. Эти установки позволяют достигать высоких энергий частиц, а также предельных точностей измерений. Такая практика применяется в крупных мировых ядерных центрах: CERN \cite{lhc:heavy_ions}, BNL \cite{rhic:design}, J-PARC \cite{j-park}. Последовательные программы экспериментов расписаны на годы и десятилетия вперед. Такие установки в первую очередь служат для фундаментальных исследований, но также поддерживают широкий спектр прикладных задач в таких областях, как медицина, материаловедение, и информационные технологии. Кроме того, ускорительные комплексы способствуют развитию необходимых технологий, что укрепляет научно-техническую базу и формирует долгосрочные положительные эффекты. Они также служат важной образовательной платформой, формируя программы для обучения молодых специалистов и предоставляя уникальные возможности для их профессионального роста. 

\par Ускорительный комплекс NICA (Nuclotron-based Ion Collider fAсility) является современным передовым центром, формирующимся на базе ОИЯИ в городе Дубна, Россия \cite{nuclotron24}. Основной установкой комплекса является коллайдер, в котором предусмотрены два места встречи пучков, где расположены два детектора: MPD (Multi-Purpose Detector) \cite{MPD} и SPD (Spin Polarized Detector) \cite{Ladygin:SPD}. Каждый из этих детекторов предназначен для различных экспериментов. MPD-детектор будет использован для исследования кварк-глюонной плазмы, возникающей в результате столкновений тяжёлых ионов. SPD-детектор направлен на изучение поведения сталкивающихся поляризованных пучков протонов и дейтронов. Таким образом, структура коллайдера должна поддерживать ускорение как тяжёлых ионов, так и лёгких частиц. При этом требования к удержанию пучков для различных типов частиц существенно отличаются.

\par Основным требованием коллайдерных экспериментов, является достижение большого количества соударений, то есть высокого уровня светимости. Для исследования кварк-глюонной плазмы это требование должно быть на уровне $10^{27}$ $\text{см}^{-2}\cdot\text{c}^{-1}$. Такие светимости являются рекордными и для их достижения может потребоваться существенная настройка всех систем ускорителя, что может потребовать большого времени. При ускорение тяжелых ионов высокая зарядность и интенсивность пучка вызывает серьёзные ограничения на параметры сгустка из-за внутрипучкового рассеяния (ВПР). Для преодоления этих проблем, спроектированная структура должна обладать высоким временем ВПР, а также содержать специальные установки стохастического и электронного охлаждения. Стохастическое охлаждение также в существенной степени зависит от конкретной магнитооптики установки и может быть оптимизировано для компенсации эффектов ВПР. Электронное охлаждение применяется на небольших энергиях сгустка и способно охладить пучок на начальных этапах ускорения.

\par Для изучения спиновой структуры протонов и дейтронов на детекторе SPD со светимостью $10^{32}$ $\text{см}^{-2}\cdot\text{c}^{-1}$ необходима подготовка и ускорение поляризованных пучков, что представляет особый интерес. Поляризация пучка является дополнительной степенью свободы и может привнести дополнительную информацию, в том числе в коллайдерные эксперименты. В этом случае определенные сечения рассеяния приобретают зависимость от поляризации сталкивающихся сгустков. Поскольку соотношение заряда к массе для протона отличается по сравнению с тяжелыми ионами почти в два раза, то максимальная энергия эксперимента кратно увеличивается. Но в существующей магнитооптике, оптимальной для тяжелоионного эксперимента подобрано значение критической энергии таким образом, что столкновение происходит до критического значения и никаких проблем по её преодолению не возникает. Стоит отметить, что критическая энергия является важным параметром ускорительный установки и при проектировании дизайна структуры этому вопросу уделяется особое внимание. Таким образом, для протонов прохождение критической энергии становится важным параметром, ограничивающем параметры сгустка и требует принятия дополнительных мер по её преодолению.

\par При ускорении пучка относительно длительное нахождение вблизи критической энергии или её пересечение существенно влияет на динамику пучка и его стабильность. Нарушается адиабатичность продольного движения, существенными становятся нелинейные эффекты, затухание Ландау оказывается неспособно подавить возникающие возмущения, а наличие пространственного заряда и других импедансов оказывает влияние на развитие продольной микроволновой неустойчивости, нестабильности отрицательной массы и поперечной голова-хвост (head-tail) \cite{ng}, \cite{lee}. В случае малых интенсивностей критическая энергия не оказывает значительного влияния на параметры сгустка. Проблема прохождения критической энергии является характерной для интенсивных сгустков с количеством частиц порядка $10^{10}-10^{12}$. Высокая интенсивность в коллайдерных экспериментах, обусловлена требованиями по достижению высокого значения светимости. Однако, любое увеличение эмиттанса пучка приведет к снижению конечной светимости эксперимента. Влияние всех приведенных эффектов, ограничено временем нахождения вблизи критической энергией, в этой связи используются методы быстрого пересечения.

\par Классическим методом преодоления является процедура скачка критической энергии \cite{risselada:jump}. При этом изменяются параметры ускорителя для внесения соответствующего возмущения и резкого кратковременного скачка критической энергии в момент близости энергии сгустка к критическому значению. После скачка, параметры установки возвращаются к исходному значению до скачка с поправкой на увеличившуюся энергию пучка. Однако, сложностью является непосредственное создание скачка с заданной величиной и темпом, что не всегда легко реализуемо. Возможный способ создания скачка критической энергии, состоит в изменении количества бетатронных колебаний или набега фазы, поскольку в случае регулярной структуры справедлива пропорциональность $\gamma_{\text{tr}} \sim \nu_{x}$. Такой метод может быть реализован путем создания возмущения в квадруполях. Этот подход был применен в 1969 году на установке PS (Proton Synchrotron), CERN \cite{cern:q-jump}. Однако, его применение существенно ограничено возможностью сдвига рабочей частоты и тем самым устанавливает предел для величины скачка, а скорость изменения -- на темп скачка. Другой метод основан на кратковременном возмущении дисперсионной функции, путем установки специальных квадруполей обратных полярностей, расположенных через один период друг от друга.  Таким образом, происходит искажение дисперсии без сдвига частоты бетатронных колебаний. Такой метод реализован позднее в 1974 году, также в PS \cite{cern:new-jump} и позволяет достигнуть большей величины и темпа измерения критической энергии.

\par Всё же существуют способы поднятия критической энергии выше конечной энергии эксперимента, что существенно упрощает эксплуатацию установки, а также не требует дополнительной манипуляции со структурой и пучком, однако, предъявляет особые условия при проектировании установки. Так, создание отрицательного значения дисперсии на прямом участке может компенсировать ненулевую дисперсию в поворотных арках \cite{pi_trans}, \cite{wo_trans}. Недостатком данного метода является его особенность -- отсутствие бездисперсионного участка, что усложняет проектирование транспортных каналов для инжекции или экстракции пучка, а также в случае коллайдера наличие дисперсии в точке столкновения, что сказывается на светимости.

\par Альтернативным способом, который применяется для того чтобы избегать потери стабильности, является создание или модификация структуры с заведомо большим значением критической энергии. Такая структура носит название 'резонансной' \cite{senichev:resonant}, \cite{senichev:construction}, впервые была предложена при проектировании каонной фабрики \cite{kaon_tr} и других установках мирового уровня Neutrino Factory в CERN \cite{neutrino_tr} и было реализовано в J-PARC для главного кольца \cite{JHP_tr}, \cite{J-PARK_tr}. Принципиальным отличием от регулярной структуры является обеспечение резонансного условия для количества суперпериодов и частоты бетатронных колебаний в горизонтальной плоскости. Однако, это справедливо только для не полностью регулярных структур, а содержащих регулярную модуляцию градиента квадруполей или кривизны орбиты. В таком случае, происходит изменение оптических функций ускорителя и варьирование критической энергии выше энергии эксперимента, в том числе до комплексных значений, полностью убирая зависимость установки от дополнительных процедур преодоления. Такой подход позволяет получить нулевую дисперсию на прямых участках в силу подавления дисперсии на поворотных арках, путём выбора целого числа бетатронных колебаний и создания ахромата первого порядка. Кроме того, в подобной структуре может быть легко реализован и ахромат второго порядка расстановкой секступолей через один суперпериод. Подобный подход способствует достижению достаточного значения динамической апертуры.

\par Непосредственное ускорение поляризованных пучков протонов является отдельным большим направлением исследований, поскольку требует решения проблема пересечения спиновых резонансов \cite{spin_res}. Стоит отметить, что спин является квантовой величиной, но в силу теоремы Эренфеста для любой квантовой величины может быть записано уравнение в квази-классическом приближении для ансамбля частиц. Поведение спина частицы в ансамбле описывается уравнением Т-БМТ \cite{TBMT}. Проекция спинов частиц на заданную ось и определяет поляризацию пучка. При ускорении до конечных энергий интерес представляет управление поляризацией пучка \cite{ST_Filatov}.

\par Особо сложным является создание когерентных пучков, а не просто поляризованных, что является передовым направлением исследований. В этом случае, сохраняется поляризация вдоль конкретной оси, а также спины частиц прецессируют с одинаковой частотой. Тогда появляется возможность исследовать также электрический дипольный момент (ЭДМ) элементарных частиц. Данная величина характеризует асимметрию распределения заряда частицы. Наличие ЭДМ объясняется тем, что он нарушает CP-симметрию, последнее было предсказано Сахаровым как одно из условий бариогенеза на ранних этапах вселенной \cite{sakharov}. Для накопления малой величины ЭДМ необходимо долгое удержание пучка с последующим анализом рассеяния на мишени на поляриметре. При этом влияние магнитного дипольного момента (МДМ) должно быть подавлено. Такая техника впервые была предложена в БНЛ (Брукхейвенская Национальная Лаборатория) и имеет название 'замороженный' спин \cite{Farley:edm}. Позднее, была предложена концепция 'квази-замороженного' спина \cite{QFS}, в которой происходит пространственное разделение полей и интегральное подавление МДМ-компоненты за полный оборот по кольцу.

\par Представленные исследования исходят из возможности изучения в комплексе NICA-Nuclotron. Построенный ускорительный комплекс является проектом мегасайнс и оборудован передовой материально-технической базой, отвечающей мировым тенденциях в ускорительной технике. Основными функционирующими установками помимо уже упомянутого коллайдера NICA являются бустер тяжелых ионов Booster, а также синхротрон Nuclotron.

\par В коллайдере NICA для реализации концепции квази-замороженного спина необходима установка соответствующего оборудования. Для реализации накопительного кольца из структуры коллайдера, необходима модернизация с созданием обходных каналов bypass. Таким образом, на полученных прямолинейных участках могут быть расположены прямые фильтры Вина, выполняющий функцию компенсации МДМ-компоненты в скрещенных магнитных и электрических полях, не возмущающие орбиту в силу равенства нулю силы Лоренца.

\par Nuclotron является бустером поляризованных частиц в коллайдер, однако, требующим модернизации. Соответствующей концепт модернизации рассмотрен с точки зрения использования Nuclotron в тесной связке с коллайдером NICA.
Использование Nuclotron для полноценных спиновых экспериментов делает эту машину столь же интересной, сколько и отдельные программы на коллайдере. Кроме того, особенности магнитооптики Nuclotron открывают возможность измерение ЭДМ не только дейтрона, но и протона, однако, при несколько меньшей энергии. На текущий день измерений ЭДМ как дейтрона, так и протона не было осуществлено и представляется передним краем физического эксперимента на ускорительной установке.

\par Ещё одним направлением исследований в рамках формирующейся программы спиновой физике является исследование аксиона. В этом случае резонансным методом между частотой спиной прецессии и частотой осциллирующего скалярного аксионного поля может быть получена масса аксиона или установлены соответствующие ограничения. Для этого ускоритель будет использован в роли зондирующей антенны по частоте прецессии спина \cite{Axion_Nikolaev}.

~\\
\par {\actuality} Исследования направлены на формирование полноценной физической программы по изучению динамики поляризованных пучков в комплексе Nuclotron-NICA. Применение изложенных в работе подходов возможно и на других похожих установках без потери общности.
~\\
\par {\aim} данной диссертации является изучение особенностей динамики многозарядных тяжёлоионных и лёгких поляризованных пучков для проведения коллайдерных экспериментов в дуальной структуре, а также исследования электрического дипольного момента с использованием квази-замороженной концепции.
Для достижения поставленной цели необходимо было 
решить следующие {\tasks}:

\begin{enumerate}[beginpenalty=10000] % https://tex.stackexchange.com/a/476052/104425
  \item Расчёт времени внутрипучкового рассеяния для тяжелых ионов;
  \item Оценка влияния методов с стохастического охлаждения пучка на время жизни;
  \item Моделирование магнитооптики с модулированной дисперсионной функцией;
  \item Проведение численного моделирования продольной динамики частиц с учетом высших порядков коэффициента уплотнения орбиты в высокочастотых резонаторах гармонического и барьерного типа;
  \item Обеспечение стабильности пучка с точки зрения динамической апертуры при процедуре скачка критической энергии, подавление хроматичности, компенсация нелинейных эффектов;
  \item Сохранение поляризации пучка при совершении процедуры скачка критической энергии;
  \item Изучение концепции «квази-замороженного» спина с целью создания установки для исследования ЭДМ дейтрона и протона;
  \item Спин-орбитальное моделирование в магнитном кольце с дополнительными элементами со скрещенными магнитными и электрическими полями;
\end{enumerate}
~\\
\par {\novelty}
\begin{enumerate}[beginpenalty=10000] % https://tex.stackexchange.com/a/476052/104425
    \item	Исследованы закономерности динамики многозарядных тяжёлых ионов и лёгких поляризованных частиц в дуальной магнитооптической структуре с учётом различий во внутрипучковом рассеянии и влияния критической энергии на устойчивость пучка;
  \item 	Предложен метод резонансной модуляции дисперсионной функции с применением дополнительного семейства квадрупольных линз, что позволило повысить критическую энергию и стабильность пучка в режиме ускорения лёгких частиц;
  \item	Приведены способы подавления дисперсии на краях поворотных арок в отсутствии регулярности, а также способы подавления нелинейных эффектов;
  \item 	Выполнено численное моделирование прохождения критической энергии с учётом высших порядков зависимости от импульсного разброса, а также влияния импедансов, что позволило количественно оценить влияние данных факторов на сохранение пучка;
  \item	Исследована продольная динамика поляризованного пучка при нахождении вблизи и прохождении критической энергии методом скачка в гармоническом и барьерном ВЧ, что позволило количественно оценить стабильность пучка в различных режимах ускорения;
  \item	Предложено применение метода фильтров Вина для сохранения направления поляризации в пучках, что расширяет возможности по исследованию электрического дипольного момента и аксионоподобных частиц;
  \item	Рассмотрены вариации изучения ЭДМ дейтрона и протона в много-периодичных структурах с использованием электростатических дефлекторов или фильтров Вина;
\end{enumerate}
~\\
\par {\influence}:
\par Влияние внутрипучкового рассеяния, а также критической энергии существенно влияет на динамику пучка. Дуальность структуры указывает на возможность её эффективного решения обоих эффектов и использования сразу для двух фундаментально значимых исследований. Во-первых, для изучения кварк-глюонной плазмы в коллайдерных экспериментах с тяжелыми ионами. Во-вторых, для исследования легких поляризованных пучков в симметричных и асимметричных коллайдерных столкновениях. Данных подход, может быть осуществлен в коллайдере NICA с использованием MPD и SPD детекторов.

\par Исследование динамики пучка вблизи критической энергии показывает необходимость её преодоления, а также  способствует определению оптимальных параметров скачка критической энергии, а также его влияние на динамику сгустка.

\par При искажении дисперсионной функции возникает необходимость её подавления на краях поворотных арок для обеспечения бездисперсионных прямых промежутков. Любое нарушение регулярности приводит к необходимости применения дополнительных усилий по подавлению дисперсии, а также коррекции возникающей нелинейности.

\par Наличие ЭДМ заряженных частиц может быть установлено только с использованием ускорительных установок в качестве накопительного кольца. Кроме того, реализация условия квази-замороженности спина может быть осуществлена без нарушения основной функции установки. Создание обводных каналов bypass позволит избежать точек встречи, также расположить прямые фильтры Вина независимо от оборудования, используемого для тяжело-ионного эксперимента. В конечном счёте, это позволит использовать NICA в режиме накопительного кольца. Такой подход может быть использован и для Nuclotron, сохраняется функция бустера для поляризованного пучка в коллайдер, а также возможно проведение независимых экспериментов по исследованию ЭДМ как протона, так и дейтрона и поиску аксиона. Такие исследования является отдельной частью программы спиновой физики, которая формируется на установке NICA-Nuclotron.

% {\progress}
% Этот раздел должен быть отдельным структурным элементом по
% ГОСТ, но он, как правило, включается в описание актуальности
% темы. Нужен он отдельным структурынм элемементом или нет ---
% смотрите другие диссертации вашего совета, скорее всего не нужен.
~\\
\par {\methods} Основными методами исследования являются математическое и компьютерное моделирование, численный эксперимент. Для исследования были использованы программы для расчёта поперечной динамики: MAD-X \cite{madx}, OPTIM \cite{optim}, BMAD \cite{bmad}, продольной динамики: BLonD \cite{blond}; спин-орбитальной динамики: COSY Infinity \cite{cosy}.
~\\
%второй вариант
\begin{comment}
\par {\defpositions}
\begin{enumerate}[beginpenalty=10000] % https://tex.stackexchange.com/a/476052/104425
  \item 	Принципы построения дуальной магнитооптической структуры с оптимизированным временем жизни пучка в регулярной структуре для многозарядных тяжелых ионов и варьированной критической энергией в резонансной структуре для легких ядер; \cite{Kolokolchikov:2025_dual}, \cite{Syresin:2021_polar}
  \item	Результаты, полученные в эксперименте на У-70 и в методе численного моделирования динамики продольного движения вблизи критической энергии с учётом влияния высших порядков зависимости от разброса по импульсу и с учетом импеданса; \cite{Kolokolchikov:2025_U70}, \cite{Kolokolchikov:2025_jump}
  \item	Результаты исследования продольной динамики поляризованного пучка для процедуры скачка критической энергии в гармоническом и барьерном ВЧ, оценка влияния продольной микроволновой неустойчивости; \cite{Kolokolchikov:2024_bb_rupac}, \cite{Kolokolchikov:2023_bb_IPAC}, \cite{Kolokolchikov:2024_bb_dspin}
  \item	Метод подавления дисперсии и влияния нелинейных эффектов, из-за нарушения периодичности за счет введения missing magnet на краях поворотных арок, для создания резонансной магнитооптической структуры; \cite{Kolokolchikov:2021trans}, \cite{Kolokolchikov:2023_pecular}
  \item	Модернизированная структура с квази-замороженным спином для исследования ЭДМ дейтронов и протонов и возможностью совместного использования Нуклотрона в качестве бустера поляризованных частиц для коллайдера; \cite{Senichev:2023_QFS}, \cite{Senichev:2023_nuclotron}, \cite{Kolokolchikov:2025_nuclotron}
  \item	Метод обводных каналов bypass для независимого исследования ЭДМ в кольце коллайдера;\cite{Kolokolchikov:2023_bypass}, \cite{Kolokolchikov:2023_bypass_IPAC}, \cite{Senichev:2024_nica_edm}, \cite{Kolokolchikov:2023_sc}, \cite{Kolokolchikov:2023_sc_IPAC}
\end{enumerate}
\end{comment}

%третий вариант
\begin{comment}
\par {\defpositions}
\begin{enumerate}[beginpenalty=10000] % https://tex.stackexchange.com/a/476052/104425
  \item 	Изучение внутрипучкового рассеяния и стохастического охлаждения для оптимизации времени жизни пучка в регулярной структуре для многозарядных тяжелых ионов и варьированной критической энергией в резонансной структуре для легких ядер с целью реализации дуальности ускорительной установки; \cite{Kolokolchikov:2025_dual}, \cite{Syresin:2021_polar}
  \item	Результаты, полученные в эксперименте на У-70 и в методе численного моделирования динамики продольного движения вблизи критической энергии с учётом влияния высших порядков зависимости от разброса по импульсу и с учетом импеданса; \cite{Kolokolchikov:2025_U70}, \cite{Kolokolchikov:2025_jump}
  \item	Результаты исследования продольной динамики поляризованного пучка для процедуры скачка критической энергии в гармоническом и барьерном ВЧ, оценка влияния продольной микроволновой неустойчивости; \cite{Kolokolchikov:2024_bb_rupac}, \cite{Kolokolchikov:2023_bb_IPAC}, \cite{Kolokolchikov:2024_bb_dspin}
  \item	Метод подавления дисперсии и влияния нелинейных эффектов в резонансной магнитооптической структуре из-за нарушения периодичности по дисперсии за счет missing magnet на краях поворотных арок; \cite{Kolokolchikov:2021trans}, \cite{Kolokolchikov:2023_pecular}
  \item	Модернизированная структура с квази-замороженным спином для исследования ЭДМ дейтронов и протонов и возможностью совместного использования Нуклотрона в качестве бустера поляризованных частиц для коллайдера; \cite{Senichev:2023_QFS}, \cite{Senichev:2023_nuclotron}, \cite{Kolokolchikov:2025_nuclotron}
  \item	Метод введения обводных каналов в кольцо синхротрона для создания независимой установки с возможностью проведения прецизионных экспериментов, в том числе изучения ЭДМ элементарных частиц;\cite{Kolokolchikov:2023_bypass}, \cite{Kolokolchikov:2023_bypass_IPAC}, \cite{Senichev:2024_nica_edm}, \cite{Kolokolchikov:2023_sc}, \cite{Kolokolchikov:2023_sc_IPAC}
\end{enumerate}
\end{comment}

\par {\defpositions}
\begin{enumerate}[beginpenalty=10000] % https://tex.stackexchange.com/a/476052/104425
  \item 	Основные свойства дуальной магнитооптической структуры для легких ядер и тяжелых частиц с учетом различия внутрипучкового рассеяния. Время жизни пучка в дуальной структуре с учетом вариации коэффициента проскальзывания в разных арках; \cite{Kolokolchikov:2025_dual}, \cite{Syresin:2021_polar}
  \item	Учет влияния высших порядков разброса по импульсам и моделей продольных импедансов в численном моделировании движения в окрестности критической энергии и сравнение с экспериментальными результатами, полученными на У-70; \cite{Kolokolchikov:2025_U70}, \cite{Kolokolchikov:2025_jump}
  \item	Результаты математического моделирования процесса прохождения ансамбля частиц через критическую энергию с различной скоростью и при различной форме ускоряющего потенциала с учетом ограничений по продольной микроволновой неустойчивости; \cite{Kolokolchikov:2024_bb_rupac}, \cite{Kolokolchikov:2023_bb_IPAC}, \cite{Kolokolchikov:2024_bb_dspin}
  \item	Резонансная модуляция дисперсионной функции, как метод вариации критической энергии. Результаты оптимизации дисперсионной функции при наличии отсутствующих магнитов (missing magnets); \cite{Kolokolchikov:2021trans}, \cite{Kolokolchikov:2023_pecular}
  \item	Особенности поляризованных пучков, используемых для исследования электрического дипольного момента в структурах с квази-замороженным спином на примере модернизированной структуры Нуклотрона; \cite{Senichev:2023_QFS}, \cite{Senichev:2023_nuclotron}, \cite{Kolokolchikov:2025_nuclotron}
  \item	Метод фильтров Вина для сохранении направления поляризации  на основе введения обводных каналов в структуре с квази-замороженным спином для выделения ЭДМ сигнала в поляризованном пучке;\cite{Kolokolchikov:2023_bypass}, \cite{Kolokolchikov:2023_bypass_IPAC}, \cite{Senichev:2024_nica_edm}, \cite{Kolokolchikov:2023_sc}, \cite{Kolokolchikov:2023_sc_IPAC}
\end{enumerate}

~\\
\par {\reliability} полученных результатов подтверждается согласованием аналитических вычислений с результатами численных экспериментов. Результаты находятся в соответствии с результатами, полученными другими авторами.
~\\
\par {\probation}
Основные результаты работы были представлены докладывались~на российских и международных конференциях, а также рабочих встречах: 
\begin{itemize}
\item Workshop “Polarized beam in NICA” в 2022 г.;
\item Молодежная конференция по теоретической и экспериментальной физике МКТЭФ-2020. Москва, Россия;
\item 63, 65, 66-ая Всероссийская научная конференция МФТИ в 2020, 2023, 2024 гг. г. Долгопрудный,
Россия;
\item XXVII и XXVIII Всероссийская конференции по ускорителям заряженных частиц RuPAC'21, RuPAC'23. Алушта; Новосибирск, Россия.
\item VII, VIII, IX и X Международная конференция Лазерные и Плазменные технологии ЛаПлаз'21, ЛаПлаз'22, ЛаПлаз'23, ЛаПлаз'24, ЛаПлас'25. Москва, Россия;
\item XIII, XIV, XVI международная конференция по ускорителям заряженных частиц IPAC'22 IPAC'23, IPAC'25. Бангкок, Тайланд; Венеция, Италия; Тайпей, Тайвань;
\item XIX Международная конференции по спиновой физике высоких энергий DSPIN'23. Дубна, Россия;
\item XI-я Международная конференция по ядерной физике в накопительных кольцах STORI’24. Хуэйчжоу, провинция Гуандун, Китай;
\end{itemize}
~\\
\par {\contribution} Все результаты, выносимые на защиту, получены автором лично, либо при его непосредственном участии. Содержание диссертации и выносимые на защиту основные положения отражают личный вклад автора в опубликованные работы. Результаты по подготовке и проведению эксперимента на ускорителе У-70 получены в соавторстве с сотрудниками ИЯИ РАН и ИФВЭ. Подготовка к публикации полученных результатов проводилась совместно с соавторами.
~\\
\par \ifnumequal{\value{bibliosel}}{0}
{%%% Встроенная реализация с загрузкой файла через движок bibtex8. (При желании, внутри можно использовать обычные ссылки, наподобие `\cite{vakbib1,vakbib2}`).
 {\publications} Основные результаты по теме диссертации изложены
    в~XX~печатных изданиях,
    X из которых изданы в журналах, рекомендованных ВАК,
    X "--- в тезисах докладов.
}%
{%%% Реализация пакетом biblatex через движок biber
    \begin{refsection}[bl-author, bl-registered]
        % Это refsection=1.
        % Процитированные здесь работы:
        %  * подсчитываются, для автоматического составления фразы "Основные результаты ..."
        %  * попадают в авторскую библиографию, при usefootcite==0 и стиле `\insertbiblioauthor` или `\insertbiblioauthorgrouped`
        %  * нумеруются там в зависимости от порядка команд `\printbibliography` в этом разделе.
        %  * при использовании `\insertbiblioauthorgrouped`, порядок команд `\printbibliography` в нём должен быть тем же (см. biblio/biblatex.tex)
        %
        % Невидимый библиографический список для подсчёта количества публикаций:
        \printbibliography[heading=nobibheading, section=1, env=countauthorvak,          keyword=biblioauthorvak]%
        \printbibliography[heading=nobibheading, section=1, env=countauthorwos,          keyword=biblioauthorwos]%
        \printbibliography[heading=nobibheading, section=1, env=countauthorscopus,       keyword=biblioauthorscopus]%
        \printbibliography[heading=nobibheading, section=1, env=countauthorconf,         keyword=biblioauthorconf]%
        \printbibliography[heading=nobibheading, section=1, env=countauthorother,        keyword=biblioauthorother]%
        \printbibliography[heading=nobibheading, section=1, env=countregistered,         keyword=biblioregistered]%
        \printbibliography[heading=nobibheading, section=1, env=countauthorpatent,       keyword=biblioauthorpatent]%
        \printbibliography[heading=nobibheading, section=1, env=countauthorprogram,      keyword=biblioauthorprogram]%
        \printbibliography[heading=nobibheading, section=1, env=countauthor,             keyword=biblioauthor]%
        \printbibliography[heading=nobibheading, section=1, env=countauthorvakscopuswos, filter=vakscopuswos]%
        \printbibliography[heading=nobibheading, section=1, env=countauthorscopuswos,    filter=scopuswos]%
        %
        \nocite{*}%
        %
        {\publications} Основные результаты по теме диссертации изложены в~\arabic{citeauthor}~печатных изданиях,
        \arabic{citeauthorvak} из которых изданы в журналах, рекомендованных ВАК\sloppy%
        \ifnum \value{citeauthorscopuswos}>0%
            , \arabic{citeauthorscopuswos} "--- в~периодических научных журналах, индексируемых Web of~Science и Scopus\sloppy%
        \fi%
        \ifnum \value{citeauthorconf}>0%
            , \arabic{citeauthorconf} "--- в~тезисах докладов.
        \else%
            .
        \fi%
        \ifnum \value{citeregistered}=1%
            \ifnum \value{citeauthorpatent}=1%
                Зарегистрирован \arabic{citeauthorpatent} патент.
            \fi%
            \ifnum \value{citeauthorprogram}=1%
                Зарегистрирована \arabic{citeauthorprogram} программа для ЭВМ.
            \fi%
        \fi%
        \ifnum \value{citeregistered}>1%
            Зарегистрированы\ %
            \ifnum \value{citeauthorpatent}>0%
            \formbytotal{citeauthorpatent}{патент}{}{а}{}\sloppy%
            \ifnum \value{citeauthorprogram}=0 . \else \ и~\fi%
            \fi%
            \ifnum \value{citeauthorprogram}>0%
            \formbytotal{citeauthorprogram}{программ}{а}{ы}{} для ЭВМ.
            \fi%
        \fi%
        % К публикациям, в которых излагаются основные научные результаты диссертации на соискание учёной
        % степени, в рецензируемых изданиях приравниваются патенты на изобретения, патенты (свидетельства) на
        % полезную модель, патенты на промышленный образец, патенты на селекционные достижения, свидетельства
        % на программу для электронных вычислительных машин, базу данных, топологию интегральных микросхем,
        % зарегистрированные в установленном порядке.(в ред. Постановления Правительства РФ от 21.04.2016 N 335)
    \end{refsection}%
    \begin{refsection}[bl-author, bl-registered]
        % Это refsection=2.
        % Процитированные здесь работы:
        %  * попадают в авторскую библиографию, при usefootcite==0 и стиле `\insertbiblioauthorimportant`.
        %  * ни на что не влияют в противном случае
        \nocite{vakbib2}%vak
        \nocite{patbib1}%patent
        \nocite{progbib1}%program
        \nocite{bib1}%other
        \nocite{confbib1}%conf
    \end{refsection}%
        %
        % Всё, что вне этих двух refsection, это refsection=0,
        %  * для диссертации - это нормальные ссылки, попадающие в обычную библиографию
        %  * для автореферата:
        %     * при usefootcite==0, ссылка корректно сработает только для источника из `external.bib`. Для своих работ --- напечатает "[0]" (и даже Warning не вылезет).
        %     * при usefootcite==1, ссылка сработает нормально. В авторской библиографии будут только процитированные в refsection=0 работы.
}


