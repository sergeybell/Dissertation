\par Данная работа посвящена исследованию динамики поляризованных пучков в ускорителях и накопителях.

\par Поведение спина частицы в ансамбле описывается уравнением Т-БМТ. 
Проекция спинов частиц на заданную ось определяет поляризацию пучка.

\par Поляризованный пучок представляет большой интерес в коллайдерных исследованиях, при котором сечение рассеяния зависит от поляризации пучка.
Также, долгое сохранение поляризации пучка может быть использовано и в накопителях. 
Более тонким, являются не просто поляризованные пучки, а также когерентные. В этом случае, пучок становится не просто поляризованным, но и спины частиц прецессируют с одинаковой частотой. 

\par Представленные исследования исходят из возможности изучения в комплексе NICA-Nuclotron. Построенный ускорительный комплекс является проектом мегасайнс и оборудован передовой материально-технической базой, отвечающей мировым тенденциях в ускорительной технике. Основными функционирующими установками являются: коллайдер NICA, бустер тяжелых ионов Booster, а также Nuclotron. 

\par Коллайдер NICA, имеет 2 места встречи, в которых расположены детектора: MPD(Multi-Purpose Detector) и SPD(Spin Polarized Detector). Каждый из них предназначен для разных экспериментов. MPD-детектор – будет использован для исследования кварк-гюонной плазмы, возникающей в результате столкновений тяжелых ионов золота. SPD-детектор направлен на изучение поведения сталкивающихся поляризованных пучков протонов и дейтронов. Таким образом, структура коллайдера должна быть использована как для ускорения пучков тяжелых ионов, так и легких. При этом требования, предъявляемые для удержания пучка для разного сорта частиц, отличаются. При ускорение тяжелых ионов, из-за внутрипучкового рассеяние, 

\par Подготовка и ускорение поляризованных пучков для экспериментов на детекторе SPD представляет особый интерес и будет рассмотрено в этой работе. 

\par Nuclotron является бустером, однако, требующем модернизации. Соответствующей концепт модернизации рассмотрен с точки зрения использования Nuclotron в тесной связке с коллайдером NICA. 

{\actuality} Исследования направлены на формирование полноценной физической программы по исследованию спиновой динамике в комплексе NICA-Nuclotron.

{\aim} данной диссертации является изучение 
особенностей динамики поляризованного пучка в ускорительном 
комплексе NICA-Nuclotron с учетом возможной модернизации 
магнитооптической структуры комплекса для исследования 
электрического дипольного момента.
Для достижения поставленной цели необходимо было 
решить следующие {\tasks}:

\begin{enumerate}[beginpenalty=10000] % https://tex.stackexchange.com/a/476052/104425
  \item Моделирование магнитооптики с модулированной дисперсионной функцией;
  \item Расчёт времени внутрипучкового рассеяния;
  \item Проведение численного моделирования продольной динамики частиц с учетом высших порядков коэффициента скольжения в ВЧ гармонического и барьерного типа;
  \item Обеспечение стабильности пучка с точки зрения динамической апертуры при процедуре скачка критической энергии, подавление хроматичности, компенсация нелинейных эффектов;
  \item Сохранение поляризации пучка при совершении процедуры скачка критической энергии;
  \item Проектирование кольцевого ускорителя с возможностью применения метода «квази-замороженного спина»;
  \item Спин-орбитальное моделирование в магнитном кольце с дополнительными элементами со скрещенными магнитными и электрическими полями; 
\end{enumerate}

\ifsynopsis
Этот абзац появляется только в~автореферате.
Для формирования блоков, которые будут обрабатываться только в~автореферате,
заведена проверка условия \verb!\!\verb!ifsynopsis!.
Значение условия задаётся в~основном файле документа (\verb!synopsis.tex! для
автореферата).
\else
%Этот абзац появляется только в~диссертации.
%Через проверку условия \verb!\!\verb!ifsynopsis!, задаваемого в~основном файле
%документа (\verb!dissertation.tex! для диссертации), можно сделать новую
%команду, обеспечивающую появление цитаты в~диссертации, но~не~в~автореферате.
\fi

% {\progress}
% Этот раздел должен быть отдельным структурным элементом по
% ГОСТ, но он, как правило, включается в описание актуальности
% темы. Нужен он отдельным структурынм элемементом или нет ---
% смотрите другие диссертации вашего совета, скорее всего не нужен.

{\methods}. Основными методами исследования являются математическое и компьютерное моделирование, численный эксперимент. Для исследования поперечной динамики: MAD-X, OPTIM, продольной динамики: BLonD; спин-орбитальной динамики: COSY Infinity.

{\novelty}
\begin{enumerate}[beginpenalty=10000] % https://tex.stackexchange.com/a/476052/104425
  \item	Исследована возможность проектирования дуальной магнитооптической структуры с возможностью преодоления критической энергии методом вариации критической энергии;
  \item	Исследована динамика поляризованного пучка при прохождении критической энергии скачком в ВЧ различных типов;
  \item	Разработка альтернативных прямых секций, путем создания обходных каналов ByPass;
  \item	Изучена реализации метода «Квази-Замороженного Спина» с установленными фильтрами Вина на альтернативных прямых секциях для возможности изучения ЭДМ дейтронов в накопительном кольце NICA;
  \item	Модернизация кольца канала Nuclotron с учётом возможности создания режима «Квази-Замороженного Спина» и изучения ЭДМ протона;
  \item	Изучение спин-орбитальной динамики в предложенных структурах.
\end{enumerate}

{\influence} работы состоит в том, что рассмотрены общие принципы проектирования магнитооптических структур.

{\defpositions}
\begin{enumerate}[beginpenalty=10000] % https://tex.stackexchange.com/a/476052/104425
  \item Принципы построения дуальной магнитооптической структуры для тяжелых ионов и протонов (дейтронов? Может сказать «легких ядер»);
  \item	Методы позволяющие минимизировать влияние внутрипучкового рассеяния (IBS) для обеспечения достаточного времени жизни пучка;
  \item	Методы вариации критической энергии в резонансных магнитооптических структурах путем суперпериодической модуляции дисперсионной функции;
  \item	Принципы построения регулярной структуры с различными методами подавления дисперсии;
  \item	Результаты исследования продольной динамики поляризованного пучка для процедуры скачка критической энергии;
  \item	Методы подавления натуральной хроматичности и компенсации нелинейных эффектов секступолями;
  \item Принципы проектирования оптимальных магнитооптических структур для изучения электрического дипольного момента легких ядер в режиме «Квази-Замороженного Спина»;
  \item	Реализована адаптация существующей структуры методом создания альтернативных обходных прямых секций ByPass;
  \item	Результаты спин-орбитального моделирования динамики поляризованного пучка в спроектированных структурах.
\end{enumerate}

{\reliability} полученных результатов подтверждается согласованием аналитических вычислений с результатами численных экспериментов. Результаты находятся в соответствии с результатами, полученными другими авторами.

{\probation}
Основные результаты работы докладывались~на конференциях: ИТЭФ’20, МФТИ’20, RuPAC'21, ЛаПлаз'21, ЛаПлаз'22, IPAC'22, ЛаПлаз'23, IPAC'23, DSPIN RuPAC'23, IPAC'25.

{\contribution}  Все результаты, выносимые на защиту, получены автором лично. Содержание диссертации и выносимые на защиту основные положения отражают личный вклад автора в опубликованные работы. Подготовка к публикации полученных результатов проводилась совместно с соавторами.

\ifnumequal{\value{bibliosel}}{0}
{%%% Встроенная реализация с загрузкой файла через движок bibtex8. (При желании, внутри можно использовать обычные ссылки, наподобие `\cite{vakbib1,vakbib2}`).
    {\publications} Основные результаты по теме диссертации изложены
    в~XX~печатных изданиях,
    X из которых изданы в журналах, рекомендованных ВАК,
    X "--- в тезисах докладов.
}%
{%%% Реализация пакетом biblatex через движок biber
    \begin{refsection}[bl-author, bl-registered]
        % Это refsection=1.
        % Процитированные здесь работы:
        %  * подсчитываются, для автоматического составления фразы "Основные результаты ..."
        %  * попадают в авторскую библиографию, при usefootcite==0 и стиле `\insertbiblioauthor` или `\insertbiblioauthorgrouped`
        %  * нумеруются там в зависимости от порядка команд `\printbibliography` в этом разделе.
        %  * при использовании `\insertbiblioauthorgrouped`, порядок команд `\printbibliography` в нём должен быть тем же (см. biblio/biblatex.tex)
        %
        % Невидимый библиографический список для подсчёта количества публикаций:
        \printbibliography[heading=nobibheading, section=1, env=countauthorvak,          keyword=biblioauthorvak]%
        \printbibliography[heading=nobibheading, section=1, env=countauthorwos,          keyword=biblioauthorwos]%
        \printbibliography[heading=nobibheading, section=1, env=countauthorscopus,       keyword=biblioauthorscopus]%
        \printbibliography[heading=nobibheading, section=1, env=countauthorconf,         keyword=biblioauthorconf]%
        \printbibliography[heading=nobibheading, section=1, env=countauthorother,        keyword=biblioauthorother]%
        \printbibliography[heading=nobibheading, section=1, env=countregistered,         keyword=biblioregistered]%
        \printbibliography[heading=nobibheading, section=1, env=countauthorpatent,       keyword=biblioauthorpatent]%
        \printbibliography[heading=nobibheading, section=1, env=countauthorprogram,      keyword=biblioauthorprogram]%
        \printbibliography[heading=nobibheading, section=1, env=countauthor,             keyword=biblioauthor]%
        \printbibliography[heading=nobibheading, section=1, env=countauthorvakscopuswos, filter=vakscopuswos]%
        \printbibliography[heading=nobibheading, section=1, env=countauthorscopuswos,    filter=scopuswos]%
        %
        \nocite{*}%
        %
        {\publications} Основные результаты по теме диссертации изложены в~\arabic{citeauthor}~печатных изданиях,
        \arabic{citeauthorvak} из которых изданы в журналах, рекомендованных ВАК\sloppy%
        \ifnum \value{citeauthorscopuswos}>0%
            , \arabic{citeauthorscopuswos} "--- в~периодических научных журналах, индексируемых Web of~Science и Scopus\sloppy%
        \fi%
        \ifnum \value{citeauthorconf}>0%
            , \arabic{citeauthorconf} "--- в~тезисах докладов.
        \else%
            .
        \fi%
        \ifnum \value{citeregistered}=1%
            \ifnum \value{citeauthorpatent}=1%
                Зарегистрирован \arabic{citeauthorpatent} патент.
            \fi%
            \ifnum \value{citeauthorprogram}=1%
                Зарегистрирована \arabic{citeauthorprogram} программа для ЭВМ.
            \fi%
        \fi%
        \ifnum \value{citeregistered}>1%
            Зарегистрированы\ %
            \ifnum \value{citeauthorpatent}>0%
            \formbytotal{citeauthorpatent}{патент}{}{а}{}\sloppy%
            \ifnum \value{citeauthorprogram}=0 . \else \ и~\fi%
            \fi%
            \ifnum \value{citeauthorprogram}>0%
            \formbytotal{citeauthorprogram}{программ}{а}{ы}{} для ЭВМ.
            \fi%
        \fi%
        % К публикациям, в которых излагаются основные научные результаты диссертации на соискание учёной
        % степени, в рецензируемых изданиях приравниваются патенты на изобретения, патенты (свидетельства) на
        % полезную модель, патенты на промышленный образец, патенты на селекционные достижения, свидетельства
        % на программу для электронных вычислительных машин, базу данных, топологию интегральных микросхем,
        % зарегистрированные в установленном порядке.(в ред. Постановления Правительства РФ от 21.04.2016 N 335)
    \end{refsection}%
    \begin{refsection}[bl-author, bl-registered]
        % Это refsection=2.
        % Процитированные здесь работы:
        %  * попадают в авторскую библиографию, при usefootcite==0 и стиле `\insertbiblioauthorimportant`.
        %  * ни на что не влияют в противном случае
        \nocite{vakbib2}%vak
        \nocite{patbib1}%patent
        \nocite{progbib1}%program
        \nocite{bib1}%other
        \nocite{confbib1}%conf
    \end{refsection}%
        %
        % Всё, что вне этих двух refsection, это refsection=0,
        %  * для диссертации - это нормальные ссылки, попадающие в обычную библиографию
        %  * для автореферата:
        %     * при usefootcite==0, ссылка корректно сработает только для источника из `external.bib`. Для своих работ --- напечатает "[0]" (и даже Warning не вылезет).
        %     * при usefootcite==1, ссылка сработает нормально. В авторской библиографии будут только процитированные в refsection=0 работы.
}


