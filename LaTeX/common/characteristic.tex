\par {\actuality} Данная работа посвящена исследованию поведения тяжелых ионов и поляризованных лёгких заряженных частиц в ускорительных и накопительных установках с целью изучения фундаментальных свойств материи. Представленные результаты направлены на формирование комплексной физической программы исследований, включающей вопросы по разрешению спинового кризиса и изучению электрического дипольного момента элементарных частиц.

\par	На сегодняшний день, механизм формирования и эволюции Вселенной остается загадкой. По текущим представлениям, на ранних этапах формирования Вселенной материя находилась в экстремально плотном состоянии, известной как кварк-глюонная плазма \autocite{phase_transition_universe}. Подобное состояние материи может наблюдаться в недрах нейтронных звезд \autocite{neutron_stars}, а также в результате столкновения тяжелых заряженных частиц. Подобные эксперименты могут осуществляться в рамках коллайдерных исследований с тяжелыми частицами и помогут в изучении фазовых переходов и критических явлений в сильновзаимодействующей ядерной материи при экстремальных барионных плотностях \autocite{quark_gluon}.

\par	Для достижения статистически значимых результатов любого коллайдерного эксперимента требуется набор достаточного количества статистических данных, что выражается в такой интегральной характеристике как светимость. Обеспечение её высокого уровня является ключевым требованием. Для исследования кварк-глюонной плазмы это требование находится на уровне порядка $10^{27}$~$\text{см}^{-2}\cdot\text{c}^{-1}$ \autocite{RHIC_luminosity_heavy}. Такие светимости являются рекордными и для их достижения может потребоваться существенная настройка всех систем ускорителя, что может потребовать большого времени. При ускорении тяжелых ионов высокая зарядность и интенсивность пучка вызывает серьёзные ограничения на параметры сгустка из-за внутрипучкового рассеяния (ВПР) \autocite{2016_IBS}. Для преодоления этих проблем, спроектированная структура должна обладать высоким временем ВПР, а также содержать специальные установки стохастического и электронного охлаждения для компенсации эффектов разогрева пучка.

\par	Другой нерешенной проблемой современной физики остается вопрос о распределении спина внутри протона, так называемый "спиновый кризис протона". В 1989 году коллаборацией EMC (European Muon Collaboration) \autocite{spin_crisis_1989} было показано, что вклад кварков в спин протона составляет лишь небольшую часть и по современным оценкам находится на уровне около 30$\%$ \autocite{quarks_overview_2022}. Исследования этого вопроса проводились при изучении в коллайдерных экспериментах c поляризованными пучками протонов и дейтронов в COSY-ANKE \autocite{COSY_ANKE} и SATURNE \autocite{SATURNE} при низких энергиях и протонов в RHIC при высоких \autocite{RHIC_2014}. 

\par	Для изучения спиновой структуры протонов и дейтронов необходима подготовка и ускорение поляризованных пучков для достижения светимости порядка $10^{32}$ $\text{см}^{-2}\cdot\text{c}^{-1}$ \autocite{RHIC_luminosity}. Поляризация пучка является дополнительной степенью свободы, определенные сечения рассеяния приобретают зависимость от поляризации сталкивающихся сгустков. Поскольку соотношение заряда к массе для протона отличается по сравнению с тяжелыми ионами почти в два раза, то максимальная энергия эксперимента кратно увеличивается. В структуре, оптимальной для тяжелоионного эксперимента, подобрано значение критической энергии таким образом, что столкновение происходит до критического значения и никаких проблем по её преодолению не возникает. Критическая энергия является важным параметром ускорительный установки и при проектировании структуры этому вопросу уделяется особое внимание. Таким образом, для протонов прохождение критической энергии становится важным параметром, ограничивающем параметры сгустка, и требует принятия дополнительных мер для её преодоления.

\par	При ускорении протонного пучка относительно длительное нахождение вблизи критической энергии или её пересечение существенно влияет на параметры пучка и его стабильность. Нарушается адиабатичность продольного движения, существенными становятся нелинейные эффекты, затухание Ландау оказывается неспособно подавить возникающие возмущения, а наличие пространственного заряда и других импедансов оказывает влияние на развитие продольной микроволновой неустойчивости, нестабильности отрицательной массы и поперечной голова-хвост (head-tail) \autocite{ng, lee}. В случае малых интенсивностей критическая энергия не оказывает значительного влияния на параметры сгустка. Проблема прохождения критической энергии является характерной для интенсивных сгустков с количеством частиц порядка $10^{10}-10^{12}$. Высокая интенсивность в коллайдерных экспериментах, обусловлена требованиями по достижению высокого значения светимости. Однако, любое увеличение эмиттанса пучка приводит к снижению конечной светимости эксперимента. Влияние всех приведенных эффектов, ограничено временем нахождения вблизи критической энергией, в этой связи используются методы быстрого пересечения или поднятия критической энергии.

\par	Известная проблема физики состоит в объяснении барионной асимметрии, то есть наблюдаемым преобладанием материи над антиматерией. До сих пор, существующие физические законы не способны полностью объяснить такой дисбаланс. В работе 1967 год А. Д. Сахаровым были сформулированы общие необходимые условия для наличия барионной асимметрии: 1) Нарушение закона сохранения барионного заряда; 2) Нарушение C- и CP-симметрии; 3) Нарушение на ранних этапах формирования Вселенной термодинамического равновесия \autocite{sakharov}. Согласно второму условию, "\textit{Возникновение С-асимметрии по нашей гипотезе является следствием нарушения СР-инвариантности при нестационарных процессах расширения горячей Вселенной на сверхплотной стадии, которое проявляется в эффекте различия парциальных вероятностей зарядово-сопряженных реакций}".  Ранее в 1958 году С. Окубо теоретически показал такой эффект при рассмотрении распада сигма гиперона $\Sigma^{+}$ и его античастицы $\bar{\Sigma}^{+}$. Позднее в 1964 году Д. Кронин и В. Фитч экспериментально обнаружили нарушение CP-инвариантности слабого взаимодействия в процессах распада нейтральных каонов $K_{2}^{0}$ на два пиона $\pi^{+}, \pi^{-}$ \autocite{CP}, за что в 1980 году были удостоены Нобелевской премии по физике. 

\par	В современной Стандартной модели частиц P-\autocite{P-violation} и CP-симметрии нарушаются. Источником CP-нарушения является наличие комплексной фазы в матрице смешивания кварков Кабиббо-Кабаяси-Маскава для слабых взаимодействий \autocite{CKM} и коэффициента $\theta_{\text{QCD}}$ в лагранжиане квантовой хромодинамики \autocite{CPstrong}, однако не обнаружено CP-нарушений в сильных взаимодействиях. Согласно CPT-теореме, CP-инвариантность эквивалентна T-инвариантности. Источником такого нарушения может являться ненулевой электрический дипольный момент (ЭДМ) элементарных частиц, фундаментальное свойство материи и обусловленное неоднородностью распределения заряда внутри частицы. Поскольку ЭДМ представляется полярным вектором, а не псевдовектором, то для него нарушается как P-, так и T-инвариантность, что показано на рис. \ref{fig:4edmpt}.  Величина ЭДМ в Стандартной Модели слишком мала для экспериментального детектирования и находится на уровне $\abs{d_{n}}< 10^{-30}-10^{-32}$ $e\cdot \text{см}$ для нейтрона \autocite{EMD_overview}. Возможность его существования была сформулирована в заметке 1950 Перселл и Рэмси \autocite{EDM}, однако ненулевое ЭДМ пока точно не обнаружено. Другие теоретические модели, такие как Суперсимметричные (SUSY), также предсказывают наличие ЭДМ, но на уровне $\abs{d_{n}}< 10^{-27}-10^{-29}$ $e\cdot \text{см}$ для нейтрона, которые оставляют существенную надежду на экспериментальное обнаружение. Стоит отметить, что и таких точностей пока достигнуто не было, а сделаны только существенные ограничения для нейтрального нейтрона, впервые появившиеся в работе Н. Рамси и его коллег $\abs{d_{n}}< 5\times10^{-20}$ $e\cdot \text{см}$ ($90\%$ C.L.) \autocite{NeutronEDM}, текущее ограничение находится на уровне $\abs{d_{n}}< 1.8\times 10^{-26}$ $e\cdot \text{см}$ ($90\%$ C.L.), что получено в работе nEDM \autocite{neutron_EDM_current}.

\begin{figure}
	\centering
	\includegraphics[width=0.4\linewidth]{images/4_EDM_P_T}
	\caption{Схематическое изображение нарушение P- и Т-симметрии ненулевым электрическим дипольным моментом.}
	\label{fig:4edmpt}
\end{figure}

\par	Исследование ЭДМ осуществляется согласно уравнению Т-БМТ по его влиянию на поведение поляризации в электромагнитных полях. В случае ЭДМ нейтрона, а также нейтральных атомов их положение сохраняется при действии внешних магнитных и электрических полей. В случае заряженных частиц происходит движение согласно силе Лоренца, что и приводит к необходимости применения ускорительных установок, позволяющих длительное накопление пучка с заданными параметрами и выступающих в роли накопительного кольца. Наиболее интересным и перспективным направлением выглядит изучение ЭДМ протона и дейтрона. Для этого требуется создание пучков поляризованных частиц с максимально близкими свойствами с точки зрения прецессии спина во внешних полях. Тогда сохраняется поляризация вдоль конкретной оси, а также спины частиц прецессируют с одинаковой частотой. Для накопления величины ЭДМ на уровне $10^{-29}$ $e\cdot \text{см}$ необходимо удерживать пучок на орбите с сохранением поляризации течение времени $\sim1000$ секунд с последующим анализом рассеяния на мишени поляриметра. При этом влияние магнитного дипольного момента (МДМ) должно быть подавлено до величины, меньшей сигнала ЭДМ. Такая техника впервые была предложена в БНЛ (Брукхейвенская Национальная Лаборатория) и имеет название 'замороженный' спин \autocite{Farley:edm}. Позднее, была предложена концепция ’квази-замороженного’ спина \autocite{QFS}, в которой происходит пространственное разделение электрического и магнитного полей и условия подавления влияния МДМ-компоненты за полный оборот по кольцу.

\par	Ещё одним перспективным направлением исследований в рамках программы спиновой физики является поиск аксионоподобных частиц. При этом изучается резонанс при совпадении $g-2$ частоты спиновой прецессии вокруг ведущего магнитного поля ускорителя с частотой колебаний аксионного поля. Для этого ускоритель будет использован в роли широкополосной зондирующей антенны по частоте прецессии спина \autocite{Axion_Nikolaev}.

\par	Приведённые вопросы фундаментальной физики подлежат детальному рассмотрению и могут быть исследованы с использованием ускорительных установок, предназначенных для проведения разнообразных экспериментов. Эти установки позволяют достигать высоких энергий частиц, а также предельных точностей измерений. Такая практика применяется в крупных мировых ядерных центрах: CERN \autocite{lhc:heavy_ions}, BNL \autocite{rhic:design}, J-PARC \autocite{j-park}. 

\par	Ускорительный комплекс NICA (Nuclotron-based Ion Collider fAсility) является современным передовым центром, который оборудован передовой материально-технической базой, отвечающей мировым тенденциям и формируется на базе ОИЯИ в городе Дубна, Россия \autocite{nuclotron24}. Основной установкой комплекса является коллайдер, в котором предусмотрены два места встречи пучков, где расположены два детектора: MPD (Multi-Purpose Detector) и SPD (Spin Physics Detector) \autocite{Ladygin:SPD}. Каждый из этих детекторов предназначен для различных экспериментов. MPD-детектор будет использован для исследования кварк-глюонной плазмы, возникающей в результате столкновения тяжёлых ионов \autocite{Tech, MPD}. SPD-детектор направлен на изучение поведения сталкивающихся поляризованных пучков протонов и дейтронов. Кинематическая область, охватываемая SPD, уникальна и никогда не использовалась целенаправленно при поляризованных адронных столкновениях. Кроме того, уникальной возможностью станет изучение поляризованных дейтронов. Таким образом, структура коллайдера должна поддерживать ускорение как тяжёлых ионов, так и лёгких частиц. При этом требования к удержанию пучков для различных типов частиц существенно отличаются.

\par Исследования направлены на формирование полноценной физической программы. В работе отдается приоритет исследованиям свойств пучков частиц, где ускоритель выступает в роли детектирующего устройства и позволяет решить поставленные задачи фундаментальной физики в комплексе Nuclotron-NICA. Применение изложенных в работе подходов возможно и на других похожих установках без потери общности.
~\\
\par {\aim} данной диссертации является изучение особенностей поведения лёгких поляризованных пучков в предлагаемой дуальной структуре, а также исследования электрического дипольного момента с использованием квази-замороженной концепции.
Для достижения поставленной цели необходимо было 
решить следующие {\tasks}:
\begin{enumerate}[beginpenalty=10000] % https://tex.stackexchange.com/a/476052/104425
	\item	Определение требований к дуальной структуре для тяжелых ионов;
	\item	Регулирование критической энергии для поляризованных частиц методом резонансной модуляции дисперсионной функции;
	\item	Проведение численного моделирования динамики пучка легких частиц с учетом высших порядков коэффициента уплотнения орбиты в высокочастотых резонаторах гармонического и барьерного типа;
	\item 	Изучение поведения динамической апертуры с учетом высших порядков при прохождении пучка через критическую энергию с скомпенсированной хроматичностью;
	\item 	Определение особенностей поведения поляризации пучка при совершении процедуры скачка критической энергии;
	\item 	Изучение концепции «квази-замороженного» спина с целью создания установки для исследования ЭДМ дейтрона и протона;
	\item	Исследование спин-орбитального движения поляризованного пучка в магнитном кольце с фильтрами Вина;
\end{enumerate}
~\\
\par {\novelty}
\begin{enumerate}[beginpenalty=10000] % https://tex.stackexchange.com/a/476052/104425
	\item	Впервые предложена дуальная структура для тяжелых ионов и лёгких частиц для коллайдера NICA;
	\item 	Впервые предложены методы подавления дисперсии поворотной аркой в резонансной магнитооптической структуре с отсутствующими магнитами;
	\item 	Впервые исследован метод скачка критической энергии с использованием барьерного ускоряющего потенциала с учётом ограничений по продольной микроволновой неустойчивости;
	\item	Были проведены исследования продольной динамики с учётом высших порядков разложения по импульсу, а также влиянием импеданса. На их базе сформулированы ограничения на величину и темп скачка критической энергии;
	\item	Были разработаны 8- и 16-периодичная квази-замороженная структура Nuclotron для выделения ЭДМ сигнала легких ядер;
	\item	Была разработана структура коллайдера NICA с обводными каналами, неориентированная изначально на эксперименты по поиску ЭДМ дейтрона методом квази-замороженного спина;
\end{enumerate}
~\\
\par {\influence}:
\par Исследование динамики пучка вблизи критической энергии показывает необходимость её преодоления, а также  способствует определению оптимальных параметров скачка критической энергии. Определено существенное влияние критической энергии на динамику поляризованного протонного пучка.

\par В качестве решения проблем с тяжелыми и легкими частицами разработана дуальная структура. Дуальность структуры указывает на возможность её эффективного решения обоих эффектов и использования сразу для двух фундаментально значимых исследований: для изучения кварк-глюонной плазмы в коллайдерных экспериментах с тяжелыми ионами и для исследования легких поляризованных пучков в симметричных и асимметричных коллайдерных столкновениях.

\par Расширена применимость метода «резонансных структур» для случая отсутствия периодичности дисперсионной функции на арках.

\par В части поляризованных частиц адаптирован метод "квази-замороженного спина" для коллайдера NICA, продемонстрировавший возможность проведения исследований по электрическому дипольному моменту без значительных изменений структуры ускорителя. Разработана магнитооптическая структура обводных каналов bypass, позволяющая обойти точки встречи для обоих детекторов с расположенными на них прямыми фильтрами Вина. Данная схема позволяет реализовать концепцию "квази-замороженного спина" для исследования электрического дипольного момента дейтрона. 

\par Методы разработанные для кольца NICA могут быть использованы и для Nuclotron с сохранением функций бустера для поляризованного пучка в коллайдер. Это также позволит проведение независимых экспериментов по исследованию ЭДМ и поиску аксиона. Такие исследования является отдельной частью программы спиновой физики, которая формируется на уста­новке NICA-Nuclotron.

%\par {\progress}
%Разработанность темы ТАКАЯ
%~\\

\par {\methods} Основными методами исследования являются математическое и компьютерное моделирование, численный эксперимент. Для исследования были использованы программы для расчёта поперечной динамики: MAD-X \autocite{madx}, OPTIM \autocite{optim}, BMAD \autocite{bmad}, продольной динамики: BLonD \autocite{blond}; спин-орбитальной динамики: COSY Infinity \autocite{cosy}.
~\\

\par {\defpositions}
\begin{enumerate}[beginpenalty=10000] % https://tex.stackexchange.com/a/476052/104425
	\item 	Предложена реализация дуальной структуры для комплекса NICA-Nuclotron, оптимальная для тяжелых частиц с точки зрения внутрипучкового рассеяния и легких частиц с поднятой критической энергией выше энергии эксперимента; \autocite{Kolokolchikov:2025_dual, Syresin:2021_polar}
	\item	Реализован метод вариации критической энергии для магнитооптики коллайдера NICA с отсутствующими магнитами при подавлении дисперсионной функции двумя семействами квадруполей и двумя крайними ячейками поворотной арки; \autocite{Kolokolchikov:2021trans, Kolokolchikov:2023_pecular}
	\item	Представлены результаты численного моделирования продольной динамики с учетом влияния высших порядков разброса по импульсам и моделей продольных импедансов в окрестности критической энергии и сравнение с экспериментальными результатами, полученными на У-70; \autocite{Kolokolchikov:2025_U70, Kolokolchikov:2025_jump}
	\item 	Проведен анализ использования гармонического ВЧ при процедуре скачка в коллайдере NICA. Для барьерного ВЧ представлены данные моделирования продольной динамики, а также предложено сокращение длины между барьерами из-за продольной микроволновой неустойчивости; \autocite{Kolokolchikov:2024_bb_rupac, Kolokolchikov:2023_bb_IPAC, Kolokolchikov:2024_bb_dspin}
	\item	Предложены модернизированные 8/16-периодичные структуры Nuclotron с квази-замороженным спином для исследования электрического дипольного момента легких ядер, с сохранением функции бустера; \autocite{Senichev:2023_QFS, Senichev:2023_nuclotron, Kolokolchikov:2025_nuclotron}
	\item	Применен метод фильтров Вина для сохранении направления поляризации на основе введения обводных каналов в структуре коллайдера NICA с квази-замороженным спином для выделения ЭДМ сигнала в поляризованном пучке дейтронов; \autocite{Kolokolchikov:2023_bypass_ru, Kolokolchikov:2023_bypass_IPAC, Senichev:2024_nica_edm, Kolokolchikov:2023_sc, Kolokolchikov:2023_sc_IPAC}
\end{enumerate}

~\\
\par {\reliability} полученных результатов подтверждается согласованием аналитических вычислений с результатами численных экспериментов. Результаты находятся в соответствии с результатами, полученными другими авторами.
~\\
\par {\probation}
Основные результаты работы были представлены~на российских и международных конференциях, а также были представлены на рабочих встречах: 
\begin{itemize}
\item Workshop “Polarized beam in NICA” в 2022 г.;
\item Молодежная конференция по теоретической и экспериментальной физике МКТЭФ-2020. Москва, Россия;
\item 63, 65, 66-ая Всероссийская научная конференция МФТИ в 2020, 2023, 2024 гг. г. Долгопрудный,
Россия;
\item XXVII и XXVIII Всероссийская конференции по ускорителям заряженных частиц RuPAC'21, RuPAC'23. Алушта; Новосибирск, Россия;
\item VII, VIII, IX и X Международная конференция Лазерные и Плазменные технологии ЛаПлаз'21, ЛаПлаз'22, ЛаПлаз'23, ЛаПлаз'24, ЛаПлас'25. Москва, Россия;
\item XIII, XIV, XVI международная конференция по ускорителям заряженных частиц IPAC'22 IPAC'23, IPAC'25. Бангкок, Тайланд; Венеция, Италия; Тайпей, Тайвань;
\item XIX Международная конференции по спиновой физике высоких энергий DSPIN'23. Дубна, Россия;
\item XI-я Международная конференция по ядерной физике в накопительных кольцах STORI’24. Хуэйчжоу, провинция Гуандун, Китай;
\end{itemize}
~\\
\par {\contribution} Все результаты, выносимые на защиту, получены автором лично, либо при его непосредственном участии. Содержание диссертации и выносимые на защиту основные положения отражают личный вклад автора в опубликованные работы. Результаты по подготовке и проведению эксперимента на ускорителе У-70 получены в соавторстве с сотрудниками ИЯИ РАН и ИФВЭ. Подготовка к публикации полученных результатов проводилась совместно с соавторами.
~\\

\par \ifnumequal{\value{bibliosel}}{0}
{%%% Встроенная реализация с загрузкой файла через движок bibtex8. (При желании, внутри можно использовать обычные ссылки, наподобие `\autocite{vakbib1,vakbib2}`).
	{\publications} Основные результаты по теме диссертации изложены
	в~XX~печатных изданиях,
	X из которых изданы в журналах, рекомендованных ВАК,
	X "--- в тезисах докладов.
}%
{%%% Реализация пакетом biblatex через движок biber
	\begin{refsection}[bl-author, bl-registered]
		% Это refsection=1.
		% Процитированные здесь работы:
		%  * подсчитываются, для автоматического составления фразы "Основные результаты ..."
		%  * попадают в авторскую библиографию, при usefootcite==0 и стиле `\insertbiblioauthor` или `\insertbiblioauthorgrouped`
		%  * нумеруются там в зависимости от порядка команд `\printbibliography` в этом разделе.
		%  * при использовании `\insertbiblioauthorgrouped`, порядок команд `\printbibliography` в нём должен быть тем же (см. biblio/biblatex.tex)
		%
		% Невидимый библиографический список для подсчёта количества публикаций:
		\printbibliography[heading=nobibheading, section=1, env=countauthorvak,          keyword=biblioauthorvak]%
		\printbibliography[heading=nobibheading, section=1, env=countauthorwos,          keyword=biblioauthorwos]%
		\printbibliography[heading=nobibheading, section=1, env=countauthorscopus,       keyword=biblioauthorscopus]%
		\printbibliography[heading=nobibheading, section=1, env=countauthorconf,         keyword=biblioauthorconf]%
		\printbibliography[heading=nobibheading, section=1, env=countauthorother,        keyword=biblioauthorother]%
		\printbibliography[heading=nobibheading, section=1, env=countregistered,         keyword=biblioregistered]%
		\printbibliography[heading=nobibheading, section=1, env=countauthorpatent,       keyword=biblioauthorpatent]%
		\printbibliography[heading=nobibheading, section=1, env=countauthorprogram,      keyword=biblioauthorprogram]%
		\printbibliography[heading=nobibheading, section=1, env=countauthor,             keyword=biblioauthor]%
		\printbibliography[heading=nobibheading, section=1, env=countauthorvakscopuswos, filter=vakscopuswos]%
		\printbibliography[heading=nobibheading, section=1, env=countauthorscopuswos,    filter=scopuswos]%
		%
		\nocite{*}%
		%
		{\publications} Основные результаты по теме диссертации изложены в~\arabic{citeauthor}~печатных изданиях, 13 из которых изданы в журналах, рекомендованных ВАК\sloppy%
		\ifnum \value{citeauthorscopuswos}>0%
		, \arabic{citeauthorscopuswos} "--- в~периодических научных журналах, индексируемых Web of~Science и Scopus\sloppy%
		\fi%
		\ifnum \value{citeauthorconf}>0%
		, \arabic{citeauthorconf} "--- в~тезисах докладов.
		\else%
		.
		\fi%
		\ifnum \value{citeregistered}=1%
		\ifnum \value{citeauthorpatent}=1%
		Зарегистрирован \arabic{citeauthorpatent} патент.
		\fi%
		\ifnum \value{citeauthorprogram}=1%
		Зарегистрирована \arabic{citeauthorprogram} программа для ЭВМ.
		\fi%
		\fi%
		\ifnum \value{citeregistered}>1%
		Зарегистрированы\ %
		\ifnum \value{citeauthorpatent}>0%
		\formbytotal{citeauthorpatent}{патент}{}{а}{}\sloppy%
		\ifnum \value{citeauthorprogram}=0 . \else \ и~\fi%
		\fi%
		\ifnum \value{citeauthorprogram}>0%
		\formbytotal{citeauthorprogram}{программ}{а}{ы}{} для ЭВМ.
		\fi%
		\fi%
		% К публикациям, в которых излагаются основные научные результаты диссертации на соискание учёной
		% степени, в рецензируемых изданиях приравниваются патенты на изобретения, патенты (свидетельства) на
		% полезную модель, патенты на промышленный образец, патенты на селекционные достижения, свидетельства
		% на программу для электронных вычислительных машин, базу данных, топологию интегральных микросхем,
		% зарегистрированные в установленном порядке.(в ред. Постановления Правительства РФ от 21.04.2016 N 335)
	\end{refsection}%
	\begin{refsection}[bl-author, bl-registered]
		% Это refsection=2.
		% Процитированные здесь работы:
		%  * попадают в авторскую библиографию, при usefootcite==0 и стиле `\insertbiblioauthorimportant`.
		%  * ни на что не влияют в противном случае
		%\nocite{vakbib2}%vak
		%\nocite{patbib1}%patent
		%\nocite{progbib1}%program
		%\nocite{bib1}%other
		%\nocite{confbib1}%conf
	\end{refsection}%
	%
	% Всё, что вне этих двух refsection, это refsection=0,
	%  * для диссертации - это нормальные ссылки, попадающие в обычную библиографию
	%  * для автореферата:
	%     * при usefootcite==0, ссылка корректно сработает только для источника из `external.bib`. Для своих работ --- напечатает "[0]" (и даже Warning не вылезет).
	%     * при usefootcite==1, ссылка сработает нормально. В авторской библиографии будут только процитированные в refsection=0 работы.
}
