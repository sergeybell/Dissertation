%% Согласно ГОСТ Р 7.0.11-2011:
%% 5.3.3 В заключении диссертации излагают итоги выполненного исследования, рекомендации, перспективы дальнейшей разработки темы.
%% 9.2.3 В заключении автореферата диссертации излагают итоги данного исследования, рекомендации и перспективы дальнейшей разработки темы.
\begin{enumerate}

  \item На основе анализа внутрипучкового рассеяния, а также стохастического охлаждения показано, что варьирование коэффициента проскальзывания (slip-factor) в резонансной структуре способно увеличить эффективность стохастического охлаждения. Особенно эффективным может быть использование комбинированной структуры. Однако, эффекты ВПР для приведенных структур оказались в несколько раз большими и в конечном счёте недостаточными, делая предпочтительной регулярную структуру для тяжелоионного эксперимента с минимально модулированным Твисс-функциями;
  
  \item Для коллайдерных экспериментов с поляризованными протонами резонансная структура позволяет поднять критическую энергию выше энергии эксперимента, путем искажения дисперсионной функции. Такой подход не требует существенных затрат и делает возможным реализацию дуальной структуры для двух полноценных физических программ;
  
  \item Численные исследования показали, что прохождение критической энергии может вызывать нестабильность продольного фазового движения. Использование процедуры скачка критической энергии может быть использовано для преодоления этой проблемы. Получены экспериментальные данные процедуры скачка критической с синхротрона У-70, которые находятся в соответствии с проведенным численными оценками с учетом высших порядков разложения коэффициента уплотнения орбиты и импедансов для различных интенсивностей сгустка;
  
  \item Использование процедуры скачка критической энергии может быть ограничено его величиной, а также для темпом изменения градиентов в квадруполях арки по сравнению с темпом ускорения пучка. Рассмотрено различие применения гармонического и барьерного ВЧ на особенности рассмотренного скачка. Приведены оценки продольной микроволновой неустойчивости, показывающие существенное ограничение на параметры сгустка.
  
  \item Исследована спиновая динамики для возможности изучения ЭДМ заряженных частиц. Реализована концепция квази-замороженного спина с введением обводных каналов и сохранением изначального предназначения установки. На прямых участках предлагается расположение фильтров Вина для компенсации поворота спина под действием МДМ в магнитной арке.
  
  \item Рассмотрена модернизированная структура синхротрона Nuclotron с сохранением функции бустера поляризованного пучка в коллайдер NICA. В предложенных 8/16-периодичных структурах возможно проведение независимых прецизионных экспериментов по исследованию ЭДМ дейтрона и протона, а также осуществлению поиска аксиона в режиме сканирующей антенны.
  
\end{enumerate}
