%% Согласно ГОСТ Р 7.0.11-2011:
%% 5.3.3 В заключении диссертации излагают итоги выполненного исследования, рекомендации, перспективы дальнейшей разработки темы.
%% 9.2.3 В заключении автореферата диссертации излагают итоги данного исследования, рекомендации и перспективы дальнейшей разработки темы.
\begin{enumerate}

  \item На основе анализа внутрипучкового рассеяния, а также стохастического охлаждения показано, что варьирование коэффициента проскальзывания (slip-factor) в резонансной структуре способно увеличить эффективность стохастического охлаждения. Особенно эффективным может быть использование комбинированной структуры. Однако, эффекты ВПР для приведенных структур оказались в несколько раз большими и в конечном счёте недостаточными, делая предпочтительной регулярную структуру для тяжелоионного эксперимента с минимально модулированным Твисс-функциями;
  
  \item Для коллайдерных экспериментов с поляризованными протонами резонансная структура позволяет поднять критическую энергию выше энергии эксперимента, путем искажения дисперсионной функции. Такой подход не требует существенных затрат и делает возможным реализацию дуальной структуры для двух полноценных физических программ;
  
  \item В новой конфигурации кольца проведено численное исследование продольной динамики с учётом высших порядков разложения по импульсу и влияния импеданса. Экспериментальные данные пересечения критической энергии на синхротроне У-70 подтвердили соответствие численных оценок для различных интенсивностей сгустка;
  
  \item Использование процедуры скачка критической энергии ограничено его величиной, а также темпом изменения градиентов в квадруполях арки по сравнению с темпом ускорения пучка. Рассмотрено влияние гармонического и барьерного ВЧ на особенности рассмотренного скачка. Приведены оценки продольной микроволновой неустойчивости, показывающие существенное ограничение на параметры сгустка;
  
  \item Исследована спиновая динамики для возможности изучения ЭДМ заряженных частиц. Показана потенциальная возможность реализации концепции квази-замороженного спина с введением обводных каналов и сохранением изначального предназначения установки на основе моделирования. На прямых участках предлагается расположение фильтров Вина для компенсации поворота спина под действием МДМ в магнитной арке;
  
  \item Рассмотрена модернизированная структура синхротрона Nuclotron с сохранением функции бустера поляризованного пучка в коллайдер NICA. В предложенных 8/16-периодичных структурах возможно проведение независимых прецизионных экспериментов по исследованию ЭДМ дейтрона и протона.
  
\end{enumerate}

\par Проведенные исследования показывают особенности применения ускорительной техники для фундаментальных экспериментов.

\par В качестве ближайших исследований влияния критической энергии на параметры пучка предполагается изучение метода скачка критической энергии, включающее детальное рассмотрение продольной динамики с учётом точных данных по импедансу кольца с использованием экспериментально измеренных данных на действующем коллайдере.

\par Одновременно для будущей реализации концепции дуальной структуры для легких и тяжелых частиц планируется введение схемы раздельного питания квадруполей на установке коллайдера NICA для создания резонансной структуры с целью исключения прохождения критической энергии для протонного пучка во всем диапазоне энергий ускоренного пучка. 

\par В части поляризованных пучков важной задачей, вытекающей из результатов диссертации, будет применение методики измерения ЭДМ. Она включает в себя вопрос достижения высокого времени когерентности, необходимой для детектирования сигнала с использованием поляриметра. 

\par В части развития магнитооптической структуры комплекса NICA-Nuclotron будут рассмотрены возможности создания отдельного кольца с замороженным или квази-замороженным спином. 





